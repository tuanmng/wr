
\chapter{ESTIMATING THE RATES OF ELECTRON CHARGE
\\MIS-IDENTIFICATION}\label{c:cid}

Many physics analyses involve charged leptons in their final states, where
leptons typically refer to electrons or muons. Such an electron or a muon
leaves a track in the detector, and its charge, or more specifically the sign
of the charge, is determined from the curvature --- due to the installed
magnetic fields --- of the track. Because of some factors which will be
discussed below, this determination could occasionally be erroneous, leading to
what is called charge misidentification.

Electron charge mis-identification is important for analyses that involve
same-sign electrons in the final state. Examples of such analyses include
measurements of same-sign WW scattering \cite{sswwscatting}, analyses that
involve the production of a Higgs in association with a $t\tbar$ pair ($t\tbar
H$), and supersymmetry search with two same-sign leptons \cite{ssssleptons}. In
general, charge mis-identification rates occur on the order of $\text{O}(1\%)$,
while Standard Modle processes that provide opposite-sign dileptons (dominantly
$Z\to e^+e^-$ bosons) occur approximately $10^3$ times more commonly than
genuine Standard Model sources of same-sign leptons (dominantly $WZ$
production). As a result, opposite-sign sources of dileptons suffering from
charge mis-identification can constitute a large background in these searches,
and so it is crucial to estimate the charge mis-identification background
precisely.



This chapter describes a method for estimating the rate of charge
mis-identification using a likelihood function. Section~\ref{s:chargereasons}
discusses briefly how electron charge mis-identification might arise at ATLAS.
Section~\ref{s:cllhmethod} discusses the likelihood method, including the
Poisson likelihood used as well as how it is applied to $Z\to e^+e^-$ events to
measure the charge mis-identification rates. Finally, Section~\ref{s:chargecon}
provides some conclusions.

It is to be noted that muon charge mis-identification is known to be
negligible, except at very high $p_T$. Indeed, the magnetic field in the Muon
Spectrometer (see Section~\ref{s:decmuons}) allows the measurement of the track
curvature over a larger radius, thereby reducing the chance the charge could be
mis-identified. Analyses such as the supersymmetry search with two same-sign
leptons mentioned above have found that muon charge mis-identification is in
indeed negligible ({\color{pink} cite}).



\section{Electron Charge Mis-identification}\label{s:chargereasons}

At ATLAS, the sign of the charge of an electron is determined from its track in
the Inner Detector (see Section~\ref{s:decinner}). Indeed, as the electron
passes through the Inner Detector , its track is bent by the installed magnetic
fields. The direction of the curvature of the track determines the charge of
the electron.

Charge mis-identification, where the charge of the electron is identified
incorrectly, occurs mainly because of two reasons:

\begin{itemize}[label=\ding{111}]

	\item As the electron passes through the detector and interacts with the
	      materials in the detector, it may radiate photons. These radiated photons may
	      in turn convert to electron-positron pairs. A charge mis-identification occurs
	      when the electron candidate is matched to the wrong track.

	\item The reconstructed track of the electron appears rather straight, i.e. the
	      curvature of the track is small, at very high momentum or at large
	      pseudorapidity, the latter because the lever arm of the tracker is limited.

\end{itemize}

%%%%%%%%%%%%%%%%%%%%%%%%%%%%%%%%%%%%%%%%%%%%%%%%%%%%%%%%
%%%%%%%%%%%%%%%%%%%%%%%%%%%%%%%%%%%%%%%%%%%%%%%%%%%%%%%%
\section{The Likelihood Method}\label{s:cllhmethod}


Since it is impossible to know with absolute certainty, after the charge of an
electron has been measured, that a charge mis-identification has occured or
not, we seek instead to determine the rates of charge mis-identification for an
ensemble of electrons. Essentially, we start from a sample of electrons for
which the true charges are known. Charge measurements on the sample will result
in another sample that consists of electrons with the original charges as well
as elecrons whose charges have switched. A key step is to write down the
probababilistic distribution of charge assuming a rate of charge
mis-identification, and then seek to determine this rate from the actual data.
This method is called the likelihood method and will be discussed in section.

This chapter uses $Z\to e^+e^-$ events because these provide a good source of
clean, high-statistics sample of electrons.


\subsection{The Poisson Likelihood}\label{s:cpllh}

Consider a pair of electrons $e^+e^-$ (an opposite-sign pair) at truth level.
The charges of the electrons are opposite of each other, but because of charge
mis-identification there is a chance of this pair being identified as having
the same charge (a same-sign pair). Assuming a probability $p$ of such a
chance, then in considering $n$ pairs $e^+e^-$, the probability of seeing
exactly $n_{ss}$ same-sign pairs is given by the binomial distribution

$$
	P(n_{ss}) = \binom{n}{n_{ss}}p^{n_{ss}}(1-p)^{n-n_{ss}}.
$$

Since it is known that the charge mis-identification probability $p$ is
typically small while the number of pairs considered $n$ is typically very
large , the Poisson distribution may be used to approximate the binomial
distribution. Thus, let

\begin{equation}\label{eq:cmss}
	m_{ss} = np
\end{equation}

denote the expected number  of same-sign pairs, it follows that



\begin{equation}\label{eq:cpoissonn}
	P(n_{ss}) = \frac{m_{ss}^{n_{ss}} e^{-m_{ss}}}{n_{ss}!}
\end{equation}

is the Poisson probability of seeing $n_{\text{ss}}$ same-sign pairs, given the
average number of same-sign pairs $m_{\text{ss}}$. This will also be called a
likelihood function.

Consider again an opposite-pair $e^+e^-$ with the probability $p$ of being
identified as a same-sign pair. We may speak directly of a probability
$\epsilon$ of an electron in the pair having its charge incorrectly identified.
Then a same-sign pair results if only one of the electrons in the original
opposite-sign pair has its charge identified incorrectly, which means we may
write

\begin{equation}\label{eq:cprobpair}
	p = (1-\epsilon)\epsilon +  \epsilon(1-\epsilon).
\end{equation}

The Poisson likelihood of Equation~\ref{eq:cpoissonn} may now be written to
depend explicitly on $\epsilon$:

\begin{equation}\label{eq:cpoisson}
	P(n_{ss}|\epsilon) = \frac{m_{ss}^{n_{ss}} e^{-m_{ss}}}{n_{ss}!}, \quad
	m_{ss} = np = n(1-\epsilon)\epsilon +  \epsilon(1-\epsilon).
\end{equation}

In an actual measurement of the rates of charge mis-identification, the
individual rates $\epsilon$'s are in general different for the electrons in the
pair, if for example the dependence of $\epsilon$ on the transverse momenta
$p_T$ is taken into account. We may introduce then a number of bins in $p_T$
and may speak of a charge mis-identification probability associated with a bin
$i$. Consequently an electron pair is associated with a pair of bins $(i,j)$,
and we have the following quantities:

\begin{itemize}[label=\ding{109}]
	\item The probability

	      \begin{equation}\label{eq:cijw}
		      p_{ij} = (1-\epsilon_i)\epsilon_j +  \epsilon_i(1-\epsilon_j)
	      \end{equation}

	      in place of the probability $p$ in Equation~\ref{eq:cprobpair}. This is the
	      probability an opposite-sign pair
	      may be seen as a same-sign pair in the bin pair $(i,j)$

	\item The number of electron pairs considered, $n_{ij}$,  in the bin pair $(i,j)$

	\item The expected number of same-sign pairs

	      \begin{equation}\label{eq:cmssij}
		      m_{\text{ss,ij}} = n_{ij} p_{ij}
	      \end{equation}

	      in place of the expected number of same-sign pairs in Equation~\ref{eq:cmss}


	\item The Poisson likelihood

	      \begin{equation}\label{eq:cpoissonij}
		      P(n_{ss,ij} | \epsilon_i, \epsilon_j) = \frac{m_{ss,ij}^{n_{ss,ij}} e^{-m_{ss,ij}}}{n_{ss,ij}!}
	      \end{equation}

	      in place of the Poison likelihood in Equation~\ref{eq:cpoisson}. This will also
	      be denoted simply as $L_{ij}$

\end{itemize}


These equations remain the same if instead of one-dimensional bins, two
dimensional bins are used. Indeed, suppose the dependency of the charge
mis-identification rates on, say $p_T$ and $\eta$, needs to be taken into
account. If there are $a$ bins in $p_T$ and $b$ bins in $\eta$, the total
number of bins $ab$ could be labelled $1, 2, \cdots, ab$ and treated as an
ordered sequence of one-dimensional bins. The actual $p_T$ and $\eta$ binning
together with the estimation of the charge mis-identificaion rates will be
discussed in the following section.

All the possible bin pairs $(i,j)$ need to be used and therefore, assuming
statistically-independent rates, we will maximize the likelihood function

$$
	L = \prod_{i,j}L_{ij}
$$

to find the rates $\epsilon_i$, the data being $n_{ij}$, the numbers of
electrons observed in the bin pair $(i,j)$, and $n_{ss,ij}$, the number of
same-sign electron pairs observed in the bin pair $(i,j)$.


\subsection{Estimation of the Rates on $Z\to e^+e^-$ sample}\label{s:zeerates}

The Poisson likelihood is applied on a $Z\to e^+e^-$ data sample to determine
the charge mis-identification rates as follows. First, several selections are
applied, including

\begin{itemize}[label=\ding{111}]

	\item Logical OR between two single-electron triggers, one with $E_T > 24$ GeV
	      plus Medium identification, one with $E_T > 60$ GeV plus Loose identification

	\item At least two electron candidates with $\abs{\eta} < 2.47$

	\item One electron is required to pass the Tight identification requirement,
	      and to have $E_T > 25$ GeV. The other electron must have $E_T > 10$ GeV and
	      must satisfy the track quality criteria (the tracks associated with the
	      electron must have at least one hit in the pixel detector and at least seven
	      hits in the pixel and SCT detectors)

	\item The invariant mass is within $\pm 15$ GeV of the $Z$ mass
\end{itemize}

The invariant mass of the pair of electrons will play an important role in the
following discussion. Figure~\ref{f:meea} \cite{atlaselcid} shows the invariant
mass distribution $m_{ee}$ for two different $\eta$ range, each electron having
$0.0 < \eta < 0.8$ and $2.0 < \eta < 2.47$; in both figures the each electron
in the pairs is selected to have $E$ between $25$ GeV and $50$ GeV. Due to
charge mis-identification same-sign electron pairs exist in addition to
opposite-sign pairs, and they are plotted alongside opposite-sign pairs. It is
seen that same-sign pairs have a broader peak which is also slightly shifted to
lower values, consistent with the fact that radiation which causes
charge-misidentification also causes energy loss. It is also seen that
charge-misidentification is higher at higher $\eta$, as have been commented
previously.

To continue, an invariant mass interval $(m_l,m_h)$ is selected, where $m_l=15$
GeV is the low mass point and $m_h=15$ GeV the high mass point around the $Z$
mass peak. Then electrons in the events are binned according to their $\eta$
and $p_T$ values. Each electron pair then is associated to a pair of bin
$(i,j)$, and all such bin pairs need to be taken into account. The quantities
needed are (see Section~\ref{s:cpllh}):

\begin{itemize}[label=\ding{109}]

	\item $n_{ij}$, the number of electrons counted in the bin pair $(i,j)$

	\item $n_{ss,ij}$,  the number of same-sign electron pairs counted in the bin pair $(i,j)$


\end{itemize}

There are non-signal events contamination in these quantities and to deal with
them we assume that the two sides of the interval $(m_l,m_h)$ are dominated by
background contributions, and adopt following method. To begin, in addition to
the original invariant mass interval $(m_l,m_h)$, we consider the interval
$(m_l-w_l, m_h+w_h)$ where $w_l=15$ GeV and $w_h=15$ GeV are some widths. The
latter interval is made up of three intervals:

\begin{itemize}[label=\ding{109}]
	\item $(m_l, m_h)$. This is the original invariant mass interval.

	\item $(m_l-w_l, m_l)$. This is the interval that lies to the left of the
	      original interval

	\item $(m_h, m_h+w_h)$. This is the interval that lies to the right of the
	      original interval


\end{itemize}

\begin{figure}[H]
	\includegraphics[width=6cm]{figures/fig_16a}
	\includegraphics[width=6cm]{figures/fig_16b}
	\centering

	\caption{Distribution of the invariant mass $m_{ee}$ for $E_T$ between $25$
		and $50$ GeV and $\abs{\eta} $ between $0.0$ and $0.8$ \cite{atlaselcid}. Due
		to charge mis-identification same-sign pairs as well as opposite-sign pairs
		are seen.}

	\label{f:meea}

\end{figure}


Then, in addition to the quantities $n_{ij}$ and $n_{ss,ij}$ in the original
central interval $(m_l, m_h)$, we will consider the corresponding quantities in
the two new intervals, to be denoted $n_{ij}^l$ and $n_{ss,ij}^l$ in the left
interval and $n_{ij}^h$ and $n_{ss,ij}^h$ in the right interval. We assume the
left and right intervals are background intervals and subsequently compute the
weighted quantities $b(n_{ij})$, to mean the background contamination in
$n_{ij}$, and $b(n_{ss,ij})$, to mean background contamination in $n_{ss,ij}$:

$$
	b(n_{ij}) = \frac{w_l\times n_{ij}^l + w_h\times n_{ij}^h}{w_l + w_h}, \qquad
	b(n_{ss,ij})= \frac{w_l\times n_{ss,ij}^l + w_h \times n_{ss,ij}^h }{w_l + w_h}
$$

which will be taken as the backgrounds in $n_{ij}$ and $n_{ss,ij}$ in the
central interval respectively.

The terms $n_{ij}$ and $n_{ss,ij}$ and the background terms $b(n_{ij})$ and
$b(n_{ss,ij})$ are to be used as follows. According to
Equation~\ref{eq:cpoissonij} the Poisson likelihood to be fitted is

$$
	P(n_{ss,ij} | \epsilon_i, \epsilon_j) = \frac{m_{ss,ij}^{n_{ss,ij}} e^{-m_{ss,ij}}}{n_{ss,ij}!}
$$

The background terms make a contribution to the expected number of same-sign
$m_{ss,ij}$ in the likelihood, modifying it from $m_{ss,ij} = n_{ij} p_{ij}$
(see Equation~\ref{eq:cmssij}) to

$$
	m_{ss,ij} = (n_{ij} - b(n_{ij})) \times p_{ij} + b(n_{ss,ij})
$$

The first quantity on the right in the equation above is the same-sign
contribution from signal events where the background has to be subtracted, and
the second quantity is the contribution from background events.

%%%%%%%%%%%%%%%%%%%%%%%%%%%%%%%%%%%%%%%%%%%%%%%%%%%%%%%%
\subsection{Charge Mis-identification Rates and Uncertainties}

The rates are obtained upon the maximization of the likelihood function
discussed in the previous section. The statistical uncertainties associated
with the estimated rates depend on the statistics of the data, and are given by
the statistical tool that maximizes the Poisson likelihood.

\vspace{3mm}

The following sources of systematic uncertainties are evaluated:

\begin{itemize}[label=\ding{111}]


	\item Systematic uncertainty that comes from background subtraction, which is
	      evaluated by determining the rates with and without background subtraction. The
	      inclusion of this uncertainty ensures a conservative figure of systematic
	      uncertainty in the charge mis-identification rates; it has a small impact
	      because the background is small.


	\item The invariant mass interval $(m_l,m_h)$ may be varied, from $15$ GeV
	      around the $Z$ mass to $10$ and $20$ GeV additionally. In this way an idea of
	      how the selection of an interval may affect the rates may be obtained.


	\item The invariant mass widths $w_l$ and $w_h$ may be varied, taking values
	      $20$, $25$, or $30$ GeV. This takes into account the uncertainty on the rates
	      due to the choice of a mass width.

\end{itemize}

The actual rates are estimated for the following three sets of requirements:

\begin{itemize}[label=\ding{109}]
	\item Medium: Medium identification requirements

	\item Tight + isolation: Tight identification requirements plus track isolation
	      cut $p_T^{\text{cone 0.2}} / E_T < 0.14$.

	\item Tight + isolation + impact parameter: Tight identification plus
	      $E_T^{\text{cone 0.3}} / E_T < 0.14$ and $p_T^{\text{cone 0.2}} / E_T < 0.07$
	      and additionally $\abs{z_0}\times \sin\theta < 0.5$ mm and
	      $\abs{d_0}/\sigma_{d_0} < 5.0$


\end{itemize}

Figure~\ref{f:meerates} \cite{atlaselcid} show the estimated rates in data and
simulation. The dashed lines indicate the bins in which the rates are
calculated. Total uncertainty, which is computed as the sum in quadrature of
statistical and systematic uncertainties, is also showed. In most bins,
simulation over-estimates the rates as compared to the data by 5-$20\%$
depending on $\eta$ and electron requirements.

\begin{figure}[H]
	\includegraphics[width=12cm]{figures/fig_17}
	\centering

	\caption{Charge mis-identification probabilities in $\eta$ bins, $E_T$ between
		$25$ GeV and $50$ GeV \cite{atlaselcid}. Three different sets of selection
		requirements (Medium, Tight + Isolation, and Tight + Isolation + impact
		parameter) are shown, along with simulation expectations. Displayed in the
		lower panel is the data-to-simulation ratios. The uncertainties are the total
		uncertainties from the sum in quadrature of statistical and systematic
		uncertainties. The dashed lines indicate the bins in which the rates are
		calculated.}

	\label{f:meerates}
\end{figure}


\subsection{Estimating Charge Mis-identification Background from the Charge
	Mis-identification Rates}

In this section we give an example of how the charge mis-identification rates
may be used to estimate the charge mis-identificaion background in analysis
with a same-sign lepton pair signature. Suppose a sample of same-sign electron
pairs has been selected in the bin pairs $(i,j)$ (see
Section~\ref{s:cllhmethod}). Let there be $n_{\text{ss,ij}}$ of such pairs, and
we wish to determine the charge misidentification contribution to this number.

To begin, we have to distinguish between the number of same-sign electron pairs
$n_{\text{ss,ij}}$ that has been selected and the number of genuine same-sign
electron pairs. The latter is what would be counted if there were no charge
misidentification. Denote it by $\bar{n}_{\text{ss,ij}}$.

A charge misidentification contribution occurs whenever there is an
opposite-sign pair of electrons in which one of the electron has its charge
mis-identified. The probability for this to happen is, according to
Equation~\ref{eq:cijw},

$$p_{ij} = (1-\epsilon_i)\epsilon_j +  \epsilon_i(1-\epsilon_j,)$$

where $\epsilon_i$ and $\epsilon_j$ are the charge mis-identification rates in
the bins. This probability has to be multiplied by the real number of
opposite-sign pairs, and not the number of opposite-sign pairs counted in the
bin pair, because the latter involves contribution from same-sign pairs
$\bar{n}_{\text{ss,ij}}$ as well.

Denote the number of opposite-sign pairs counted in the bin pair $(i,j)$ by
$n_{\text{os,ij}}$, and the corresponding real quantity by
$\bar{n}_{\text{os,ij}}$. The quantities available are $n_{\text{ss,ij}}$,
$n_{\text{os,ij}}$, and the mis-identification rates $\epsilon_i$ and
$\epsilon_j$. The unknown are $\bar{n}_{\text{ss,ij}}$ and
$\bar{n}_{\text{os,ij}}$, but they are needed to determine charge
mis-identification contribution. Regarding this, the following relation holds

$$n_{\text{os,ij}} = \bar{n}_{\text{os,ij}} - \bar{n}_{\text{os,ij}}\times
	p_{ij} + \bar{n}_{\text{ss,ij}}\times p_{ij},$$

which says that the number of opposite-sign lepton pairs counted in the bin
pair $(i,j)$ is the corresponding real number minus the portion that is
identified as same-sign plus the contribution from real same-sign pairs. This
may be re-written as

$$ n_{\text{os,ij}} = \bar{n}_{\text{os,ij}}\times (1 - p_{ij}) +
	\bar{n}_{\text{ss,ij}} \times p_{ij}. $$

Similarly we have the following relation

$$ n_{\text{ss,ij}} = \bar{n}_{\text{ss,ij}}\times (1 - p_{ij}) +
	\bar{n}_{\text{os,ij}} \times p_{ij} $$

These two relations form a system of equations from which the unknown
$\bar{n}_{\text{os,ij}}$ and $\bar{n}_{\text{ss,ij}}$ may be solved. Then the
charge mis-identification contribution to $n_{\text{ss,ij}}$ is simply
$\bar{n}_{\text{os,ij}}\times p_{ij}$.

The method just discussed is to be contrasted with the scale factor method,
where scale factors that adjust the charge mis-identification rates in
simulations to match the data are provided to different analyses. The former
method excludes the need for the use of all systematic uncertainties that are
associated with the use of simulation samples.


\section{Conclusions}\label{s:chargecon}

This chapter described the electron charge mis-identification problem at ATLAS
and how the charge mis-identification rates are measured by fitting a Poisson
likelihood function using the $Z\to e^+e^-$ data sample ({\color{pink} data
		set}). Three sets of charge mis-identification rates are measured and provided
to ATLAS analyses, corresponding to three different sets of selection
requirements (Medium, Tight + Isolation, and Tight + Isolation + impact
parameter) ({\color{pink} the range of flip rates observed}). In general,
simulation underestimates the charge mis-identificaion rates as compared to
those in the data.

It is to be noted that in addition to measuring the charge mis-identification
rates, a separate effort was started by the physics team at Universit\'{e} de
Montr\'{e}al aiming at reducing charge mis-identification. The technique relies
on the output of a boosted decision tree (BTD) using a simulated sample of
single electrons (see Reference~\cite{atlaselcid}) ({\color{pink} fix
		reference}).
