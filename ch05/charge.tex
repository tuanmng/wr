
\chapter{ESTIMATING THE RATES OF ELECTRON CHARGE
  \\MIS-IDENTIFICATION}\label{c:cid}

Many physics analyses involve charged leptons in their final states, where
leptons typically refer to electrons or muons. Not only are the kinematic
quantities associated with these particles measured, their charges have to be
determined as well, using the curvatures of the tracks which result from the
inner detector magnetic field. As will be discussed below, the measured charges
are not always correct, causing what is called charge mis-identification.

Charge mis-identification is important for analyses that involve same-sign
leptons~\footnote{For all practical purposes, muon charge mis-identification is
negligible at ATLAS~\cite{muonchargemid}. Compared to electrons, muons are much
less likely to undergo bremsstrahlung and pair-production in the detector.
Moreover, muon tracks are measured in the inner detector as well as in the muon
spectrometer, prodiving a larger lever arm for curvature measurements.} in the
final state, such as measurements of the same-sign $WW$
scattering~\cite{sswwscatting}, Higgs production in association with a $t\tbar$
pair ($t\tbar H$), or SUSY search with two same-sign
leptons~\cite{ssssleptons}. In general, electron charge mis-identification
rates occur on the order of $\text{O}(1\%)$, whereas Standard Model processes
that provide opposite-sign dileptons, dominantly $Z\to e^+e^-$, occur
approximately $10^3$ times more commonly than genuine Standard Model sources of
same-sign leptons (dominantly $WZ$ production). Accordingly, opposite-sign
sources of dileptons that suffer from charge mis-identification can constitute
a significant background in these searches, and must therefore be estimated as
precisely as possible.


This chapter describes a method for estimating the rate of charge
mis-identification using a likelihood function. Section~\ref{s:chargereasons}
discusses briefly how electron charge mis-identification might arise at ATLAS.
Section~\ref{s:cllhmethod} discusses the likelihood method, including the
Poisson likelihood used as well as how it is applied to $Z\to e^+e^-$ events to
measure the charge mis-identification rates. Finally, Section~\ref{s:chargecon}
provides some conclusions.

The data used was collected with the ATLAS detector in $2012$, at $8$ TeV
center-of-mass energy and corresponds to an integrated luminosity of $20.3$
$\text{fb}^{-1}$.

\section{Electron Charge Mis-identification}\label{s:chargereasons}

At ATLAS, the sign of the charge of an electron is determined from the
curvature of its track in the inner detector~(Section~\ref{s:decinner}). Charge
mis-identification occurs mainly because of two reasons:

\begin{itemize}

	\item The electron may radiate photons as it passes through the detector and
	      interacts with the detector materials. These radiated photons may in turn
	      convert to electron-positron pairs. A charge mis-identification occurs when the
	      electron candidate is matched to the wrong track. This is
	      the dominant source of charge mis-identification.

	\item The reconstructed track associated with the electron has a small
	      curvature, which may happen at very high momentum or at large pseudorapidity,
	      the latter case because of the limit of the lever arm of the tracker. Indeed,
	      for $\abs{\eta}\geq 2.0$, the track is oriented in the endcap region of the
	      ATLAS detector and will not reach the full available lever arm of $\sim 1.2$
	      m transverse to the beam of the inner detector.

\end{itemize}

%%%%%%%%%%%%%%%%%%%%%%%%%%%%%%%%%%%%%%%%%%%%%%%%%%%%%%%%
%%%%%%%%%%%%%%%%%%%%%%%%%%%%%%%%%%%%%%%%%%%%%%%%%%%%%%%%
\section{The Likelihood Method}\label{s:cllhmethod}

We assume that there is a probability associated to charge mis-identification
and seek to determine this rate in a sample of electrons. At ATLAS, $Z\to
e^+e^-$ events are used for this purpose because they are a dominant source of
opposite-sign electrons as compared to other Standard Model sources. A very
clean and high-statistics sample of electrons may be obtained by selecting two
isolated electrons around the invariant $Z$ mass peak. Due to charge
mis-identification, not only are opposite-sign pair of electrons observed,
same-sign pairs will be encountered as well, from which the charge
mis-identification rates could be determined. More specifically, the
mis-identification rates to be extracted are parameters of a Poisson likelihood
function that will be discussed below.

The rates obtained will be applied to an opposite-sign control sample in data,
or to correct the MC simulation, to estimate the electron charge
mis-identification background in a same-sign lepton analysis.

\subsection{The $Z\to e^+e^-$ Sample}\label{s:zeerates}

At ATLAS, electron charge mis-identification rates are extracted from $Z\to
e^+e^-$ events using a likelihood function (Section~\ref{s:cpllh}). These
events are required to undergo the following preliminary selections. Further
selections that will be applied before the rates are extracted are discussed 
in Section~\ref{s:crates}.

\paragraph{Preliminary event selections}

\begin{itemize}

	\item A logical OR between two single-electron triggers, one with $E_T > 24$
	      GeV plus Medium identification, and one with $E_T > 60$ GeV plus Loose
	      identification.

	\item At least two reconstructed electron candidates with $\abs{\eta} < 2.47$.
	      The invariant mass of the electron pair must be within $\pm 15$ GeV of the $Z$ mass.

\end{itemize}

Figure~\ref{f:meea}~\cite{atlaselcid} shows the invariant mass distribution
$m_{ee}$ in data and simulation for $E_T$ between $25$ GeV and $50$ GeV and
$0.0 < \eta < 0.8$ (left) or $2.0 < \eta < 2.47$ (right). Due to charge
mis-identification same-sign electron pairs also exist in addition to
opposite-sign pairs, indicating a charge mis-identification rate of $\sim
10^{-3}$ in the central region and $\sim 10\%$ in the high $\eta$ region. The
higher rates in the latter is expected because of the larger amount of material
and the limited lever arm in the forward region. In both cases, same-sign pairs
show a broader peak that is also slightly shifted towards lower values,
consistent with the fact that the radiation that causes charge
mis-identification also results in energy loss.


\begin{figure}[H]
	\includegraphics[width=7cm]{figures/fig_16a}
	\includegraphics[width=7cm]{figures/fig_16b}
	\centering

	\caption{Distribution of the invariant mass $m_{ee}$ for $E_T$ between $25$
		and $50$ GeV and $\abs{\eta} $ between $0.0$ and $0.8$ \cite{atlaselcid}. Due
		to charge mis-identification same-sign pairs as well as opposite-sign pairs
		are observed.}

	\label{f:meea}

\end{figure}

The next section discusses the Poisson likelihood function that is used to
fit the data.


\subsection{The Poisson Likelihood}\label{s:cpllh}

In a truth-level $e^+e^-$ pair, which will also be called a truth-level
opposite-sign pair, if the charge of any one of the electrons is
mis-identified, then a same-sign pair will be observed instead\footnote{ In
	order to distinguish between truth-level electron pairs and identified ones, we
	will always write truth-level to indicate the former. Thus if a pair is not
	preceded by ``truth-level'', it is tacitly understood to be an identified pair. }.
Assuming a probability $p$ that a truth-level opposite-pair will be identified
as a same-sign pair, then in considering $n$ truth-level pairs $e^+e^-$, the
probability that exactly $n_{ss}$ same-sign pairs will be counted follows the
binomial distribution

$$
	P(n_{ss}) = \binom{n}{n_{ss}}p^{n_{ss}}(1-p)^{n-n_{ss}}.
$$

The charge mis-identification probability $p$ is typically small while the
sample of $n$ pairs of electrons considered is typically very large, and
therefore the Poisson distribution may be used instead. Thus, let
%%
\begin{equation}\label{eq:cmss}
	m_{ss} = np
\end{equation}
%%
denote the expected number of same-sign pairs, then the Poisson distribution
%%
\begin{equation}\label{eq:cpoissonn}
	P(n_{ss}) = \frac{m_{ss}^{n_{ss}} e^{-m_{ss}}}{n_{ss}!}
\end{equation}
%%
gives the probability of counting $n_{\text{ss}}$ same-sign pairs, given the
expected number of same-sign pairs $m_{\text{ss}}$. This will be used as
a likelihood function, to be maximized to extract the charge mis-identification
rates, as will be explained further below.

The probability $p$ that a truth-level opposite-pair will be identified as a
same-sign pair may be written directly in term of the probability of charge
mis-identification associated to an individual electron. If $\epsilon$ denotes
the latter probability, then because a same-sign pair will be observed
precisely when only one of the electrons has its charge mis-identified, we may
write

\begin{equation}\label{eq:cprobpair}
	p = (1-\epsilon)\epsilon +  \epsilon(1-\epsilon).
\end{equation}

The Poisson likelihood of Equation~\ref{eq:cpoissonn} may now be written to
depend explicitly on $\epsilon$:

\begin{equation}\label{eq:cpoisson}
	P(n_{ss}|\epsilon) = \frac{m_{ss}^{n_{ss}} e^{-m_{ss}}}{n_{ss}!}, \quad
	\text{ where } \quad
	m_{ss} = np = n(1-\epsilon)\epsilon +  \epsilon(1-\epsilon).
\end{equation}

The maximization of this function gives the mis-identification rates
$\epsilon$'s.

On the other hand, because charge mis-identification rates are expected to show
strong dependencies on $p_T$ and $\eta$ of the electrons
(Section~\ref{s:chargereasons}), they are often measured in bins of these two
quantities. In such a situation the electrons in a pair generally belong to
different bins and that needs to be taken into account in the likelihood
function. Thus, the electrons are assigned charge mis-identification
probabilities $\epsilon_i$ and $\epsilon_j$, where the indices $i$ and $j$
indicate the bins, and we write


\begin{itemize}
	\item The probability

	      \begin{equation}\label{eq:cijw}
		      p_{ij} = (1-\epsilon_i)\epsilon_j +  \epsilon_i(1-\epsilon_j)
	      \end{equation}

	      in place of the probability $p$ in Equation~\ref{eq:cprobpair}. This is the
	      probability an opposite-sign pair
	      may be seen as a same-sign pair in the bin pair $(i,j)$

	\item The number of electron pairs considered, $n_{ij}$,  in the bin pair $(i,j)$

	\item The expected number of same-sign pairs

	      \begin{equation}\label{eq:cmssij}
		      m_{ss,ij} = n_{ij} p_{ij}
	      \end{equation}

	      in place of the expected number of same-sign pairs in Equation~\ref{eq:cmss}


	\item The Poisson likelihood

	      \begin{equation}\label{eq:cpoissonij}
		      P(n_{ss,ij} | \epsilon_i, \epsilon_j) = \frac{m_{ss,ij}^{n_{ss,ij}} e^{-m_{ss,ij}}}{n_{ss,ij}!}
		      , \quad
		      \text{ where } \quad
		      m_{ss,ij} = n_{ij} p_{ij} = (1-\epsilon_i)\epsilon_j +  \epsilon_i(1-\epsilon_j),
	      \end{equation}

	      in place of the Poisson likelihood in Equation~\ref{eq:cpoisson}. This will also
	      be denoted simply as $L_{ij}$

\end{itemize}

These equations are valid whether the rates are extracted in only $p_T$ bins,
only $\eta$ bins, or both, because in the latter case the grid of
two-dimensional bins may be treated as a long one-dimensional sequence of bins.
On the other hand, all the possible bin pairs $(i,j)$ need to be used and
therefore, assuming statistically-independent rates, the rates $\epsilon_i$ to
be extracted come from the maximization of the likelihood function
%
$$
	L = \prod_{i,j}L_{ij},
$$
%
the data being $n_{ij}$, the numbers of electrons observed in the bin pair
$(i,j)$, and $n_{ss,ij}$, the number of same-sign electron pairs observed in
the bin pair $(i,j)$.

\paragraph{Background subtractions} Backgrounds to $Z\to e^+e^-$ events
consistly mostly of events involving top quarks, diboson events, and $W$+jets
events. They are assumed to be flat in the invariant $Z$ mass peak selection
and are subtracted by a method called the sideband method. To this end, we will
denote the invariant mass interval around the $Z$ mass peak by $(m_l,m_h)$,
where $m_l=15$ GeV is the low mass point and $m_h=15$ GeV the high mass point.
Then an interval of $15$ GeV is selected to the left of $m_l$ and to the right
of $m_h$, i.e. $m_l = m_h = 15$ GeV and there are now two side intervals
$(m_l-w_l, m_l)$ and $(m_h, m_h+w_h)$ in addition to the original interval
$(m_l,m_h)$. The side intervals are assumed to be dominated by background
events and are used to compute the backgrounds in the $(m_l,m_h)$ interval,
i.e. to subtract background contamination in $n_{ij}$ and $n_{ss,ij}$,
quantities that need to be counted in the $(m_l,m_h)$ interval. We will write
$b(n_{ij})$ for the background contamination in $n_{ij}$, and $b(n_{ss,ij})$
for the background contamination in $n_{ss,ij}$; they will be computed as
weighted quantities:

$$
	b(n_{ij}) = \frac{w_l\times n_{ij}^l + w_h\times n_{ij}^h}{w_l + w_h}, \qquad
	b(n_{ss,ij})= \frac{w_l\times n_{ss,ij}^l + w_h \times n_{ss,ij}^h }{w_l + w_h}.
$$

The terms $n_{ij}$ and $n_{ss,ij}$ and the background terms $b(n_{ij})$ and
$b(n_{ss,ij})$ are put into the Poisson likelihood (Equation~\ref{eq:cpoissonij}):
%
$$
	P(n_{ss,ij} | \epsilon_i, \epsilon_j) = \frac{m_{ss,ij}^{n_{ss,ij}} e^{-m_{ss,ij}}}{n_{ss,ij}!}
$$
%
in which the background terms make a contribution to the expected number of
same-sign $m_{ss,ij}$ in the likelihood, modifying it from $m_{ss,ij} = n_{ij}
	p_{ij}$ (Equation~\ref{eq:cmssij}) to
%
$$
	m_{ss,ij} = (n_{ij} - b(n_{ij})) \times p_{ij} + b(n_{ss,ij}).
$$
%
The first quantity on the right in the equation above is the same-sign
contribution from signal events where the background events have to be
subtracted, and the second quantity is the contribution from background events.

%%%%%%%%%%%%%%%%%%%%%%%%%%%%%%%%%%%%%%%%%%%%%%%%%%%%%%%%
\subsection{Charge Mis-identification Rates and Uncertainties}\label{s:crates}

The rates are obtained upon the maximization of the likelihood function
discussed in the previous section. The statistical uncertainties associated
with the estimated rates depend on the statistics of the data, and are given by
the statistical tool that maximizes the Poisson likelihood.

\vspace{3mm}

The following sources of systematic uncertainties are evaluated:

\begin{itemize}


	\item Systematic uncertainty that comes from background subtraction, which is
	      evaluated by determining the rates with and without background subtraction. The
	      inclusion of this uncertainty ensures a conservative figure of systematic
	      uncertainty in the charge mis-identification rates; it has a small impact
	      because the background is small.


\item The invariant mass interval $(m_l,m_h)$ may be varied, from $15$ GeV
around the $Z$ mass to $10$ and $20$ GeV additionally. This provides an
estimation of the impact of the choice of mass window on the measure rates.


	\item The invariant mass widths $w_l$ and $w_h$ may be varied, taking values
	      $20$, $25$, or $30$ GeV. Thus, the uncertainty on the rates
	      due to the choice of a mass width is taken into account. 

\end{itemize}

The actual rates are estimated for the following three sets of requirements:

\begin{itemize}
	\item Medium: Medium identification requirements

	\item Tight + isolation: Tight identification requirements plus track isolation
	      cut $p_T^{\text{cone 0.2}} / E_T < 0.14$.

	\item Tight + isolation + impact parameter: Tight identification plus
	      $E_T^{\text{cone 0.3}} / E_T < 0.14$ and $p_T^{\text{cone 0.2}} / E_T < 0.07$
	      and additionally $\abs{z_0}\times \sin\theta < 0.5$ mm and
	      $\abs{d_0}/\sigma_{d_0} < 5.0$


\end{itemize}

Figure~\ref{f:meerates} \cite{atlaselcid} shows the estimated rates in data and
simulation, for electron $E_T$ between 25 and 50 GeV as a function of $\eta$, the
variable upon which they depend the most. The dashed lines indicate the bins
in which the rates are calculated. The total uncertainty, which is computed as
the sum in quadrature of statistical and systematic uncertainties, is also
showed. Charge mis-identification rates vary from below $1\%$ in the central
region to $\sim 10\%$ in high $\eta$ region, reflecting the correlation of the
rates with bremsstrahlung, and thus a dependency on the amount of the material
traversed. On the other hand, tighter selection criteria, in particular
requirements on the isolation or track parameters, may decrease the charge
misidentification probability by a factor of up to four, depending on the
additional selection requirements\footnote{The energy in the cone around an
electron could indicate the amount of energy deposited by bremsstrahlung, and
large values of the track impact parameters could mean that the track matched
to the electron is not a prompt track from the primary vertex but from a
secondary interaction or bremsstrahlung and a subsequent
conversion~\cite{atlaselcid}.}. Moreover, as is seen, simulation over-estimates
the rates as compared to the data by 5-$20\%$ depending on $\eta$ and electron
requirements.

Charge mis-identification rates are known to show a positive correlation with 
$p_T$ as well (Figure~\ref{f:meeratesbdt}).

\begin{figure}[H]
	\includegraphics[width=10cm]{figures/fig_17}
	\centering

	\caption{Charge mis-identification probabilities in $\eta$ bins, $E_T$ between
		$25$ GeV and $50$ GeV \cite{atlaselcid}. Three different sets of selection
		requirements (Medium, Tight + Isolation, and Tight + Isolation + impact
		parameter) are shown, along with simulation expectations. Displayed in the
		lower panel is the data-to-simulation ratios. The uncertainties are the total
		uncertainties from the sum in quadrature of statistical and systematic
		uncertainties. The dashed lines indicate the bins in which the rates are
		calculated.}

	\label{f:meerates}
\end{figure}



\subsection{Estimating Charge Mis-identification Background from the Charge
	Mis-identification Rates}

In this section we give an example of how the charge mis-identification rates
may be used to estimate the charge mis-identification background in analysis
with a same-sign lepton pair signature. This technique is used in the SUSY
same-sign leptons search~\cite{ssssleptons}. Thus, given $n_{\text{ss,ij}}$ of
same-sign electron pairs that has been selected in the bin pairs $(i,j)$
(Section~\ref{s:cllhmethod}), we want to determine the charge
mis-identification contribution to it.


To begin, the number of same-sign electron pairs $n_{\text{ss,ij}}$ that has
been selected under a set of selection requirements is to be distinguished from
the number of truth-level same-sign electron pairs. The latter is what would be
counted in the bin pairs $(i,j)$ if there were no charge mis-identification. In
the following we will write it by $\bar{n}_{\text{ss,ij}}$.

A charge mis-identification contribution occurs whenever there is a truth-level
opposite-sign pair of electrons in which one of the electron has its charge
mis-identified. The probability for this to happen is, according to
Equation~\ref{eq:cijw},
%
$$p_{ij} = (1-\epsilon_i)\epsilon_j +  \epsilon_i(1-\epsilon_j,),$$
%
where $\epsilon_i$ and $\epsilon_j$ are the charge mis-identification rates in
the bins. Now, in the same bin pair $(i,j)$ the number of opposite-sign pairs
obtained from the same selection requirements may be counted as well, we will write
it as $n_{\text{os,ij}}$. Moreover, as for the same-sign case, this has to be
distinguished from the number of truth-level opposite-sign pairs, which will be
denoted $\bar{n}_{\text{os,ij}}$. The number of interest is
$\bar{n}_{\text{os,ij}}$, because given the mis-identification rate $p_{ij}$,
the charge mis-identification contribution to $n_{\text{ss,ij}}$ is simply
$\bar{n}_{\text{os,ij}}\times p_{ij}$.


The only quantities known are $n_{\text{ss,ij}}$, $n_{\text{os,ij}}$, and the
mis-identification rates $\epsilon_i$ and $\epsilon_j$, while
$\bar{n}_{\text{ss,ij}}$ and $\bar{n}_{\text{os,ij}}$ are unknown. However, the
following relation holds
%
$$n_{\text{os,ij}} = \bar{n}_{\text{os,ij}} - \bar{n}_{\text{os,ij}}\times
	p_{ij} + \bar{n}_{\text{ss,ij}}\times p_{ij},$$
%
which reflects the fact that the number of opposite-sign lepton pairs counted
in the bin pair $(i,j)$ is the corresponding truth-level number minus the portion
that is identified as same-sign plus the contribution from truth-level same-sign
pairs. This may be re-written as
%%
$$ n_{\text{os,ij}} = \bar{n}_{\text{os,ij}}\times (1 - p_{ij}) +
	\bar{n}_{\text{ss,ij}} \times p_{ij}. $$
%%
Similarly we have the following relation
%%
$$ n_{\text{ss,ij}} = \bar{n}_{\text{ss,ij}}\times (1 - p_{ij}) +
	\bar{n}_{\text{os,ij}} \times p_{ij}.$$
%%
Thus there are two equations in two unknowns and as a result
$\bar{n}_{\text{os,ij}}$ and $\bar{n}_{\text{ss,ij}}$ may be solved.

At ATLAS, charge mis-identification rates are also provided to different
analyses as scale factors (the ratios of charge mis-identification rates in
data over those in simulation), to be applied to charge mis-identification
rates in simulations to match the data. If charge mis-identification rates on
data are provided directly instead of the scale factors we can avoid the need
for the use of all systematic uncertainties that are associated with the use of
simulation samples.


\section{Conclusions}\label{s:chargecon}

This chapter describes the electron charge mis-identification problem at ATLAS
and how the charge mis-identification rates are extracted by fitting a Poisson
likelihood function using the $Z\to e^+e^-$ data sample, collected at $8$ TeV
LHC center-of-mass energy in $2012$ with the ATLAS detector and corresponding
to an integrated luminosity of $20.3$ $\text{fb}^{-1}$. Three sets of charge
mis-identification rates are measured and provided to ATLAS analyses,
corresponding to three different sets of selection requirements (Medium, Tight
+ Isolation, and Tight + Isolation + impact parameter). The rates show a
variation from less than $1\%$ to nearly $10\%$ depending on $\eta$ and $p_T$.
It is also observed from the measurements that, in general, simulation
underestimates the charge mis-identification rates as compared to those in the
data.

In Run 2, in addition to measuring the charge mis-identification rates, a
separate effort was started by the physics team at Universit\'{e} de
Montr\'{e}al, aiming at reducing charge mis-identification. The technique
relies on the output of a boosted decision tree using a simulated sample of
single electrons. Figure~\ref{f:meeratesbdt} shows the impact of applying the
BDT requirement on charge mis-identification rates; it has been demonstrated to
reduce charge mis-identification rates by about a factor of 10 while
maintaining a $97\%$ efficiency on signal electrons. More details may be found
in Ref.~\cite{atlaselcid}.

\begin{figure}[H]
	\includegraphics[width=10cm]{figures/ch5fig_15a}
	\includegraphics[width=10cm]{figures/ch5fig_15b}
	\centering

	\caption{Charge mis-identification probabilities in 2016 data and $Z\to e^+e^-$
		events as a function of $E_T$ (top) and $\abs{\eta}$ (bottom) that shows also
		the impact of applying the BDT requirement (in blue) to suppress charge mis-identification.
	}

	\label{f:meeratesbdt}
\end{figure}
