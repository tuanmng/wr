
\chapter{ESTIMATING THE RATES OF ELECTRON CHARGE
  \\MIS-IDENTIFICATION}\label{c:cid}

Many physics analyses involve charged leptons in their final states, where
leptons typically refer to electrons or muons. The curvatures of the tracks of
these particles, which exist because of the detector magnetic fields, are used
to determine the particles' charges. As will be discussed below, the measured
charges are not always correct, causing what is called charge
mis-identification.

Charge mis-identification is important for analyses that involve same-sign
electrons~\footnote{Muon charge mis-identification is negligible~\cite{muonchargemid}.
	Compared to electrons, muons are much less likely to undergo bremsstrahlung and
	pair-production in the detector. Moreover, muon tracks are measured in the
	inner detector as well as in the muon spectrometer, prodiving a larger lever
	arm for curvature measurements.} in the final state, such as measurements of
the same-sign $WW$ scattering~\cite{sswwscatting}, Higgs production in
association with a $t\tbar$ pair ($t\tbar H$), or SUSY search with two
same-sign leptons~\cite{ssssleptons}. In general, charge mis-identification
rates occur on the order of $\text{O}(1\%)$, while Standard Model processes
that provide opposite-sign dileptons, dominantly $Z\to e^+e^-$, occur
approximately $10^3$ times more commonly than genuine Standard Model sources of
same-sign leptons (dominantly $WZ$ production). Thus, opposite-sign sources of
dileptons that suffer from charge mis-identification can constitute a large
background in these searches.


This chapter describes a method for estimating the rate of charge
mis-identification using a likelihood function. Section~\ref{s:chargereasons}
discusses briefly how electron charge mis-identification might arise at ATLAS.
Section~\ref{s:cllhmethod} discusses the likelihood method, including the
Poisson likelihood used as well as how it is applied to $Z\to e^+e^-$ events to
measure the charge mis-identification rates. Finally, Section~\ref{s:chargecon}
provides some conclusions.

The data used was collected with the ATLAS detector in $2012$, at $8$ TeV
center-of-mass energy and corresponds to an integrated luminosity of $20.3$
$\text{fb}^-1$.

\section{Electron Charge Mis-identification}\label{s:chargereasons}

At ATLAS, the sign of the charge of an electron is determined from the
curvature of its track in the inner detector~(Section~\ref{s:decinner}). Charge
mis-identification, where the charge of the electron is identified incorrectly,
occurs mainly because of two reasons:

\begin{itemize}

	\item The electron may radiate photons as it passes through the detector and
	      interacts with the detector materials. These radiated photons may in turn
	      convert to electron-positron pairs. A charge mis-identification occurs when the
				electron candidate is matched to the wrong track. This is the dominant source 
				of charge mis-identification. 

	\item The reconstructed track of the electron appears rather straight, i.e. the
	      curvature of the track is small, which may happen at very high momentum or at
	      large pseudorapidity, the latter case because the lever arm of the tracker is
	      limited.

\end{itemize}

%%%%%%%%%%%%%%%%%%%%%%%%%%%%%%%%%%%%%%%%%%%%%%%%%%%%%%%%
%%%%%%%%%%%%%%%%%%%%%%%%%%%%%%%%%%%%%%%%%%%%%%%%%%%%%%%%
\section{The Likelihood Method}\label{s:cllhmethod}

We assume there is a probability associated to charge mis-identification and
seek to determine this rate in a sample of electrons. At ATLAS, $Z\to e^+e^-$
events are used for this purpose because they are a dominant source of
opposite-sign electrons as compared to other Standard Model sources. A very
clean and high-statistics sample of electrons may be obtained by selecting two
isolated electrons around the invariant $Z$ mass peak. The mis-identification
rates to be extracted are parameters of a Poisson likelihood function discussed
below.

\subsection{The Poisson Likelihood}\label{s:cpllh}

In a truth-level pair $e^+e^-$ whose charges are to be measured, which we will
call an opposite-sign pair, if the charge of any one of the electrons is
mis-identified, then a same-sign pair will be seen instead. Assuming a
probability $p$ that an opposite-pair will be identified as a same-sign pair,
then in considering $n$ pairs $e^+e^-$, the probability that exactly $n_{ss}$
same-sign pairs will be counted follows the binomial distribution

$$
	P(n_{ss}) = \binom{n}{n_{ss}}p^{n_{ss}}(1-p)^{n-n_{ss}}.
$$

The charge mis-identification probability $p$ is typically small while the
sample of $n$ pairs of electrons considered is typically very large, and
therefore the Poisson distribution may be used instead. Thus, let

\begin{equation}\label{eq:cmss}
	m_{ss} = np
\end{equation}

denote the expected number of same-sign pairs, then the Poisson distribution


\begin{equation}\label{eq:cpoissonn}
	P(n_{ss}) = \frac{m_{ss}^{n_{ss}} e^{-m_{ss}}}{n_{ss}!}
\end{equation}

gives the probability of counting $n_{\text{ss}}$ same-sign pairs, given the
average number of same-sign pairs $m_{\text{ss}}$. This will also be called a
likelihood function.


The probability $p$ that a truth-level opposite-pair will be identified as a
same-sign pair may be written directly in term of the probability of charge
mis-identification associated to an individual electron. If $\epsilon$ denotes
the latter probability, then because a same-sign pair will be seen when exactly
one of the electrons has its charge mis-identified, we may write

\begin{equation}\label{eq:cprobpair}
	p = (1-\epsilon)\epsilon +  \epsilon(1-\epsilon).
\end{equation}

The Poisson likelihood of Equation~\ref{eq:cpoissonn} may now be written to
depend explicitly on $\epsilon$:

\begin{equation}\label{eq:cpoisson}
	P(n_{ss}|\epsilon) = \frac{m_{ss}^{n_{ss}} e^{-m_{ss}}}{n_{ss}!}, \quad
	\text{ where } \quad
	m_{ss} = np = n(1-\epsilon)\epsilon +  \epsilon(1-\epsilon).
\end{equation}

\subsection{Estimating Charge Mis-Identification Rates on $Z\to e^+e^-$
	events}\label{s:zeerates}

Electron charge mis-identification rates are extracted from $Z\to e^+e^-$
events using the likelihood function~\ref{eq:cpoisson}. These events, which are
also called tag-and-probe $Z\to e^+e^-$ events, are selected by applying the
following selections.

\paragraph{Event selections}

\begin{itemize}

	\item A logical OR between two single-electron triggers, one with $E_T > 24$
	      GeV plus Medium identification, and one with $E_T > 60$ GeV plus Loose
	      identification.

	\item At least two reconstructed electron candidates with $\abs{\eta} < 2.47$.

	\item One electron, called the tag candidate, is required to:
	      \begin{itemize}
		      \item Pass the Tight identification requirement.
		      \item Have $E_T > 25$ GeV.
		      \item Be matched to a trigger electron within $\Delta < 0.15$.
		      \item Have $1.37 < \abs{\eta} < 1.52$.
	      \end{itemize}

	      The other electron, called the probe candidate, must
	      \begin{itemize}
		      \item Have $E_T > 10$ GeV.
		      \item Satisfy the track quality criteria (the tracks associated with the
		            electron must have at least one hit in the pixel detector and at least seven
		            hits in the pixel and SCT detectors).
	      \end{itemize}

	\item Finally, the invariant mass of the tag-probe pair must be within $\pm 15$
	      GeV of the $Z$ mass.

\end{itemize}

Figure~\ref{f:meea}~\cite{atlaselcid} shows the invariant mass distribution
$m_{ee}$ in data and simulation for $E_T$ between $25$ GeV and $50$ GeV and
$0.0 < \eta < 0.8$ (left) or $2.0 < \eta < 2.47$ (right). Due to charge
mis-identification same-sign electron pairs also exist in addition to
opposite-sign pairs. It is seen that same-sign pairs have a broader peak which
is also slightly shifted to lower values, consistent with the fact that
radiation which causes charge mis-identification also causes energy loss. It is
also seen that the number of same-sign pairs is larger at the high $\eta$
range, which is expected because charge mis-identification rates are expected
to be higher here because of the difficulty of curvature measurements.

\begin{figure}[H]
	\includegraphics[width=7cm]{figures/fig_16a}
	\includegraphics[width=7cm]{figures/fig_16b}
	\centering

	\caption{Distribution of the invariant mass $m_{ee}$ for $E_T$ between $25$
		and $50$ GeV and $\abs{\eta} $ between $0.0$ and $0.8$ \cite{atlaselcid}. Due
		to charge mis-identification same-sign pairs as well as opposite-sign pairs
		are seen.}

	\label{f:meea}

\end{figure}


Since charge mis-identification rates are expected to show strong dependencies
on $p_T$ and $\eta$ of the electrons (Section~\ref{s:chargereasons}), they are
often measured in bins of these two quantities. In such a situation the
electrons in a pair generally belong to different bins and that needs to be
taken into account in the likelihood function. Thus, the electrons are assigned
charge mis-identification probabilities $\epsilon_i$ and $\epsilon_j$, where
the indices $i$ and $j$ indicate the bins, and we write


\begin{itemize}
	\item The probability

	      \begin{equation}\label{eq:cijw}
		      p_{ij} = (1-\epsilon_i)\epsilon_j +  \epsilon_i(1-\epsilon_j)
	      \end{equation}

	      in place of the probability $p$ in Equation~\ref{eq:cprobpair}. This is the
	      probability an opposite-sign pair
	      may be seen as a same-sign pair in the bin pair $(i,j)$

	\item The number of electron pairs considered, $n_{ij}$,  in the bin pair $(i,j)$

	\item The expected number of same-sign pairs

	      \begin{equation}\label{eq:cmssij}
		      m_{ss,ij} = n_{ij} p_{ij}
	      \end{equation}

	      in place of the expected number of same-sign pairs in Equation~\ref{eq:cmss}


	\item The Poisson likelihood

	      \begin{equation}\label{eq:cpoissonij}
		      P(n_{ss,ij} | \epsilon_i, \epsilon_j) = \frac{m_{ss,ij}^{n_{ss,ij}} e^{-m_{ss,ij}}}{n_{ss,ij}!}
		      , \quad
		      \text{ where } \quad
		      m_{ss,ij} = n_{ij} p_{ij} = (1-\epsilon_i)\epsilon_j +  \epsilon_i(1-\epsilon_j).
	      \end{equation}

	      in place of the Poisson likelihood in Equation~\ref{eq:cpoisson}. This will also
	      be denoted simply as $L_{ij}$

\end{itemize}

These equations are valid whether the rates are extracted in only $p_T$ bins,
only $\eta$ bins, or both, because in the latter case the grid of
two-dimensional bins may be treated as a long one-dimensional sequence of bins.
On the other hand, all the possible bin pairs $(i,j)$ need to be used and
therefore, assuming statistically-independent rates, the rates $\epsilon_i$ to
be extracted come from the maximization of the likelihood function


$$
	L = \prod_{i,j}L_{ij},
$$

the data being $n_{ij}$, the numbers of electrons observed in the bin pair
$(i,j)$, and $n_{ss,ij}$, the number of same-sign electron pairs observed in
the bin pair $(i,j)$.

\paragraph{Background subtractions} Backgrounds to $Z\to e^+e^-$ events
consistly mostly of events involving top quarks, diboson events, and $W$+jets
events. They are assumed to be flat in the invariant $Z$ mass peak selection
and are subtracted by a method called the sideband method. To this end, we will
denote the invariant mass interval around the $Z$ mass peak by $(m_l,m_h)$,
where $m_l=15$ GeV is the low mass point and $m_h=15$ GeV the high mass point.
Then an interval of $15$ GeV is selected to the left of $m_l$ and to the right
of $m_h$, i.e. $m_l = m_h = 15$ GeV and there are now two side intervals
$(m_l-w_l, m_l)$ and $(m_h, m_h+w_h)$ in addition to the original interval
$(m_l,m_h)$. The side intervals are assumed to be dominated by background
events and are used to compute the backgrounds in the $(m_l,m_h)$ interval,
i.e. to subtract background contamination in $n_{ij}$ and $n_{ss,ij}$,
quantities that need to be counted in the $(m_l,m_h)$ interval. We will write
$b(n_{ij})$ for the the background contamination in $n_{ij}$, and
$b(n_{ss,ij})$ for the background contamination in $n_{ss,ij}$; they will be
computed as weighted quantities:

$$
	b(n_{ij}) = \frac{w_l\times n_{ij}^l + w_h\times n_{ij}^h}{w_l + w_h}, \qquad
	b(n_{ss,ij})= \frac{w_l\times n_{ss,ij}^l + w_h \times n_{ss,ij}^h }{w_l + w_h}.
$$

The terms $n_{ij}$ and $n_{ss,ij}$ and the background terms $b(n_{ij})$ and
$b(n_{ss,ij})$ are put into the Poisson likelihood (Equation~\ref{eq:cpoissonij}):

$$
	P(n_{ss,ij} | \epsilon_i, \epsilon_j) = \frac{m_{ss,ij}^{n_{ss,ij}} e^{-m_{ss,ij}}}{n_{ss,ij}!}
$$

in which the background terms make a contribution to the expected number of
same-sign $m_{ss,ij}$ in the likelihood, modifying it from $m_{ss,ij} = n_{ij}
	p_{ij}$ (Equation~\ref{eq:cmssij}) to

$$
	m_{ss,ij} = (n_{ij} - b(n_{ij})) \times p_{ij} + b(n_{ss,ij})
$$

The first quantity on the right in the equation above is the same-sign
contribution from signal events where the background events have to be
subtracted, and the second quantity is the contribution from background events.

%%%%%%%%%%%%%%%%%%%%%%%%%%%%%%%%%%%%%%%%%%%%%%%%%%%%%%%%
\subsection{Charge Mis-identification Rates and Uncertainties}

The rates are obtained upon the maximization of the likelihood function
discussed in the previous section. The statistical uncertainties associated
with the estimated rates depend on the statistics of the data, and are given by
the statistical tool that maximizes the Poisson likelihood.

\vspace{3mm}

The following sources of systematic uncertainties are evaluated:

\begin{itemize}[label=\ding{111}]


	\item Systematic uncertainty that comes from background subtraction, which is
	      evaluated by determining the rates with and without background subtraction. The
	      inclusion of this uncertainty ensures a conservative figure of systematic
	      uncertainty in the charge mis-identification rates; it has a small impact
	      because the background is small.


	\item The invariant mass interval $(m_l,m_h)$ may be varied, from $15$ GeV
	      around the $Z$ mass to $10$ and $20$ GeV additionally. In this way an idea of
	      how the selection of an interval may affect the rates may be obtained.


	\item The invariant mass widths $w_l$ and $w_h$ may be varied, taking values
	      $20$, $25$, or $30$ GeV. This takes into account the uncertainty on the rates
	      due to the choice of a mass width.

\end{itemize}

The actual rates are estimated for the following three sets of requirements:

\begin{itemize}[label=\ding{109}]
	\item Medium: Medium identification requirements

	\item Tight + isolation: Tight identification requirements plus track isolation
	      cut $p_T^{\text{cone 0.2}} / E_T < 0.14$.

	\item Tight + isolation + impact parameter: Tight identification plus
	      $E_T^{\text{cone 0.3}} / E_T < 0.14$ and $p_T^{\text{cone 0.2}} / E_T < 0.07$
	      and additionally $\abs{z_0}\times \sin\theta < 0.5$ mm and
	      $\abs{d_0}/\sigma_{d_0} < 5.0$


\end{itemize}

Figure~\ref{f:meerates} \cite{atlaselcid} show the estimated rates in data and
simulation. The dashed lines indicate the bins in which the rates are
calculated. Total uncertainty, which is computed as the sum in quadrature of
statistical and systematic uncertainties, is also showed. In most bins,
simulation over-estimates the rates as compared to the data by 5-$20\%$
depending on $\eta$ and electron requirements.

\begin{figure}[H]
	\includegraphics[width=10cm]{figures/fig_17}
	\centering

	\caption{Charge mis-identification probabilities in $\eta$ bins, $E_T$ between
		$25$ GeV and $50$ GeV \cite{atlaselcid}. Three different sets of selection
		requirements (Medium, Tight + Isolation, and Tight + Isolation + impact
		parameter) are shown, along with simulation expectations. Displayed in the
		lower panel is the data-to-simulation ratios. The uncertainties are the total
		uncertainties from the sum in quadrature of statistical and systematic
		uncertainties. The dashed lines indicate the bins in which the rates are
		calculated.}

	\label{f:meerates}
\end{figure}



\subsection{Estimating Charge Mis-identification Background from the Charge
	Mis-identification Rates}

In this section we give an example of how the charge mis-identification rates
may be used to estimate the charge mis-identification background in analysis
with a same-sign lepton pair signature. Thus, given $n_{\text{ss,ij}}$ of
same-sign electron pairs that has been selected in the bin pairs $(i,j)$
(Section~\ref{s:cllhmethod}), we want to determine the charge
mis-identification contribution to it.


To begin, the number of same-sign electron pairs $n_{\text{ss,ij}}$ that has
been selected is to be distinguished from the number of genuine same-sign
electron pairs. The latter is what would be counted in the bin pairs $(i,j)$ if
there were no charge mis-identification. In the following we will write it by
$\bar{n}_{\text{ss,ij}}$.

A charge mis-identification contribution occurs whenever there is a genuine
opposite-sign pair of electrons in which one of the electron has its charge
mis-identified. The probability for this to happen is, according to
Equation~\ref{eq:cijw},

$$p_{ij} = (1-\epsilon_i)\epsilon_j +  \epsilon_i(1-\epsilon_j,)$$

where $\epsilon_i$ and $\epsilon_j$ are the charge mis-identification rates in
the bins. Now, in the same bin pair $(i,j)$ the number of opposite-sign pairs
may be counted as well, we will write it as $n_{\text{os,ij}}$. Moreover, as
for the same-sign case, this has to be distinguished from the number of genuine
opposite-sign pairs, which will be denoted $\bar{n}_{\text{os,ij}}$. The number
of interest is $\bar{n}_{\text{os,ij}}$, because given the mis-identification
rate $p_{ij}$, the charge mis-identification contribution to $n_{\text{ss,ij}}$
is simply $\bar{n}_{\text{os,ij}}\times p_{ij}$.


The only quantities known are $n_{\text{ss,ij}}$, $n_{\text{os,ij}}$, and the
mis-identification rates $\epsilon_i$ and $\epsilon_j$, while
$\bar{n}_{\text{ss,ij}}$ and $\bar{n}_{\text{os,ij}}$ are unknown. However, the
following relation holds

$$n_{\text{os,ij}} = \bar{n}_{\text{os,ij}} - \bar{n}_{\text{os,ij}}\times
	p_{ij} + \bar{n}_{\text{ss,ij}}\times p_{ij},$$

which reflects the fact that the number of opposite-sign lepton pairs counted
in the bin pair $(i,j)$ is the corresponding genuine number minus the portion
that is identified as same-sign plus the contribution from genuine same-sign
pairs. This may be re-written as

$$ n_{\text{os,ij}} = \bar{n}_{\text{os,ij}}\times (1 - p_{ij}) +
	\bar{n}_{\text{ss,ij}} \times p_{ij}. $$

Similarly we have the following relation

$$ n_{\text{ss,ij}} = \bar{n}_{\text{ss,ij}}\times (1 - p_{ij}) +
	\bar{n}_{\text{os,ij}} \times p_{ij} $$

Thus there are two equations in two unknowns and as a result
$\bar{n}_{\text{os,ij}}$ and $\bar{n}_{\text{ss,ij}}$ may be solved.

At ATLAS, charge mis-identification rates are also provided to different
analyses as scale factors, to be applied to charge mis-identification rates in
simulations to match the data. If charge mis-identification rates on data are
provided directly instead of the scale factors we can avoid the need for the
use of all systematic uncertainties that are associated with the use of
simulation samples.


\section{Conclusions}\label{s:chargecon}

This chapter describes the electron charge mis-identification problem at ATLAS
and how the charge mis-identification rates are extracted by fitting a Poisson
likelihood function using the $Z\to e^+e^-$ data sample, collected at $8$ TeV
LHC center-of-mass energy in $2012$ with the ATLAS detector and corresponds to
an integrated luminosity of $20.3$ $\text{fb}^{-1}$. Three sets of charge
mis-identification rates are measured and provided to ATLAS analyses,
corresponding to three different sets of selection requirements (Medium, Tight
+ Isolation, and Tight + Isolation + impact parameter). The rates show a
variation from less than $1\%$ to nearly $10\%$ depending on the bins. It is
also observed from the measurements that in general simulation underestimates
the charge mis-identification rates as compared to those in the data.

In Run 2, in addition to measuring the charge mis-identification rates, a
separate effort was started by the physics team at Universit\'{e} de
Montr\'{e}al, aiming at reducing charge mis-identification. The technique
relies on the output of a boosted decision tree using a simulated sample of
single electrons. Figure~\ref{f:meeratesbdt} shows the impact of applying the
BDT requirement on charge mis-identification rates. More details may be found
in~\cite{atlaselcid}.

\begin{figure}[H]
	\includegraphics[width=10cm]{figures/ch5fig_15a}
	\includegraphics[width=10cm]{figures/ch5fig_15b}
	\centering

	\caption{Charge mis-identification probabilities in 2016 data and $Z\to e^+e^-$ 
	events as a function of $E_T$ (top) and $\abs{eta}$ (bottom) that shows also 
	the impact of applying the BDT requirement (in blue) to suppress charge mis-identification. 
	}

	\label{f:meeratesbdt}
\end{figure}
