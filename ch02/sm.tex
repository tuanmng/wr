
\chapter{THE STANDARD MODEL OF PARTICLE PHYSICS AND SUPERSYMMETRY}\label{c:sm}

The LHC was designed to explore Higgs physics and some beyond-Standard-Model
physics accessible at the TeV scale. Three among the four currently known
fundamental interactions\footnote{Gravitational interactions are negligible at
the LHC energy scale. However, as will be discussed later in the chapter, they
are no longer negligible at the Planck scale where fundamental questions not
only about gravity but also about other aspects of the Standard Model arise.},
namely quantum electrodynamics (QED), the weak and the strong interactions
(also known as quantum chromodynamics, or QCD), are expected to come into full
play --- indeed abundantly, providing physicists with ample opportunities to
explore what had not been possible before. The theoretical foundation
underlying these interactions has been worked out by physicists in the second
half of the $20$th century and laid out in the Standard Model of particle
physics, a mathematical framework that makes possible quantitative predictions
of particle interactions. This chapter touches on some aspects of the Standard
Model, including the principle of symmetry, which is an important organizing
principle of the Standard Model.

The Standard Model, on the other hand, has a number of problems, and some of
them are reviewed in the present chapter. Supersymmetry, one among several
attempts to go beyond the Standard Model, provides solutions to these problems;
it has been and still is being actively pursued at the LHC. The theory of
supersymmetry, including some key points on phenomenology, is discussed briefly
at the end of the chapter.


\section{The Standard Model of Particle Physics}

Elementary particles and their fundamental interactions, excluding gravity, are
fully represented in the Standard Model~\cite{smtheo01, smtheo02, smtheo03}.
The concept of symmetry plays an essential role, as each component of the
Standard Model --- QED, the weak interaction, and the strong interaction ---
can be obtained by imposing an appropriate symmetry, a so-called local gauge
symmetry; in addition, each component also respects spacetime
symmetries~\cite{smtheo04}. Section~\ref{s:smsymmetries} discusses briefly
spacetime symmetries as well as the local gauge symmetries that underline the
Standard Model, while Section~\ref{s:smpf} discusses in more detail the
particle contents of the Standard Model as well as several of its aspects.

\subsection{Symmetries}\label{s:smsymmetries}

A symmetry is mathematically represented by a group. Spacetime symmetries are
represented by the Poincar\'{e} group, and local gauge symmetries local gauge
groups, both of which are discussed in the following.

\subsubsection{The Poincar\'{e} Group}\label{sm:poincareg}

Spacetime symmetries include the Lorentz symmetry, which refers to the
equivalence between inertial observers with regard to physical laws. If $K$ and
$K'$ are inertial frames, the relativity principle requires physical laws as
observed in $K$ to be also physical laws as observed in $K'$. Thus, let $x, y,
	z$ and $t$ be the coordinates of an event as measured in $K$, and $x', y', z'$
and $t'$ those of the same event as measured in $K'$. According to special
relativity, these coordinates are related by a Lorentz transformation

$${x'}^\mu = \Lambda^\mu_\nu x^\nu$$

The Lorentz transformations form a group, called the Lorentz group. It is made
up of transformations that leave the four-dimensional distance


$$\eta_{\mu\nu} x^\mu x^\nu = (ct)^2 - x^2 -y^2 -z^2 $$

invariant. Physically, Lorentz symmetry concerns rotations and boosts between
inertial frames.

To the Lorenzt symmetry may be added possible displacements of the origins of
the frames, including time and spatial displacements,

$${x'}^\mu = x^{\mu} + a^{\mu} $$

In this way, we obtain the full spacetime symmetry, or Poincar\'{e} symmetry.
The corresponding group is called the Poincar\'{e} group.

A physical theory is said to be constrained by spacetime symmetry if its
Lagrangian is invariant with respect to the Poincar\'{e} group. This concept
will be illustrated below when we discuss gauge groups.

\subsubsection{Gauge Groups}

The known fundamental interactions in the Standard Model, namely quantum
electrodynamics, the weak force, and the strong force, have been discovered to
follow the principle of symmetry, in the sense that each of them can be
obtained when an appropriate symmetry is required. The corresponding symmetry
groups can be classified into abelian groups and non-abelian groups. In this
section we review some general considerations in the use of such groups.



\paragraph{The Abelian group $U(1)$} The global $U(1)$ group arises when we
consider a transformation of the form

\begin{equation}\label{eq:uoneglobal}
	\psi \to e^{i\theta}\psi,
\end{equation}

where $\theta$ is a real number. The gauge principle turns $\theta$ into a
function of spacetime coordinates, $\theta(x)$, and the resulting
transformation is called a local gauge transformation. The Lagrangian of the
theory is then required to be invariant under the local gauge transformation.

We will consider as an example the Dirac Lagrangian

$$L = \bar{\psi}(i\gamma^{\mu}\partial_{\mu} - m)\psi,$$

which conforms to spacetime symmetry, but which is not invariant under the
local gauge transformation, since the transformed Lagrangian is

$$ L' = L + \bar{\psi} \gamma_{\mu} \psi(\partial^{\mu}\theta).$$

To come up with the new Lagrangian that would be invariant, we introduce a new
field, called a gauge field and denoted $A_{\mu}$, together with the covariant
derivative operation

$$ D_{\mu} = \partial_{\mu} + ieA_{\mu},$$

and the requirement that under the local gauge transformation, the $A_{\mu}$
would have to transform according to

$$A_{\mu} \to A'_{\mu} = A_{\mu} + \frac{1}{e}\partial_{\mu}\theta.$$

The kinematics of $A_{\mu}$ will be taken into account in the new Lagrangian
through the term

$$ -\frac{1}{4}F_{\mu\nu}F^{\mu\nu},$$

where, by definition,

$$F_{\mu\nu} = \partial_{\mu}A_{\nu} - \partial_{\nu}A_{\mu}.$$

The new Lagrangian that would be invariant under the local gauge transform is

$$L = \bar{\psi}(i\gamma^{\mu}D_{\mu} - m)\psi -\frac{1}{4}F_{\mu\nu}F^{\mu\nu}.$$

This Lagrangian is actually the Lagrangian of QED. The field $A_{\mu}$
represents the photon. If the covariant derivative is expanded we will see in
the Lagrangian the term $e\bar{\psi}\gamma_{\mu}\psi A^{\mu}$ which involves
not only the photon term $A_{\mu}$ but also the charged fermion terms
$\bar{\psi}$ and $\psi$; it represents the elementary electromagnetic
interaction and expresses the fact that at the most fundamental level currently
known, electrodynamic interactions are to be understood in terms of one simple
elementary interaction that always involves a photon and a pair of charged
fermions. This knowledge is also usually expressed graphically in
Figure~\ref{f:smqedvertex} where the wiggly line represents the photon, and the
two straight lines with arrows represent the charged particles. This single
diagram encodes different possibilities, we may understand it as the
annihilation of two charged particles in which a photon is seen at the end, or
a process in which a charged particle radiates a photon and turns to an
antiparticle, or a process where the photon radiates a pair of
particle-antiparticle.\footnote{The last process, more particularly photon to
	electron-positron pair, is important to physics at the LHC.}


\begin{figure}[h]
	\includegraphics[width=8cm]{figures/feynman_QED_vertex}
	\centering
	\caption{The Elementary QED vertex}
	\label{f:smqedvertex}
\end{figure}

\paragraph{Non-Abelian Groups $SU(n)$}

$SU(n)$ may be understood as the group of all $n\times n$ unitary matrices
whose determinants equal $1$. Such a matrix, say $U$, may be written in the
form

$$U = \exp\bigg(-i \frac{T^a}{2}\alpha^a \bigg) $$

The gauge principle turns the parameters $\alpha^a$'s into functions of
spacetime coordinates $\alpha^a(x)$'s. The $T^{\alpha}$ are called the
generators of the group and satisfy the relations

$$[T^a, T^b] = if^{ab}_cT^c $$

As in the case of QED discussed earlier, in order to write down a Lagrangian
that would be invariant under the local gauge transformation $SU(n)$, we are
forced to introduce new fields that represent bosons in the theory, the number
of bosons correspond to the number of generators of the group, plus the
covariant derivative operation and the kinematic terms involving these fields.
The possible elementary interactions of the theory may then be read off by
looking at the various terms in the Lagrangian. In QCD, for example, where the
gauge group is $SU(3)$, there are eight generators that correspond to eight
gluons in the theory. Here, in addition to a quark-gluon vertex, in
Figure~\ref{f:qcdv1}, of the type seen in QED, there are the three-point vertex
and four-point vertex, in Figure~\ref{f:qcdv23}, that correspond to gluon
self-interactions; these additional self-interactions are characteristic of
non-Abelian interactions.

\begin{figure}[H]
	\includegraphics[width=6cm]{figures/qcdv1}
	\centering
	\caption{The QCD quark-gluon vertex}
	\label{f:qcdv1}
\end{figure}

\begin{figure}[H]
	\includegraphics[width=6cm]{figures/qcdv2}
	\includegraphics[width=6cm]{figures/qcdv3}
	\centering
	\caption{The QCD gluon self-interactions}
	\label{f:qcdv23}
\end{figure}


\subsection{The Standard Model Particles and Forces}\label{s:smpf}

The Standard Model is the quantitative implementation of the idea that physics
is to be understood in terms of a small number elementary particles and their
fundamental interactions. The elementary particles are classified into fermions
and bosons. Fermions are further classfied into three families, each family is
made up of a pair of leptons and two quarks; they are listed in
table~\ref{t:smfparticles}.


\begin{table}[H]
	\centering
	\begin{tabular}{l l c l c l c l c}
		& \multicolumn{4}{c}{Leptons} & \multicolumn{4}{c}{Quarks} \\
		\hline\hline
		    & Particle        &              & Mass          & Charge & Particle &     & Mass         & Charge         \\
		\hline\hline
		I   & electron        & $e$          & $0.511$ MeV   & -1     & Up       & $u$ & $2.3$ MeV    & $+\frac{2}{3}$ \\
		    & $e$ neutrino    & $\nu_e$      & $<2$ eV       & 0      & Down     & $d$ & $4.8$ MeV    & $-\frac{1}{3}$ \\
		\hline
		II  & muon            & $\mu$        & $105.658$ MeV & -1     & Charm    & $c$ & $1.275$ GeV  & $+\frac{2}{3}$ \\
		    & $\mu$ neutrino  & $\nu_{\mu}$  & $<2$ eV       & 0      & Strange  & $s$ & $95$ MeV     & $-\frac{1}{3}$ \\
		\hline
		III & tau             & $e$          & $1776.82$ MeV & -1     & Top      & $t$ & $173.07$ GeV & $+\frac{2}{3}$ \\
		    & $\tau$ neutrino & $\nu_{\tau}$ & $<2$ eV       & 0      & Bottom   & $b$ & $4.18$ MeV   & $-\frac{1}{3}$ \\
		\hline
	\end{tabular}
	\caption{The Standard Model fermions. All are spin $1/2$ particles}
	\label{t:smfparticles}
\end{table}

The only difference between the families is the masses of the particles, and it
is still an open question why three families in fact exist.

The bosons, on the other hand, mediate the forces between the fermions. They
carry integer spins, and are listed in table~\ref{t:smbparticles}.

\begin{table}[H]
	\centering
	\begin{tabular}{l c l c c}
		Particle  &          & Mass        & Charge  & Spin \\
		\hline\hline
		Photon    & $\gamma$ & \_          & 0       & 1    \\
		$W^{\pm}$ &          & 80.385 GeV  & $\pm 1$ & 1    \\
		$Z$       &          & 91.1876 GeV & 0       & 1    \\
		Gluon     & $g$      & \_          & 0       & 1    \\
		Higgs     & $h$      & 125.9 GeV   & 0       & 0    \\
		\hline
	\end{tabular}
	\caption{The Standard Model bosons. All have integer spins}
	\label{t:smbparticles}
\end{table}


The fundamental interactions are QED, QCD, and weak interactions, mediated by
the photon, the gluons, and the weak gauge bosons respectively. QED and the
weak theory have been unified into a single electroweak theory. The relevant
gauge groups are $SU(2) \otimes U(1)$ for electroweak and $SU(3)$ for QCD. It
is a general property of gauge theories that the gauge bosons are massless. The
photon and the gluons are massless, but the weak bosons are not. This
discrepancy was resolved with a mechanism known as spontaneous symmetry
breaking~\cite{higgsssb, higgsssb1, higgsssb2}. The result is the introduction
of a new scalar field, the Higgs field, whose interactions with elementary
particles would give them masses. Electroweak symmetry is said to be broken
into QED symmetry around the energy $246$ GeV. The Higgs particle was
discovered in 2012~\cite{higgsdis01, higgsdis02}, its mass has been measured to
be $\sim 125$ GeV, giving a confirmation of electroweak unification and, at the
same time, motivates the hope for the unification of all three Standard Model
interactions.

\section{Beyond the Standard Model}\label{s:smsusy}

The Standard Model has been tested very extensively in terms of its
quantitative predictions of elementary particle interactions, and has hitherto
withstood all the tests. It, however, is not a physics theory where many of our
important questions about the physical world can be understood satisfactorily.
This section discusses some problems that cannot be answered within the
framework of the Standard Model; it also discusses supersymmetry, an attempt to
go beyond the Standard Model to address some of the questions that we still
have.

\subsection{Problems with the Standard Model}

\paragraph{Gravity} Gravitational interactions were first proposed by Newton. A
mass $M$ was postulated to exert an attractive force on another mass $m$, given
by

$$ \Fv = -G \frac{Mm}{r^2}\hat{\rv} $$

where $\hat{\rv}$ is the unit vector pointing from $M$ to $m$, and $G$ the
gravitational constant. Einstein proposed a fundamental change to gravitational
interactions where forces are completely eliminated. In general relativity
there is a direct link between the distribution of matter and energy in a
spacetime region to the geometry of spacetime, the link being given according
to the equation

$$G_{\mu\nu} = -\kappa T_{\mu\nu} $$

where $\kappa = 8\pi G / c^4$, and

$$G_{\mu\nu} = R_{\mu\nu} - \frac{1}{2}R g_{\mu\nu}$$

is the Einstein tensor, which has been written in terms of the Ricci tensor
$R_{\mu\nu}$ and the curvature scalar $R$, both of which are functions of the
metric tensor $g_{\mu\nu}$ which characterizes a geometry. In this new scheme
the paths of objects follow the geodesics of a spacetime geometry.

At present, gravitational interactions are not accounted for in the Standard
Model. Thus, as long as we are still searching for a single, all-encompassing
theoretical framework to address all of our questions about the physical world,
the Standard Model is not a complete theory.

\paragraph{The hierachy problem~\cite{hier01, hier02}} At the energy scale of
the LHC, gravitational interactions are completely negligible. A rough
comparison between the gravitational force between two equal masses $M$
separated by a distance $r$, which is $GM^2 / r^2$, with the electrostatic
force between two charges $\abs{e}$ separated by a distance $r$, which is $e^2/
	r^2$, will indicate the relative weakness of gravity. Indeed, taking as the
unit of mass $Mc^2 = 1$ GeV, the electromagnetic coupling

$$ \alpha = \frac{e^2}{4\pi \hbar c} = \frac{1}{136.036}$$

is to be compared with

$$ \frac{GM^2}{4\pi\hbar c} = 5.3\times 10^{-40}.$$

It follows that gravity is negligible at the GeV or TeV scale. In fact,
gravitational interactions are negligible up to the Planck scale $(hc/ G)^{1/2}
	\sim 10^{19}$ GeV. Given that there are only four forces currently known that
span from the scale of a few hundreds GeV to the Planck scale, it might seem
reasonable to assume that the Standard Model physics is valid up to Planck
scale, i.e. there is no new physics up to Planck scale. However, it has been
pointed out that this assumption leads to the following issue.

In quantum field theory the physical mass of an elementary particle is a sum of
its bare mass plus corrections due to interactions. The Higgs is
self-interacting and due to its mass receives a major correction from
self-interaction; in addition it receives a major correction from its
interaction with the top quark, the heaviest Standard Model particle. If $\mu$
denotes the Higgs mass, $\mu_B$ its bare mass, then the corrections have been
determined to take the form

$$\mu^2  \simeq \mu_{B}^2 +  \frac{\lambda}{8\pi^2}\Lambda^2 -
	\frac{3y_t^2}{8\pi^2}\Lambda^2 + \dots $$

where $\Lambda$ is the momentum scale up to which corrections are applied,
$\lambda$ is the Higgs coupling strength, and $y_t$ is the coupling strengh
between the top and the Higgs. If $\Lambda$ is taken to be the Planck scale,
the corrections have to be extremely precise to fit the physical Higgs mass
$\mu \sim 100$ GeV. This has been judged to be very unnatural; in addition, the
fact that the corrections are so much bigger than the Higgs mass itself has
also been considered unsatisfactory.


\paragraph{Dark matter} Astronomical and cosmological measurements accumulated
over the years~\cite{darkmt01, darkmt02, darkmt03, darkmt04} have argued
overwhelmingly for the inadequacy of ordinary matter to account for the total
matter in the universe. Indeed, it is currently estimated that Standard Model
particles account for about only $5\%$ of all matter in the universe, while
dark matter and dark energy account for the rest, about $27\%$ and $68\%$
respectively~\cite{darkmt05, darkmt06}. At present, the nature of dark matter
is still unknown, even though there have been many indirect cosmological
evidences that point to its existence. Thus, for instance, theoretically we
expect to see smaller rotational velocity of objects that are increasingly
distant from the galaxy to which they belong, shown by the dashed line in
Figure~\ref{f:darkmatterrot}. Actual measurements, however, have shown that the
rotational curve is rather flat, as indicated by the solid line in the same
figure. It is thus concluded that there is invisible mass that not only cannot
be seen but is also distributed differently from ordinary matter.

There are many other examples as well, among which gravitational lensing
furnishes another convincing evidence that indicates the existence of dark
matter. The amount of deflection of light from distant galaxies may be used to
estimate the amount of matter in the galaxy clusters between the Earth and the
distant galaxies, and has led to the conclusion that the galaxy clusters are in
co-existence with an enormous amount of dark matter~\cite{darkmt01, darkmt02,
	darkmt03, darkmt04, darkmt05, darkmt06}.


\begin{figure}[H]
	\includegraphics[width=6cm]{figures/mond}
	\centering
	\caption{The rotational velocity of spiral galaxy with distance~\cite{darkmfig01}.}
	\label{f:darkmatterrot}
\end{figure}

The majority of dark matter is expected to be cold dark matter made up of
unrelativistic particles. Most Standard Model particles are not dark matter
candidates except neutrinos, which are both stable and weakly interacting.
However, neutrinos are relativistic particles and might only account for the
so-called hot dark matter, which is only a fraction of the total amount of dark
matter.

\subsection{Supersymmetry}


Supersymmetry~\cite{Golfand:1971iw,Volkov:1973ix,Wess:1974tw,Wess:1974jb,Ferrara:1974pu,Salam:1974ig}
is an extension of the Standard Model that offers potential solutions to many
currently unsolved problems~\cite{susysol01, susysol02, susysol03, susysol04,
	susysol05}. It started with the question whether or not spacetime symmetry, the
Poincar\'{e} group, can be extended in a non-trivial way. This is to be
contrasted with gauge symmetries, which are trivial extensions of spacetime
symmetries, in the sense that the generators of the gauge groups commute with
the generators of the Poincar\'{e} group.

This section gives a brief discussion of supersymmetry. For a full reference,
see~\cite{susytext01}.

\paragraph{The Poincar\'{e} Algebra and Supersymmetry} The Poincar\'{e} group,
as discussed in Section~\ref{sm:poincareg}, is made up of the Lorentz group and
the group of spacetime translations. The elements of the groups are functions
of ten real continuous parameters, six coming from rotations and boosts in the
Lorentz group, and four from the translation group. Mathematically the
Poincar\'{e} group is associated with a set of ten generators, six associated
with the Lorentz group and usually denoted $J^{\mu\nu}$, and four associated
with the translation group, which will be denoted $P^{\mu}$. Among these
generators there exist commutation relations

$$[P^\mu, J^{\rho\sigma}], \quad [P^{\mu}, P^{\nu}], \quad [J^{\mu\nu},
		J^{\rho\sigma}]$$

whose expressions involve only the generators $J^{\mu\nu}$ and $P^{\mu}$. The
generators and the commutation relations are said to form the Poincar\'{e}
algerba. The question of the extension of the Poincar\'{e} group becomes the
question of whether or not new generators could be added to the existing set of
generators, such that the new commutation relations that arise are not all
trivial, and that they are expressions that involve only the old and the new
generators.

The Poincar\'{e} algebra was found to be extensible, but on the condition that,
when adding the new generators, we have to consider not only the commutation
relations between the old and the new generators, but also the anticommutation
relations among the new generators themselves. The result is a set of
generators and commutation and anticommutation relations among them that form a
system called the super-Poincar\'{e} algebra. One of the consequences that
follows is that the new generators map bosons into fermions and vice versa.
Theoretical considerations then require supersymmetric theory to contain only
two possible multiplets, the chiral supermultiplet that consists of two scalar
and two spinor fields, or the vector supermultiplet that consists of two spinor
and two vector fields. It was found necessary, also on theoretical ground, to
introduce one or more new particles for every Standard Model particle that
differs by spin $1/2$, called its superparners. The particles in a multiplet
otherwise have the same mass and other quantum numbers.

Supersymmetry may be classified depending to the number of new generators that
are added to the Poincar\'{e} generator. The case where there is only one new
generator added is called $N=1$ supersymmetry. The Minimal Supersymmetric
Standard Model (MSSM) is $N=1$ supersymmetry~\cite{susytext01, susytext02}, it
is the extension of the Standard Model with the least possible number of new
particles that need to be introduced. The particle contents are listed in
Table~\ref{t:mssmchiral}.

\begin{table}[H]
	\centering
	\begin{tabular}{l l | l l | c}
		\multicolumn{2}{c}{Boson} & \multicolumn{2}{c}{Fermions} & $SU(3), SU(2), U(1)$ \\
		\hline\hline
		Gluons       & $g$                          & Gluinos   & $\tilde{g}$                    & $(8, 1, 0)$            \\
		Gauge bosons & $W^{\pm}, W^0$               & Gauginos  & $\tilde{W}^{\pm}, \tilde{W}^0$ & $(1, 3, 0)$            \\
		B boson      & $B$                          & Bino      & $\tilde{B}$                    & $(1, 1, 0)$            \\
		\hline
		Sleptons     & $\tilde{\nu}_L, \tilde{e}_L$ & Leptons   & $\nu_L, e_L$                   & $(1, 2, -1)$           \\
		             & $\tilde{\bar{e}}_L$          &           & $\bar{e}_L$                    & $(1, 1, -2)$           \\
		\hline
		Squarks      & $\tilde{u}_L, \tilde{d}_L$   & Quarks    & $u_l, d_L$                     & $(3, 2, \frac{1}{3})$  \\
		             & $\tilde{u}_R$                &           & $u_R$                          & $(3, 1, \frac{4}{3})$  \\
		             & $\tilde{d}_R$                &           & $d_R$                          & $(3, 1, \frac{-2}{3})$ \\
		\hline
		Higgs        & $H_d^0, H_d^-$               & Higgsinos & $\tilde{H}_d^0, \tilde{H}_d^-$ & $(1, 2, -1)$           \\
		             & $H_d^+, H_u^0$               &           & $\tilde{H}_u^+, \tilde{H}_u^0$ & $(1, 2, 1)$            \\
		\hline\hline
	\end{tabular}
	\caption{Particles in the MSSM}
	\label{t:mssmchiral}
\end{table}

If supersymmetry exists, it has to be broken, for otherwise supersymmetric
particles would have been detected alongside Standard Model particles.
Supersymmetry is thought be broken spontaneously, in the same way the Standard
Model electroweak theory is broken spontaneously. A more detail discussion on
supersymmetry breaking is provided in~\cite{susytext01}.

\paragraph{Supersymmetry Phenomenology at the LHC} At the LHC, a class of MSSM
models known as $R$-parity conserving models figure predominantly.
$R$-parity~\cite{susytext01, susytext02} is a multiplicative quantum number
defined by

$$P_R = (-1)^{3(B-L)+2S}$$

where $B$ is the baryon number, $L$ the lepton number, and $S$ the spin. Then
each Standard Model is assigned the value $+1$ while each supersymmetric
particle the value $-1$. Phenomenologically, $R$-parity conservation implies
that

\begin{itemize}
	\item The lightest supersymmetric particle (LSP) is stable;
	\item All other supersymmetric particles decay into a state that has an odd
	      number of LSPs;
	\item Supersymmetric particles are produced in pairs.
\end{itemize}

The LSP does not participate in known interactions and manifests as missing
transverse energy, it is a good candidate for dark matter. At the LHC, both
strong and electroweak interactions are expected to be sources of
supersymmetric particles. Many searches for supersymmetric particles have been
carried out since the start of the LHC, one of which will be discussed in
Chapter~\ref{c:susys} of this thesis.

