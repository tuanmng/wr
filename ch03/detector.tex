\chapter{LHC AND THE ATLAS DETECTOR}\label{c:detector}

Among the principal instruments of modern experimental particle physicists are
accelerators and detectors. This chapter begins with a general introduction to
two basic collider parameters, the centre-of-mass energy and luminosity. Then,
CERN is introduced and the Large Hadron Collider (LHC) is presented. Finally,
the ATLAS detector, including its trigger system, is discussed.


\section{BASIC COLLIDER  PARAMETERS}

Modern collider experiments make use of accelerators and detectors, where
particles are accelerated to some energy before being made to collide. The
higher is the energy, the higher is the mass scale which an experiment can
reach. Accordingly, an important collider parameter is the centre-of-mass
energy.

\subsection{Centre-of-Mass Energy}

To begin, consider two particles with four-momenta $p_1$ and $p_2$. The total
momentum squared, which is a Lorentz invariant quantity, is

$$
	s = (p_1 + p_2)^2  = (E_1 + E_2)^2 - (\pv_1 + \pv_2)^2.
$$

In the centre-of-mass frame, the three-momenta are opposite, and $\pv_1+\pv_2 =
	0$. As a result, assuming the particles have the same energy $E$, we may write

\begin{equation}
	\sqrt{s} = 2E.
\end{equation}

This quantity, $\sqrt{s}$, is referred to as the centre-of-mass energy. The
higher are the energies of the participating particles, the higher is the
centre-of-mass energy. Modern colliders are therefore designed to be able to
accelerate the colliding particles to a very high energy.

A discussion of the LHC accelerator at CERN will be given in later sections.

\subsection{Luminosity}

In addition to the centre-of-mass energy, luminosity is also an important basic
collider parameter. The higher is the luminosity, the better is the chance a
clean signal may be extracted out of the undesired backgrounds. Indeed, given a
physics cross section $\sigma$, the number of events $N$ that is produced is
given by

$$N  = L \sigma$$

where the quantity $L$ is referred to as the luminosity. The number of events
is thus directly proportional to the luminosity.

We often speak of the number of events produced per unit time. Then, the
relevant quantity is the instantaneous luminosity $L_I$, and

$$ L = \int L_I dt $$

Colliders are built to achieve high luminosity. Consider two colliding beams
with $N_1$ and $N_2$ number of particles. A general formula for the
instantaneous luminosity, assuming Gaussian profiles of the beams, is

$$L_I = f \frac{N_1N_2}{4\pi \sigma_x\sigma_y} $$

where

\begin{itemize}
	\item $f$ is the frequency at which the beams collide

	\item $\sigma_x$ and $\sigma_y$ are the root-mean-square horizontal and
	      vertical beam sizes.

\end{itemize}

On the other hand, we need to deal with so-called pileup events that come from
high luminosity. They are undesired events on top of the hard scattering, and
may occur in two scenarios. Either many interactions occur in each collision,
in which case we have in-time pileup, or interactions that belong to different
collisions are (incorrectly) recorded together, in which case we have
out-of-time pileup.


%%%%%%%%%%%%%%%%%%%%%%%%%%%%%%%%%%%%%%%%%%%
\section{CERN AND THE LARGE HADRON COLLIDER}


Broadly speaking, the development of physics rests upon two sources, the first,
the availability of a set of physical phenomena and the second, active
theoretical investigations. At the turn of the $20$th century, atomic phenomena
confirmed the discrete nature of physical quantities previously thought to be
continuous and motivated the development of quantum mechanics. Subsequently,
the quest to unify quantum mechanics and special relativity, in parallel with
probes into sub-atomic phenomena, led to the development of quantum field
theory and eventually gave birth to the Standard Model of particle physics. At
present, more than ever before, both technological advances and active
theoretical investigations are being pushed to the limit to scrutinize the
Standard Model and go beyond it. In this respect, the Large Hadron Collider
based at CERN has been playing a leading role, being the most powerful collider
at the moment.



%%%%%%%%%%%%%%%%%%%%%%%%%%%%%%%%%%%%%%%%%%%
\subsection{CERN}

CERN \cite{cernlink}, also known as European Organization for Nuclear Research,
was established in the post-war era, the $1940$s, to foster physics development
and scientific collaboration in Europe.

CERN's core mission is fundamental physics, to uncover what the universe is
made of and how it works. It relies on the participation and funding of a
number of parties, which are classified into

\begin{itemize}

	\item Member states: These are countries that provide financial and managerial
	      assistance and responsibility to CERN.

	\item Non-member states: These are countries that contribute to the financing,
	      construction, and operation of the experiments on which they collaborate.

	\item Observers: These are countries that have made significant contributions
	      to the CERN infrastructure, and organizations which maintain close links with
	      CERN.

\end{itemize}

In addition, CERN has scientific contact with a number of countries not
belonging to the above groups.


%%%%%%%%%%%%%%%%%%%%%%%%%%%%%%%%%%%%%%%%%%%
\subsection{The Large Hadron Collider}

The Large Hadron Collider \cite{lhclink} was designed to explore physics beyond
the Standard Model. It reuses the Large Electron Position (LEP) tunnel, $26.7$
km in circumference, that was built in the period 1984-1989 at CERN. The tunnel
lies between $45$ m and $170$ m underground.

The LHC has two rings with counter-rotating proton beams. The beams are
accelerated by a high-frequency standing wave, and by design take the form of
bunches of particles. The beam particles are kept along a circular trajectory
and focused near the collision points using dipole and quadrupole magnets.
Notable at the LHC is the use of superconducting magnets that operate at 2K and
lower.

The LHC is part of a system that consists of one accelerator and four
detectors. The designed centre of mass energy is 14 TeV. At this energy, Higgs
physics and some beyond the Standard Model physics become accessible.

Figure~\ref{f:lhcmap} shows the CERN's accelerator complex. The four detectors
are ATLAS, CMS, ALICE, and LHCb, all located at different collision points.
Among them, the high luminosity experiments are ATLAS and CMS, where the target
instantaneous luminosity is around $L_I=2 \times
	10^{34}\text{cm}^{-12}\text{s}^{-1}$.

In this thesis we will focus exclusively on the ATLAS detector.

\begin{figure}[H]
	\includegraphics[width=12cm]{figures/CERN-accelerator-complex-9}
	\centering
	\caption{CERN's Accelerator Complex} \cite{acccomplex}
	\label{f:lhcmap}
\end{figure}


\section{THE ATLAS DETECTOR}

ATLAS \cite{lhcaccexp} is a general-purpose detector located at one among
several collision points at the LHC. The LHC, at 14 TeV designed centre-of-mass
energy, is capable of probing not only Higgs physics but also some beyond
Standard Model physics. Since the new hypothetical particles are typically
expected to decay to energetic Standard Model particles, ATLAS is designed to
be able to identify and measure important physics objects such as photons,
electrons, muons, taus, hadronic jets, and neutrinos and other weakly
interacting particles in the form of missing transverse energy. In addition,
with regard to jets, it needs to be able to distinguish between heavy flavour
jets (b and c quarks) and other light jets.

An overview of the ATLAS detector is shown in Figure~\ref{f:atlasd}. It is $25$
m in diameter and $44$ m in length, and weights approximately $7000$ tons.

\begin{figure}[H]
	\includegraphics[width=12cm]{figures/The-ATLAS-detector-and-subsystems}
	\centering
	\caption{The ATLAS Detector}\cite{atlasdetector}
	\label{f:atlasd}
\end{figure}


In conformity with modern detector design, ATLAS is made up of a number of
subsystems that surround one another in layers. Innermost is the inner
detector, or the tracker. Next, in order, are the electromagnetic calorimeter,
the hadronic calorimeter, and the muon chamber. These subsystems work in
combination to provide detection capability in many possible physics scenarios.
They are built out of components that are fast, precise, and that can stand
against high radiation. Moreover, they are supplemented by an efficient trigger
system.

The entire detector is nominally forward-backward symmetric with respect to the
interaction point. The magnet configuration, which determines the overall
design of the detector, consists of

\begin{itemize}
	\item A thin superconducting solenoid that surrounds the inner-detector cavity,

	\item Three large superconducting toroids around the calorimeters, arranged
	      with an eight-fold azimuthal symmetry.


\end{itemize}


%%%%%%%%%%%%%%%%%%%%%%%%%%%%%%%%%%%%%%%%%%%



%%%%%%%%%%%%%%%%%%%%%%%%%%%%%%%%%%%%%%%%%%%
\subsection{The ATLAS Coordinate System}

Each nominal interaction is given a coordinate system \cite{lhcaccexp}, where

\begin{itemize}
	\item The origin is taken to be the interaction point;
	\item The $z$-axis is defined by the beam direction
\end{itemize}

Thus the $x-y$ plane is transverse to the beam direction. The positive $x$-axis
points from the interaction point to the centre of the LHC ring. The positive
$y$-axis points upwards.

The following quantities are used to reconstruct the kinematic Lorentz vectors
of the final state particles; some of them are illustrated in
Figure~\ref{f:atlasc}, which also illustrates the ATLAS coordinate system.


\begin{itemize}
	\item The azimuthal angle $\phi$,
	\item The polar angle $\theta$,
	\item The rapidity

	      $$
		      y = \frac{1}{2}\ln\bigg( \frac{E + p_z}{E - p_z}\bigg),
	      $$

	\item The pseudorapidity

	      $$
		      \eta = -\ln\bigg( \tan \frac{\theta}{2}\bigg),
	      $$

	      The pseudorapidity is thus zero on the axis perpendicular to the beam axis, and
	      is infinite along the beam axis. At ATLAS, pseudorapidity is reachable up to
	      $4.9$.

\end{itemize}

\begin{figure}[H]
	\includegraphics[width=12cm]{figures/Figures_T_Coordinate}
	\centering
	\caption{The ATLAS Coordinate System~\cite{}}
	\label{f:atlasc}
\end{figure}


The transverse plan plays an important role and there the transverse momenta
and the transverse energy are defined according to the formulas

$$p_T = p\sin\theta, \qquad E_{\text{T}} = E\sin\theta $$

Distances in the $\eta-\phi$-plane are measured using a quantity called the
angular separation

\begin{equation}\label{eq:angulardr}
	\Delta R = \big(\Delta \eta^2 + \Delta \phi^2\big)^{1/2}
\end{equation}

%%%%%%%%%%%%%%%%%%%%%%%%%%%%%%%%%%%%%%%%%%%
\subsection{The ATLAS Detector Components}



%%%%%%%%%%%%%%%%%%%%%%%%%%%%%%%%%%%%%%%%%%%
\subsubsection{The Inner Detector}\label{s:decinner}

The Inner Detector (ID) \cite{lhcaccexp}, also called the tracker, is built to
reconstruct trajectories of charged particles from which momenta can be
computed. It is capable, at high precision and high resolution, of momentum and
primary vertex measurements. Moreover, it is also able to measure impact
parameters and secondary vertices, and thus is capable of identifying heavy
flavour jets.

Figure~\ref{f:innerd} illustrates the ID. The ID is $2.1$ m in diameter and
$6.2$ m in length.



\begin{figure}[H]
	\includegraphics[width=12cm]{figures/Cut-away-image-of-the-ATLAS-Inner-Detector}
	\centering
	\caption{The ATLAS Inner Detector}
	\label{f:innerd}
\end{figure}


The ID is surrounded by a $2$T magnetic field parallel to the beam axis. It is
this field that deflects the paths of charged particles. The tracks are bent,
and their curvatures are then used to compute momenta of charged particles.

To achieve the desired performance, the ID is built up of semiconductor pixel
and strip detectors in the inner part and straw-tube tracking detectors in the
outer part. The pixel and the strip detectors are silicon detectors.

The ID is divided into three main components. Thus, the Pixel Detector and the
semiconductor tracker (SCT) are to work in combination with the transition
radiation tracker (TRT).

\vspace{5mm}

The Pixel Detector and the SCT work with each other to provide precision
tracking near the interaction point. They cover the region $\abs{\eta}<2.5$. In
terms of arrangement:

\begin{itemize}
	\item In the barrel region, they lie on concentric cylinders around the beam axis.
	\item In the end-cap regions, they lie on disks perpendicular to the beam axis.
\end{itemize}

\paragraph{The Pixel Detector} The Pixel Detector surrounds the beam pipe. Made
up of silicon pixel detectors, it is able to cope with very high track density
that is expected. It is constructed in the form of segmented layers of
identical sensors in the $R-\phi$ plane and along the $z$-axis, where the
sensors are of size $500\times$\SI{400}{\micro\meter^2}. Typically, a track is
expected to cross three such layers. The achieved accuracies in the barrel are
approximately \SI{10}{\micro\meter} in the $R-\phi$ plane and
\SI{115}{\micro\meter} in the $z$ direction. The Pixel Detector has
approximately $80.4$ millions readout channels.



In May 2014, an additional pixel layer was installed between the pixels and the
beam spot, at a distance of $3.3$ cm from the beam pipe. It is called the
insertable B-layer (the IBL \cite{atlasblayer}) and provides an additional $8$
millions pixels. The results are improvements in track reconstruction, vertex
measurement, and b-jet identification.

\paragraph{SCT} A track is typically expected to cross eight strip layers of
the SCT. In the barrel region, the $R-\phi$ coordinates are measured by
small-angle stereo strips that lie along the beam direction, and which are
distributed one set per layer. In the end-cap regions there are two sets of
strips, one running radially and one at a small angle. The accuracies achieved
in the barrel are \SI{17}{\micro\meter} in the $R-\phi$ plane and
\SI{580}{\micro\meter} in the $z$ direction, while those in the disks are
\SI{17}{\micro\meter} in the $R-\phi$ plane and \SI{580}{\micro\meter} in the
$R$ direction. The SCT has approximately $6.3$ millions readout channels in
total.

\vspace{2mm}

The Pixel Detector and the SCT function at small radii.

\vspace{5mm}

\paragraph{TRT} The TRT is only in the $R-\phi$ plane and covers the region
$\abs{\eta} < 2.0$. It consists of straw tubes, 4mm in diameter, that typically
register $36$ hits per track. Each straw tube achieves an accuracy of
\SI{130}{\micro\meter}.

In terms of arrangement:

\begin{itemize}

	\item In the barrel region, the straw tubes are manufactured at length $144$ cm
	      and lie parallel to the beam axis.

	\item In the end-cap regions, they are manufactured at length $37$ cm and lie
	      radically, in wheels.

\end{itemize}

The TRT is installed at a larger radius, and has approximately $351000$ readout
channels. High precision momentum measurement is achieved with a large number
of measurements and longer track length.

\vspace{5mm}


The ID system supplements the calorimeters and the muon detector, to be
discussed below.

%%%%%%%%%%%%%%%%%%%%%%%%%%%%%%%%%%%%%%%%%%%
\subsubsection{The Calorimeters}

Calorimeters are built to measure energies of particles. As particles traverse
the calorimeters, they interact with the materials in the calorimeters, losing
their energies and exhibiting characteristics that enable themselves to be
identified.

The ATLAS calorimeters \cite{lhcaccexp, atlasdreport} consist of two systems,
the Electromagnetic Calorimeter and the Hadronic Calorimeter. Electromagnetic
particles develop shower shapes in the former, whereas hadronic particles
penetrate the latter and also develop shower shapes.

An outline of the calorimeter system is shown in Figure~\ref{f:emc}. The
required resolutions are

\begin{itemize}

	\item The Electromagnetic Calorimeter: $\sigma_E / E = 10\% /\sqrt{E} \oplus
		      0.7\%$

	\item The Hadronic Calorimeter: $\sigma_E / E = 50\% /\sqrt{E} \oplus 3\%$
	      (barrel and end-cap regions)

\end{itemize}

Each resolution is a quadratic combination of two separation terms, one coming
from the statistical nature of the shower shape, and one constant term
reflecting other uncertainties such as calibration etc.

\begin{figure}[H]
	\includegraphics[width=12cm]{figures/The-calorimeter-system-in-the-ATLAS-experiment-at-the-Large-Hadron-Collider}
	\centering
	\caption{The ATLAS Calorimeter System}\cite{atlasdetector}
	\label{f:emc}
\end{figure}

\paragraph{The Electromagnetic Calorimeter}\label{p:emc} The Electromagnetic
Calorimeter provides electron and photon identification and measurements. These
particles lose their energies mainly through bremsstrahlung, pair production,
and ionization, where the first two dominate for high-energy particles, leading
to the development of shower shapes in the calorimeter.

By design, the Electromagnetic Calorimeter is a lead-LAr (Liquid Argon)
detector. The lead, in the form of lead plates, functions as an absorber, and
the liquid argon is used in the active layers.


\vspace{3mm}

There are:

\begin{itemize}

	\item The barrel part, covering $\abs{\eta} < 1.475 $, which is $> 22$
	      radiation lengths in thickness.

	\item Two end-cap components, covering $1.375 < \abs{\eta} < 3.2$, where each
	      is $>24$ radiation lengths in thickness.

\end{itemize}

The barrel is made up of two identical half-barrels, with a gap of $4$ mm in
between at $z=0$. On the other hand, each end-cap is made up of an outer wheel
that covers the region $1.375 < \abs{\eta} < 2.5$ and an inner wheel that
covers the remaining region.

There is a presampler detector, which is an active LAr layer of $1.1$ cm in
thickness in the barrel and $0.5$ cm in thickness in each end-cap, covering the
region $\abs{\eta} < 1.8$. It is used to correct the energy lost by electrons
and photons in the materials they traverse before they reach the calorimeter,
such as those in the inner detector.



The Electromagnetic Calorimeter is complemented by the Hadronic Calorimeter,
discussed in more detail below. Together, they contain electromagnetic and
hadronic showers and limit penetration into the muon system.

\paragraph{The Hadronic Calorimeter} The Hadronic Calorimeter surrounds the
Electromagnetic Calorimeter. It is built to measure energy of hadronic
particles, which also show up in the form of showers in the calorimeter.

In terms of thickness, the Hadronic Calorimeter is approximately 10 interaction
lengths in the barrel region as well as in the end-cap regions. This not only
provides a good containment but also helps with measurement of missing
transverse energy.

\vspace{5mm}

The Hadronic Calorimeter is divided into three parts:

\begin{itemize}

	\item Tile calorimeter: A sampling calorimeter, where steel is used as the
	      absorber and scintillating tiles as the active material. It is placed directly
	      outside the EM calorimeter envelope. It has a barrel that covers the region
	      $\abs{\eta} < 1.0$ and two extended barrels which cover the region $0.8 <
		      \abs{\eta} < 1.7$. The barrel as well as the extend barrels have three layers,
	      designed with sufficient interaction lengths.

	\item LAr hadronic end-cap calorimeter: The Hadronic End-cap Calorimeter is
	      also a sampling calorimeter, where copper in the form of plates functions as
	      the absorber and LAr gaps as the active medium. It has two independent wheels
	      per end-cap. The wheels are put directly behind the end-cap electromagnetic
	      calorimeters. Those closest to the interaction points use $25$ mm parallel
	      copper plates, and the rest uses $50$ mm copper plates. The copper plates
	      extend a radius from approximately $0.4$ m to approximately $2$ m, and in
	      between are gaps of LAr materials.

	\item LAr forward calorimeter: The Forward Calorimeter is approximately 10
	      interaction lengths in depth. It has three modules in each end-cap, with one
	      (copper) optimized for electromagnetic measurements and the remaining two
	      (tungsten) hadronic measurements. Each module is a metal matrix of longitudinal
	      channels, where the channels are filled with electrode structure which are in
	      turns made up of concentric rods and tubes parallel to the beam axis, with LAr
	      between them.

\end{itemize}

%%%%%%%%%%%%%%%%%%%%%%%%%%%%%%%%%%%%%%%%%%%
\subsubsection{The Muon Spectrometer}\label{s:decmuons}

The Muon Spectrometer \cite{lhcaccexp} provides muon identification as well as
muon momentum and charge measurement. It surrounds the hadronic calorimeter and
defines the overall dimensions of the ATLAS detector.

The Muon Spectrometer is illustrated in Figure~\ref{f:muons}. It is made up of
three layers of precision tracking chambers plus trigger chambers.

Deflection of the muon tracks is effected by the built-in superconducting
magnets, which is a system of three large air-core toroids. In detail,

\begin{itemize}
	\item The barrel toroid provides deflection over the range $\abs{\eta} < 1.4$

	\item The end-cap magnets at both ends of the barrel toroid, over the range
	      $1.6 < \abs{\eta} < 2.7$

	\item A combination of barrel and end-cap magnets, over the range $1.4 <
		      \abs{\eta} < 1.6$ (also called the transition region)

\end{itemize}

The magnetic field created by this configuration is approximately orthogonal to
the muon trajectories. At the same time, this setting minimizes the effect of
multiple scattering on momentum resolution.


Muon tracks are measured in the tracking chambers. They are the Monitored Drift
Tubes over most of the $\abs{\eta}$-range, and Cathode Strip Chambers at large
$\abs{\eta}$-range. The chambers are arranged in the following manner:

\begin{itemize}
	\item Around the beam axis, in the form of three cylindrical layers

	\item In the transition ($1.4 < \abs{\eta} < 1.6$) and end-cap regions, in
	      three planes perpendicular to the beam

\end{itemize}

\begin{figure}[H]
	\includegraphics[width=12cm]{figures/MuonSystem_d3}
	\centering
	\caption{The ATLAS Muon Spectrometer}
	\label{f:muons}
\end{figure}

The trigger system works in the range $\abs{\eta} < 2.4$. The trigger chambers,
which are Resistive Plate Chambers in the barrel and Thin Gap Chambers in the
end-cap regions, provide the following functionalities:

\begin{itemize}
	\item Bunch-crossing identification

	\item Well-defined transverse momentum thresholds

	\item Measurements muon coordinate in the direction orthogonal to that
	      determined by the precision-tracking chambers.

\end{itemize}

\subsection{The ATLAS Trigger System}

Due to the high luminosity of the LHC, ATLAS, with limited storage capacity and
technology, is only able to record potentially interesting physics events. An
event is classified as potentially interesting or not by a trigger system that
is made up of three distinct levels, L1, L2, and the event filter. L1 is a
hardware level trigger, while L2 and the event filter make up the high-level
software trigger (HTL) at ATLAS.

The trigger system helps reducing the event rate from approximately $1$ GHz at
the designed luminosity of $10^{34}\text{cm}^2\text{s}^{-1}$ to approximately
$200$ Hz.

\subsubsection{The Hardware L1 Trigger}

At L1, a decision is made in less than \SI{2.5}{\micro\second}. L1 helps
reducing the rate to about $75$ kHz.

L1 identifies high transverse-momentum muons, electrons, photons, jets, and
taus that decay into hadrons. It also searches for events with large missing
and total transverse energy. To achieve its purpose, L1 is implemented using
custom-made electronics, and uses low-resolution information from the
calorimeters and the muon spectrometer.

An event passing L1 is forwarded to the subsequent triggers. In addition, the
event is also defined one or more regions of interests --- for examples the
$\eta$ and $\phi$ coordinates where something potentially useful has been seen
--- which will subsequently be checked by the high level trigger.

\subsubsection{The L2 and Event Filters}

The events that pass the L1 trigger are sent to the L2 trigger. L2 reduces the
event rate to approximately $3.5$ kHz and takes, on average, about $40$ ms to
process an event. It looks at the regions of interests defined by L1 where it
uses the best possible information available from the calorimeters and the
muon spectrometer for its selections.

Those events that pass L2 are sent to the event filter, which further reduces
the event rate to about $200$ Hz. Here events are processed using offline
analysis procedures, and on average each event takes approximately four
seconds.





