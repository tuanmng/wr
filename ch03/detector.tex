\chapter{LHC AND THE ATLAS DETECTOR}\label{c:detector}

Among the principal instruments of modern experimental particle physics are
accelerators and detectors. Accelerators accelerate particles to some energy
before colliding them, and subsequently the collision debris in the form of new
particles are collected and analysed in the detectors. A high centre-of-mass
energy, which is required to reach high mass scale physics, leads to many
technical challenges as well as an increasing in the overall complexity of
modern accelerators and detectors.

This chapter discusses the Large Hadron Collider (LHC) based at CERN and the
ATLAS detector, one among the four main detectors located at the LHC.

%%%%%%%%%%%%%%%%%%%%%%%%%%%%%%%%%%%%%%%%%%%
\section{CERN AND THE LARGE HADRON COLLIDER}


Broadly speaking, the development of physics rests upon two sources, the first,
the availability of a set of physical phenomena and the second, active
theoretical investigations. At the turn of the $20$th century, atomic phenomena
confirmed the discrete nature of physical quantities previously thought to be
continuous and motivated the development of quantum mechanics. Subsequently,
the quest to unify quantum mechanics and special relativity, in parallel with
probes into sub-atomic phenomena, led to the development of quantum field
theory and eventually gave birth to the Standard Model of particle physics. At
present, more than ever before, both technological advances and active
theoretical investigations are being pushed to the limit to scrutinize the
Standard Model and go beyond it. In this respect, the Large Hadron Collider
based at CERN has been playing a leading role, being the most powerful collider
at the moment.

CERN \cite{cernlink}, also known as European Organization for Nuclear Research,
was established in the post-war era, the $1940$s, to foster physics development
and scientific collaboration in Europe. The Large Hadron Collider
\cite{lhclink}, built in the period 1984-1989 at CERN, was designed to explore
physics beyond the Standard Model. It is a complex of successive accelerators
that increase the accelerated particle energy by approximately an order of
magnitude at each pass from one accelerator to the next.


The LHC reuses the Large Electron Position (LEP) tunnel, which produced
$e^+e^-$ collisions. It is $26.7$ km in circumference and lies between $45$ m
and $170$ m underground. The designed centre of mass energy is 14 TeV, at which
Higgs physics and some beyond the Standard Model physics become accessible. It
has two rings with counter-rotating proton beams. The beams are accelerated by
a high-frequency standing wave, and by design take the form of bunches of
particles, which are spaced by $25$ ns and each of which contains up to
$1.1\times 10^{11}$ protons. The beam particles are kept along a circular
trajectory and are focused near the collision points using dipole and
quadrupole magnets. Notable at the LHC is the use of superconducting magnets
that operate at 1.9K.

Figure~\ref{f:lhcmap} shows the CERN's accelerator complex. The four main
detectors are ATLAS, CMS, ALICE, and LHCb, all located at different collision
points. Among them, the high luminosity experiments are ATLAS and CMS.

\begin{figure}[H]
	\includegraphics[width=12cm]{figures/CERN-accelerator-complex-9}
	\centering
	\caption{CERN's Accelerator Complex} \cite{acccomplex}
	\label{f:lhcmap}
\end{figure}


Initially in Run-1 (2010-2012), the LHC was operating at $7$ and $8$ TeV
center-of-mass energy. The superconducting beampipe magnets were upgraded
during the long shutdown 2012-2015, helping to reach $13$ TeV center-of-mass
energy in Run-2 (2015-2018). Followng Run-2 is another shutdown (2019-2020),
during which upgrades are performed in preparation for Run-3 (2021-2013).

At the end Run-2 a total of approximately $160\text{ fb}^{-1}$ of data was
delivered by the LHC. This total luminosity, denoted $L$, is proportional to
the number of events $N$ produced for a physics process with cross section
$\sigma$, according to the formula

$$ N = \sigma L$$

The number of events produced per unit time is a function of the instantaneous
luminosity $L_I$ which is related to $L$ by

$$ L = \int L_I dt $$

The LHC was designed to achieve high instantaneous luminosity. The peak
luminosity to date was achived during Run-2 (2018), at $2.0\times 10^{34}\text{
		cm}^{-2}\text{s}^{-1}$. In general, given two colliding beams with $N_1$ and
$N_2$ number of particles, a general formula for the instantaneous luminosity,
assuming Gaussian profiles of the beams, is

$$L_I = f \frac{N_1N_2}{4\pi \sigma_x\sigma_y} $$

where

\begin{itemize}
	\item $f$ is the frequency at which the beams collide

	\item $\sigma_x$ and $\sigma_y$ are the root-mean-square horizontal and
	      vertical beam sizes.

\end{itemize}

On the other hand, we need to deal with so-called pileup events that come from
high luminosity. They are undesired events on top of the hard scattering, and
may occur in two scenarios. Either many interactions occur in each collision,
in which case we have in-time pileup, or interactions that belong to different
collisions are (incorrectly) recorded together, in which case we have
out-of-time pileup.

\section{THE ATLAS DETECTOR}

ATLAS \cite{lhcaccexp} is a general-purpose detector located at one among
several collision points at the LHC. The LHC, at 14 TeV designed centre-of-mass
energy, is capable of probing not only Higgs physics but also some beyond
Standard Model physics. Since the new hypothetical particles are typically
expected to decay to energetic Standard Model particles, ATLAS is designed to
be able to identify and measure important physics objects such as photons,
electrons, muons, taus, hadronic jets, neutrinos, and other weakly interacting
particles in the form of missing transverse energy. In addition, with regard to
jets, it needs to be able to distinguish between heavy flavour jets ($b$ and
$c$ quarks) and other light jets.

Figure~\ref{f:atlasd} shows an overview of the ATLAS detector . It is $25$ m in
diameter and $44$ m in length, and weights approximately $7000$ tons.

In conformity with modern detector design, ATLAS is made up of a number of
subsystems that surround one another in layers. Innermost is the inner
detector, or the tracker. Next, in order, are the electromagnetic calorimeter,
the hadronic calorimeter, and the muon chamber. These subsystems work in
combination to provide detection capability in many possible physics scenarios.
They are built out of components that are fast, precise, and that can stand
against high radiation. Moreover, they are supplemented by an efficient trigger
system.

\begin{figure}[H]
	\includegraphics[width=12cm]{figures/The-ATLAS-detector-and-subsystems}
	\centering
	\caption{The ATLAS Detector}\cite{atlasdetector}
	\label{f:atlasd}
\end{figure}

The entire detector is nominally forward-backward symmetric with respect to the
interaction point. The magnet configuration, which determines the overall
design of the detector, consists of

\begin{itemize}
	\item A thin superconducting solenoid that surrounds the inner-detector cavity,

	\item Three large superconducting toroids around the calorimeters, arranged
	      with an eight-fold azimuthal symmetry.


\end{itemize}


%%%%%%%%%%%%%%%%%%%%%%%%%%%%%%%%%%%%%%%%%%%



%%%%%%%%%%%%%%%%%%%%%%%%%%%%%%%%%%%%%%%%%%%
\subsection{The ATLAS Coordinate System}

Each nominal interaction is given a coordinate system \cite{lhcaccexp}, where

\begin{itemize}
	\item The origin is taken to be the interaction point;
	\item The $z$-axis is defined by the beam direction
\end{itemize}

Thus the $x-y$ plane is transverse to the beam direction. The positive $x$-axis
points from the interaction point to the centre of the LHC ring. The positive
$y$-axis points upwards.

The following quantities are used to reconstruct the kinematic Lorentz vectors
of the final state particles; some of them are illustrated in
Figure~\ref{f:atlasc}, which also illustrates the ATLAS coordinate system.


\begin{itemize}
	\item The azimuthal angle $\phi$,
	\item The polar angle $\theta$,
	\item The rapidity

	      $$
		      y = \frac{1}{2}\ln\bigg( \frac{E + p_z}{E - p_z}\bigg),
	      $$

	      whose difference is invariant with respect to Lorentz boosts along the
	      $z$-direction. This implies that the number of particles produced per unit
	      of rapidity is approximately constant.

	\item The pseudorapidity

	      $$
		      \eta = -\ln\bigg( \tan \frac{\theta}{2}\bigg),
	      $$

	      For a highly-energetic particle $\eta$ almost coincides with $y$ but is much
	      easier to measure.

\end{itemize}

\begin{figure}[H]
	\includegraphics[width=12cm]{figures/Figures_T_Coordinate}
	\centering
	\caption{The ATLAS Coordinate System~\cite{}}
	\label{f:atlasc}
\end{figure}


In the transverse plane perpendicular to the beam-axis the transverse momenta
and the transverse energy are defined according to the formulas

$$p_T = p\sin\theta, \qquad E_{\text{T}} = E\sin\theta $$

Distances in the $\eta-\phi$-plane are measured using a quantity called the
angular separation

\begin{equation}\label{eq:angulardr}
	\Delta R = \big(\Delta \eta^2 + \Delta \phi^2\big)^{1/2}
\end{equation}

%%%%%%%%%%%%%%%%%%%%%%%%%%%%%%%%%%%%%%%%%%%
\subsection{The ATLAS Detector Components}



%%%%%%%%%%%%%%%%%%%%%%%%%%%%%%%%%%%%%%%%%%%
\subsubsection{The Inner Detector}\label{s:decinner}

The Inner Detector (ID) \cite{lhcaccexp}, also called the tracker, is built to
reconstruct trajectories of charged particles from which momenta can be
computed. It is capable, at high precision and high resolution, of momentum and
primary vertex measurements. Moreover, it is also able to measure impact
parameters and secondary vertices, and thus is capable of identifying heavy
flavour jets.

Figure~\ref{f:innerd} illustrates the ID. The ID is $2.1$ m in diameter and
$6.2$ m in length.



\begin{figure}[H]
	\includegraphics[width=12cm]{figures/Cut-away-image-of-the-ATLAS-Inner-Detector}
	\centering
	\caption{The ATLAS Inner Detector}
	\label{f:innerd}
\end{figure}


The ID is surrounded by a $2$T magnetic field parallel to the beam axis. This
field deflects the paths of charged particles the curvatures of which are used
for the measurements of their momenta and charges.

To achieve the desired performance, the ID is built up of semiconductor pixel
and strip detectors in the inner part and straw-tube tracking detectors in the
outer part. The pixel and the strip detectors are silicon detectors.

The ID is divided into three main components. Thus, the Pixel Detector and the
semiconductor tracker (SCT) are to work in combination with the transition
radiation tracker (TRT).

\vspace{5mm}

The Pixel Detector and the SCT work with each other to provide precision
tracking near the interaction point. They cover the region $\abs{\eta}<2.5$. In
terms of arrangement:

\begin{itemize}
	\item In the barrel region, they lie on concentric cylinders around the beam axis.
	\item In the end-cap regions, they lie on disks perpendicular to the beam axis.
\end{itemize}

\paragraph{The Pixel Detector} The Pixel Detector surrounds the beam pipe. Made
up of silicon pixel detectors, it is able to cope with very high track density
that is typically expected. It is constructed in the form of segmented layers of
identical sensors in the $R-\phi$ plane and along the $z$-axis, where the
sensors are of size $50\times$\SI{400}{\micro\meter^2}. Typically, a track is
expected to cross three such layers. The achieved accuracies in the barrel are
approximately \SI{10}{\micro\meter} in the $R-\phi$ plane and
\SI{115}{\micro\meter} in the $z$ direction. The Pixel Detector has
approximately $80.4$ millions readout channels.



In May 2014, an additional pixel layer was installed between the pixels and the
beam spot, at a distance of $3.3$ cm from the beam pipe. It is called the
insertable B-layer (the IBL \cite{atlasblayer}) and provides an additional $8$
millions pixels. The results are improvements in track reconstruction, vertex
measurement, and b-jet identification.

\paragraph{SCT} A track is typically expected to cross eight strip layers of
the SCT. In the barrel region, the $R-\phi$ coordinates are measured by
small-angle stereo strips that lie along the beam direction, and which are
distributed one set per layer. In the end-cap regions there are two sets of
strips, one running radially and one at a small angle. The accuracies achieved
in the barrel are \SI{17}{\micro\meter} in the $R-\phi$ plane and
\SI{580}{\micro\meter} in the $z$ direction, while those in the disks are
\SI{17}{\micro\meter} in the $R-\phi$ plane and \SI{580}{\micro\meter} in the
$R$ direction. The SCT has approximately $6.3$ millions readout channels in
total.

\vspace{2mm}

The Pixel Detector and the SCT function at small radii.

\vspace{5mm}

\paragraph{TRT} The TRT measures only $R-\phi$ coordinates and covers the
region $\abs{\eta} < 2.0$. It consists of straw tubes, 4mm in diameter, that
typically register $36$ hits per track. Each straw tube achieves an accuracy of
\SI{130}{\micro\meter}. In terms of arrangement:

\begin{itemize}

	\item In the barrel region, the straw tubes are manufactured at length $144$ cm
	      and lie parallel to the beam axis.

	\item In the end-cap regions, they are manufactured at length $37$ cm and lie
	      radically, in wheels.

\end{itemize}

The TRT is installed at a larger radius, and has approximately $351000$ readout
channels. High precision momentum measurement is achieved with a large number
of measurements and longer track length.

\vspace{5mm}


The ID system supplements the calorimeters and the muon detector, which will be
discussed below.

%%%%%%%%%%%%%%%%%%%%%%%%%%%%%%%%%%%%%%%%%%%
\subsubsection{The Calorimeters}

Calorimeters are built to measure energies of particles, using the facts that
particles interact with the detector materials along their paths and cause
radiation in the process. The ATLAS calorimeters \cite{lhcaccexp, atlasdreport}
consist of two systems, the Electromagnetic Calorimeter and the Hadronic
Calorimeter, designed to measure the energies of electrons/photons and hadrons
respectively.

An outline of the calorimeter system is shown in Figure~\ref{f:emc}. The
required resolutions are

\begin{itemize}

	\item The Electromagnetic Calorimeter: $\sigma_E / E = 10\% /\sqrt{E} \oplus
		      0.7\%$

	\item The Hadronic Calorimeter: $\sigma_E / E = 50\% /\sqrt{E} \oplus 3\%$
	      (barrel and end-cap regions), $\sigma_E / E = 100\% /\sqrt{E} \oplus 10\%$
	      (forward region).

\end{itemize}

where each resolution is a quadratic combination of two separation terms, one,
called the stochastic term, that takes into acount the statistical nature of
the shower shape, and one constant term that includes the effects of detector
instabilities and mis-calibration. The effect of the stochastic term decreases
with growing energy.

\begin{figure}[H]
	\includegraphics[width=12cm]{figures/The-calorimeter-system-in-the-ATLAS-experiment-at-the-Large-Hadron-Collider}
	\centering
	\caption{The ATLAS Calorimeter System}\cite{atlasdetector}
	\label{f:emc}
\end{figure}

\paragraph{The Electromagnetic Calorimeter}\label{p:emc} The Electromagnetic
Calorimeter provides electron and photon identification and kinematic
measurements. These particles lose their energies mainly through
bremsstrahlung, pair production, and ionization, where the first two dominate
for high-energy particles, leading to the development of shower shapes in the
calorimeter.

By design, the Electromagnetic Calorimeter is a lead-LAr (Liquid Argon)
detector. The lead, in the form of lead plates, functions as an absorber, and
the liquid argon is used in the active layers.


\vspace{3mm}

There are:

\begin{itemize}

	\item The barrel part, covering $\abs{\eta} < 1.475 $, which is $> 22$
	      radiation lengths in thickness.

	\item Two end-cap components, covering $1.375 < \abs{\eta} < 3.2$, where each
	      is $>24$ radiation lengths in thickness.

\end{itemize}

The barrel is made up of two identical half-barrels, with a gap of $4$ mm in
between at $z=0$. On the other hand, each end-cap is made up of an outer wheel
that covers the region $1.375 < \abs{\eta} < 2.5$ and an inner wheel that
covers the remaining region.

There is a presampler detector, which is an active LAr layer of $1.1$ cm in
thickness in the barrel and $0.5$ cm in thickness in each end-cap, covering the
region $\abs{\eta} < 1.8$. It is used to correct the energy lost by electrons
and photons in the materials they traverse before they reach the calorimeter,
such as those in the inner detector.



The Electromagnetic Calorimeter is complemented by the Hadronic Calorimeter,
discussed in more detail below. Together, they contain electromagnetic and
hadronic showers and limit penetration into the muon system.

\paragraph{The Hadronic Calorimeter} The Hadronic Calorimeter surrounds the
Electromagnetic Calorimeter. It is built to measure energy of hadronic
particles, which also show up in the form of showers in the calorimeter.

In terms of thickness, the Hadronic Calorimeter is approximately 10 interaction
lengths in the barrel region as well as in the end-cap regions. It is divided
into three parts:

\begin{itemize}

	\item Tile calorimeter: A sampling calorimeter, where steel is used as the
	      absorber and scintillating tiles as the active material. It is placed directly
	      outside the EM calorimeter envelope. It has a barrel that covers the region
	      $\abs{\eta} < 1.0$ and two extended barrels which cover the region $0.8 <
		      \abs{\eta} < 1.7$, each made up of three layers.


	\item LAr hadronic end-cap calorimeter: The Hadronic End-cap Calorimeter covers
	      the pseudorapidity range $1.4 < \abs{\eta} < 3.2$. It is also a sampling
	      calorimeter, where copper in the form of plates functions as the absorber and
	      LAr gaps as the active medium. It has two independent wheels per end-cap. The
	      wheels are put directly behind the end-cap electromagnetic calorimeters. Those
	      closest to the interaction points use $25$ mm parallel copper plates, and the
	      rest uses $50$ mm copper plates. The copper plates extend a radius from
	      approximately $0.4$ m to approximately $2$ m, and in between are gaps of LAr
	      materials.

	\item LAr forward calorimeter: The Forward Calorimeter extends the coverage up
	      to $\abs{\eta}=5$, and is approximately 10 interaction lengths in depth. It has
	      three modules in each end-cap, with one (copper) optimized for electromagnetic
	      measurements and the remaining two (tungsten) hadronic measurements. Each
	      module is a metal matrix of longitudinal channels, where the channels are
	      filled with electrode structure which are in turns made up of concentric rods
	      and tubes parallel to the beam axis, with LAr between them.

\end{itemize}

%%%%%%%%%%%%%%%%%%%%%%%%%%%%%%%%%%%%%%%%%%%
\subsubsection{The Muon Spectrometer}\label{s:decmuons}

Muons, being more massive than electrons (approximately $200$ times), are
subject to reduced bremsstrahlung as compared to electrons, and are able to
pass through the ATLAS calorimeters with minimal interactions. The Muon
Spectrometer \cite{lhcaccexp} provides muon identification as well as muon
momentum and charge measurement. It is capable of identifying muon candidates
with $p_T$ from $3$ GeV and has a design resolution of $10\%$ for muons with
$p_T = 1$ TeV.

The Muon Spectrometer surrounds the hadronic calorimeter and defines the
overall dimensions of the ATLAS detector (Figure~\ref{f:muons}). It is made up
of three layers of precision tracking chambers plus trigger chambers.

Deflection of the muon tracks is effected by the built-in superconducting
magnets, which is a system of three large air-core toroids. In detail,

\begin{itemize}
	\item The barrel toroid provides deflection over the range $\abs{\eta} < 1.4$

	\item The end-cap magnets at both ends of the barrel toroid, over the range
	      $1.6 < \abs{\eta} < 2.7$

	\item A combination of barrel and end-cap magnets, over the range $1.4 <
		      \abs{\eta} < 1.6$ (also called the transition region)

\end{itemize}

Muon tracks are measured in the tracking chambers. They are the Monitored Drift
Tubes over most of the $\abs{\eta}$-range, and Cathode Strip Chambers at large
$\abs{\eta}$-range. The chambers are arranged in the following manner:

\begin{itemize}
	\item Around the beam axis, in the form of three cylindrical layers

	\item In the transition ($1.4 < \abs{\eta} < 1.6$) and end-cap regions, in
	      three planes perpendicular to the beam

\end{itemize}

\begin{figure}[H]
	\includegraphics[width=12cm]{figures/MuonSystem_d3}
	\centering
	\caption{The ATLAS Muon Spectrometer}
	\label{f:muons}
\end{figure}

The trigger system works in the range $\abs{\eta} < 2.4$. The trigger chambers,
which are Resistive Plate Chambers in the barrel and Thin Gap Chambers in the
end-cap regions, provide the following functionalities:

\begin{itemize}
	\item Bunch-crossing identification

	\item Well-defined transverse momentum thresholds

	\item Measurements muon coordinate in the direction orthogonal to that
	      determined by the precision-tracking chambers.

\end{itemize}

\subsection{The ATLAS Trigger System}

Due to the high luminosity of the LHC, the ATLAS detector, with limited storage
capacity and technology, is only able to record potentially interesting physics
events. This is achieved in Run-2 with a trigger system made up of two levels,
the hardware level L1 and the High Level Trigger (HTL) software trigger, which
together helps reducing the event rate from approximately $30$ MHz to
approximately $1$ kHz~\cite{atlastrigger}.

L1 helps reducing the rate to about $100$ kHz, taking $2.5$ $\mu\text{s}$
for each event. It identifies high transverse-momentum muons, electrons,
photons, jets, and taus that decay into hadrons. It also searches for events
with large missing and total transverse energy. L1 is implemented using
custom-made electronics, and uses low-resolution information from the
calorimeters and the muon spectrometer. An event passing L1 is defined one or
more regions of interests --- for examples the $\eta$ and $\phi$ coordinates
where something potentially useful has been seen --- before being passed to the
HLT.

On the other hand, the HLT is a software-based trigger. It helps to reduce the
event rate to about $1$ kHz, using information either in the regions of
interests defined by L1 or the whole event. At the HTL each event takes on
average about $200$ms.





