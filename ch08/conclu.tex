\chapter{CONCLUSIONS}\label{c:con}


This thesis presents the work done in association with the ATLAS experiment.
The common theme is electrons, specifically improving the selection of signal
electrons in SUSY searches. The main work is discussed in three chapters:

\begin{itemize}[label=\ding{111}]

\item Chapter~\ref{c:cid} describes the estimation of the charge
mis-identification rates for electrons by a likelihood function, the Poisson
likelihood in particular. These rates are important for new physics searches in
which the final state consists of a pair of same-sign leptons, where the
leptons refer to electrons and muons. The method uses $Z\to e^+e^-$ events,
which furnish a source of clean and high-statistics set of electrons. A Poisson
likelihood function is constructed, taking into account the dependency of
charge mis-identification rates on kinematic properties such as on $p_T$ and on
$\eta$ of the electrons. The results showed that in most bins, simulation
over-estimates the rates as compared to the data by 5-20$\%$.

\item Chapter~\ref{c:susys} describes a SUSY search for gluino pair-production,
which is highly motivated as gluinos are expected by naturalness to have a mass
around the TeV scale, and moreover the production cross section is high at the
LHC. The data is collected in the 2015-2016 data taking period, at
center-of-mass $\sqrt{s} = 13$ TeV and corresponds to an integrated luminosity
of 36.1 $\text{fb}^{-1}$. The final state consists of large missing transverse
momentum and multiple jets, among which at least three must be $b$-jets. The
thesis focuses on the leptonic final state, which requires in addition at least
one lepton (either an electron or a muon). Following a lead from the $t\tbar$
resonance search, a boosted overlap removal procedure between jets and muons is
introduced into the analysis, which is adopted for jets and elecrons in a
subsequent version of the analysis. An optimization for the selection of the
discriminating variables for the leptonic channel is also described. No excess
relative to the Standard Model background is claimed. Model-independent limits
are set on the visible cross-section for new physics processes, and
model-dependent limits are set of gluino and neutralino masses. Gluino masses
of less than $1.97$ TeV for neutralino masses below approximately $300$ GeV are
excluded at the $95\%$ CL, showing an improvement over the same analysis using
the 2015 dataset alone.


\item Chapter~\ref{c:eid} describes the measurements of the identification
efficiencies for electrons found within $\Delta R = 0.4$ of high-$p_T$ jets.
The measurements are motivated by a considerable increase in signal acceptance
seen in some SUSY searches (the SUSY search described in Chapter~\ref{c:susys}
is a particular example) when electrons overlapping with jets are selected, as
well as by the fact that prior to the measurements only electrons
non-overlapping with jets had been calibrated. The data used corresponds to an
integrated luminosity of 36.1 $\text{fb}^{-1}$, collected at center-of-mass
$\sqrt{s} = 13$ TeV. The measurements use a dilepton ($e\mu$) $t\tbar$ sample
enriched in boosted top quarks. The results present the integrated efficiencies
and the efficiencies as a function of the $p_T$ of the electrons, $\abs{\eta}$
of the electrons, $\Delta R$ between the electrons and the closest jets, and of
the $p_T$ of the closest jets.

\end{itemize}

