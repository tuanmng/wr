\chapter{SEARCH FOR SUPERSYMMETRY IN EVENTS WITH MISSING TRANSVERSE MOMENTUM AND
  MULTIPLE $B$-JETS}\label{c:susys}

Supersymmetry
(SUSY)~\cite{Golfand:1971iw,Volkov:1973ix,Wess:1974tw,Wess:1974jb,Ferrara:1974pu,Salam:1974ig}
is an extension of spacetime symmetry (Section~\ref{s:smsusy}). Not only does
SUSY unify fermions and bosons, it also simultaneously solves a number of
problems, including the hierachy problem and the nature of dark matter. It is a
leading candidate for beyond-the-Standard-Model physics.

Searching for SUSY is one of the main activities at ATLAS. This chapter
discusses a SUSY search that involves gluino pair-production, where the final
state includes missing transverse energy and multiple jets, of which at least
three must be $b$-jets. The search is highly motivated as the gluinos are
expected by naturalness~\cite{Barbieri:1987fn} to have a mass around the TeV
scale --- the LHC energy, and moreover the production cross section is high at
the LHC. Section~\ref{mbmodel} gives an introduction to the gluino
pair-production model. Section~\ref{mbdatasm} discusses the data and simulation
samples that are used in the analysis. Section~\ref{mbobjs} discusses the
physics objects involved. Section~\ref{mbselection} discusses event selection.
Section~\ref{mbbganal} discusses analysis strategies and the results of the
search. Section~\ref{mbinter} discusses the interpretation of the search.
Finally, Section~\ref{mb:conc} presents some conclusions.


The data was collected in the period 2015--2016, at $13$ TeV centre-of-mass
energy and corresponds to an integrated luminosity of $36.1~\text{fb}^{-1}$.


\section{Gluino Pair-Production}\label{mbmodel}

The gluino pair-production models in this analysis, Gbb and Gtt, belong to the
class of simplified models \cite{Alwall:2008ag,Alves:2011wf}, which are used to
optimize search event selections as well as to interpret search results. In
terms of signature, they always contain at least four $b$-jets that originate
either from gluino or top quark decays, and two neutralinos.

In the models, gluinos, which are superpartners of the Standard Model gluons,
are hypothesized to be produced in pairs (Figure~\ref{f:gttfey}). Each of the
gluino $\glu$ in the pair $\glu\glu$ is assumed to decay into a $\sto\bar{b}$
pair (in Gbb) or a $\sto\tbar$ pair (in Gtt) at $100\%$ branching
ratio\footnote{We could also consider the possibility that one gluino in the
	pair will decay to $\sto\bar{b}$ and the other will decay to $\sto\tbar$. This
	possibility is however not discussed in the present chapter.}. In both models
the supersymmetric $\sto$ is assumed to be off-shell, and as a result the
parameters of the models consist of only two parameters, the mass of $\glu$ and
the mass of $\sneu$, simplifying the search as well as the interpretations of
the results. The $\sneu$'s are assumed to be the lightest supersymmetric
particles; they are stable and serve as candidates for dark matter.

\begin{figure}[H]
	\includegraphics[width=6cm]{figures/ch6fig_01a}
	\includegraphics[width=6cm]{figures/ch6fig_01b}
	\centering

	\caption{The Gbb and Gtt models. Both belong to the class of simplified SUSY
		models. In both models, the supersymmetric $\sto$ is assumed to be off-shell.
		The parameters of the models are the mass of $\glu$ and the mass of $\sneu$.}

	\label{f:gttfey}
\end{figure}

In the present chapter, only the Gtt model will be discussed, as it is
connected directly with the contribution of the work of the chapter. In the
model, due to the top decay $t\to Wb$, with subsequent decays of the four
$W$'s, signal regions with higher jet multiplicity than those in Gbb are
expected.


Since a top quark decays almost all of
the time into a $W$ and a $b$ jet,  we expect the final state to include

\begin{itemize}[label=\ding{111}]

	\item Four $b$-jets from the decays of four top quarks

	\item As many as twelve jets if the $W$'s decay purely hadronically, or
	      otherwise additional jets and leptons (electrons or muons, and the number of
	      which depends on how many $W$'s would decay leptonically) along with missing
	      transverse energy from leptonic $W$ decays and the $\sneu$'s, assuming the
	      latter do not participate in known interactions and manifest as missing
	      transverse energy.

\end{itemize}

In this thesis we are only concerned with the leptonic channel, even though the
hadronic channel is part of the analysis as well, because the work of this
chapter is directly linked to the leptonic channel. The leptonic final state
consists of one or more leptons, large missing transverse energy, and multiple
jets in which at least three must be identified as $b$-jets.

%%%%%%%%%%%%%%%%%%%%%%%%%%%%%%%%%%%%%%%%%%%%%%%%%%%%%%%%%%%%%%%%%%%%%%%
\section{Data and Simulated Event Samples}\label{mbdatasm}

The data used in the analysis was collected by the ATLAS
detector~\cite{PERF-2007-01} in the period 2015--2016, delivered by the LHC at
$13$ TeV centre-of-mas energy and $25$ ns bunch spacing. The full data
corresponds to an integrated luminosity of $36.1~\text{fb}^{-1}$, and the
uncertainty on the luminosity is $2.1\%$~\cite{DAPR-2013-01}. An HLT
$E_{\text{T}}^{\text{miss}}$ trigger is applied on the data at three online
thresholds, $70$ GeV for 2015, $100$ GeV for early 2016, and $110$ GeV for late
2016. The trigger is fully efficient for the events that pass the preselection
requirement defined in Section~\ref{mb:pres}, which imposes the offline
condition $E_{\text{T}}^{\text{miss}} > 200$ GeV.

Most signal and background processes are generated using simulations, described
below. An exception is the multi-jet process which is estimated from data. All
simulated event samples were passed through the full ATLAS detector simulation
using Geant4~\cite{Agostinelli:2002hh}.

\paragraph{SUSY signal samples} The signal processes, where gluino pairs
$\tilde{g}\tilde{g}$ are produced and each member in the pair decays according
to $\tilde{g}\to \tbar\tilde{t}\tilde{\chi}^0_1$, are generated up to two
additional partons using \MGMCatNLO~\cite{Alwall:2014hca} at leading order, the
parton distribution function (PDF) set being NNPDF 2.3~\cite{Ball:2012cx}. The
samples are interfaced to \PYTHIA v8.186~\cite{Sjostrand:2007gs} for parton
showering, hadronization, and underlying events. All signal samples are
normalized with NLO cross-section calculations.


\paragraph{Standard Model background samples} The dominant background is
$t\tbar$ plus high $p_T$ jets. It is generated with the
\POWHEGBOX~\cite{Alioli:2010xd}~v2 event generator using the
CT10~\cite{Lai:2010vv} PDF set, interfaced with \PYTHIA v6.428. Events in which
all tops decay hadronically are excluded because of insufficient
$E_{\text{T}}^{\text{miss}}$ to constitute a significant background.

The \POWHEGBOX~v2 event generator is also used for single top quark in the
$Wt$- and $s$- channels, along with the CT10 PDF set and \PYTHIA v6.428.
\POWHEGBOX~v1 is used for the $t$- channel process. All events with at least
one $W$ that decays leptonically are included, and events in which all tops
decay hadronically are excluded because of insufficient
$E_{\text{T}}^{\text{miss}}$.

Smaller background contributions include $t\tbar$ plus $W/Z/h$ possibly along
with jets, and $t\tbar t\tbar$, $W/Z$+jets, and $WW/WZ/ZZ$ events. Their
generations are as follows.

\begin{itemize}[label=\ding{111}]

	\item $t\tbar$ plus $W/Z$: \MGMCatNLO~v2.2.2 and \PYTHIA v8.186. The PDF set is
	      NNPDF 2.3.

	\item $t\tbar h$: \MGMCatNLO v2.2.1 and \MYHERWIG++~\cite{Bahr:2008pv} v2.7.1.
	      The PDF set is CT10.

	\item $t\tbar t\tbar$: \MGMCatNLO v2.2.2 and \PYTHIA v8.186.

	\item $W/Z$+jets: \SHERPA v2.2.0~\cite{Gleisberg:2008ta} and the NNPDF 3.0 PDF
	      set.

	\item $WW/WZ/ZZ$: \SHERPA v2.1.1 and the CT10 PDT set.
\end{itemize}

Other potential sources of backgrounds, such as three top quark or three gauge
boson processes, are negligible.

\section{Physics Objects}\label{mbobjs}

In this section, physics objects that are used to select events for the
analysis are described. Electrons, muons, and jets are required to undergo an
overlap removal procedure to remove double-counting, which will also be
described.

\paragraph{Interaction vertices} Each interaction vertex in the event is
required to be associated with at least two tracks, each having $p_T > 0.4$
GeV. The primary vertex is defined to be the vertex having the largest sum of
squares of transverse momenta of the associated
tracks~\cite{ATL-PHYS-PUB-2015-026}.

\paragraph{Jets} Candidate jets are reconstructed using the anti-$k_t$ jet
algorithm~\cite{PERF-2014-07, Cacciari:2008gp, Cacciari:2011ma} with a radius
parameter $\Delta R = 0.4$; these jets will be referred to as small $R$-jets.
They are required to have $p_T > 20$ GeV and $\abs{\eta} < 2.8$, and undergo an
overlap removal procedure with electrons and muons, described below, after
which they are required to pass the requirement $p_T > 30$ GeV.

An event is rejected if it contains jets that arise from non-collision sources
or detector noise or pile-up interactions~\cite{ATLAS-CONF-2015-029}.

\paragraph{$b$-jets} A $b$-jet travels a short distance from the primary vertex
before decaying, thereby creating a secondary vertex from which additional
tracks originate (Figure~\ref{f:btagging}~\cite{bjetwiki}). The $b$-jets in the
event are identified by a multivariate algorithm, which relies on three pieces
of information, including the impact parameters of the tracks that belong to
the jets, the secondary vertices that are present in the event, and the flight
paths of heavy hadrons inside the
jets~\cite{PERF-2012-04,ATL-PHYS-PUB-2016-012}. In this analysis the
$b$-tagging working point that correponds to a $77\%$ efficiency for $b$-jets
with $p_T >20$ GeV is chosen.

\begin{figure}[H]
	\includegraphics[width=7cm]{figures/B-tagging_diagram}
	\centering

	\caption{$b$-jet secondary vertex which is displaced with respect to the
		primary vertex. In addition to tracks that originate from the primary vertex
		there are tracks that originate from the secondary vertex as
		well~\cite{bjetwiki}.}

	\label{f:btagging}
\end{figure}


\paragraph{Large $R$-jets} These jets refer to jets that are built up from the
small $R$-jet candidates~\cite{Nachman:2014kla} that have undergone overlap
removals with electrons and muons, using the anti-$k_t$ algorithm with $R=0.8$.
These large $R$-jets are required to have $p_T > 100$ GeV and $\abs{\eta} <
	2.0$. They are a tool to identify boosted top quarks, as a boosted top quark
that decays hadronically will produce jets that stay collimated to each other
(Figure~\ref{f:boostedtop}).


\begin{figure}[H]
	\includegraphics[width=10cm]{figures/btop}
	\centering

	\caption{Boosted top quark decay (right) compared to low-$p_T$ top quark
		decay. In the former case, the decay products stay collimated.}

	\label{f:boostedtop}
\end{figure}


\paragraph{Leptons} The leptons in the analysis include electrons and muons,
which are initially called lepton candidates if they pass certain preliminary
requirements discussed in the following. Below, leptons will mean lepton
candidates, unless stated otherwise. Each electron candidate must pass the
Loose quality criteria~\cite{PERF-2013-05,ATLAS-CONF-2016-024} and have
$\abs{\eta} < 2.47$. On othe other hand, each muon candidate must pass the
Medium quality criteria~\cite{PERF-2015-10} and have $\abs{\eta} < 2.5$.

The leptons, if they pass overlap removal, are required to pass an isolation
requirement, in order that fake and non-prompt leptons from jets may be
removed. This isolation requirement uses a $p_T$-dependent cone with radius
$\min(0.2, 10~\text{GeV} / p_T^{\text{lep}})$, where $p_T^{\text{lep}}$ is the
$p_T$ of the lepton, to take into account the fact that the angular seperation
between a lepton in the event and the $b$-jet narrows as $p_T$ of the top quark
increases (Figure~\ref{f:boostedtop}).

Then, electrons are required to pass the tight quality
criteria~\cite{PERF-2013-05,ATLAS-CONF-2016-024}. On the other hand, the lepton
is matched to the primary vertex by requiring the ratio $\abs{d_0}
	/\sigma_{d_0}$, where $d_0$ is the transverse impact parameter of the
associated ID track and $\sigma_{d_0}$ is the measured uncertainty of $d_0$, to
be $<5$ in the case of electrons and $< 3$ in the case of muons. Each lepton is
also required to have its longitudinal impact parameter $z_0$ satisfy
$\abs{z_0\sin\theta} < 0.5$mm. In addition, events that contain muon candidates
with $d_0 > 0.2$ mm or $z_0 > 1$ mm are rejected to suppress cosmic muons.

Lepton candidates passing the above requirements are referred to as signal
leptons.

\paragraph{Missing Transverse Energy} This is defined as the magnitude of the
negative vector sum of the transverse momenta of all calibrated objects in the
event, with an extra term~\cite{ATL-PHYS-PUB-2015-027,ATL-PHYS-PUB-2015-023} to
account for energy deposits not associated with any of the selected objects.


\paragraph{Overlap Removal} Electrons, muons, and jets are required to undergo
the following sequential overlap removal procedure.

\begin{itemize}[label=\ding{111}]

	\item Electrons that overlap with muons within a distance $\Delta R < 0.01$ are
	      removed, to suppress contributions from muon bremsstrahlung.

	\item Overlap removal between electrons and jets are performed, to suppress the
	      counting of electrons as jets and to remove fake electrons from hadron decays.
	      Here, to begin, non $b$-jets\footnote{In a standard overlap removal procedure,
		      jets within $\Delta R= 0.2$ of electrons are removed because electrons are
		      reconstructed as jets also, but since $b$-jets are important objects in the
		      analysis only non $b$-jets are removed.} within $\Delta R < 0.2$ of electrons
	      are removed. Then, electrons with $E_T < 50$ GeV that are found within $\Delta
		      R = 0.4$ of jets are removed, but the jets are kept. Electrons with higher
	      $E_T$ are likely to be in boosted top quark decays, in which they will be found
	      closer to the jets the higher are their $E_T$. Accordingly, a distance that
	      takes into account this fact is used, $\Delta R = \min(0.4, 0.04+10~\text{GeV})
		      / E_T$\footnote{This distance makes sure that as soon as an electron is found
		      within $\Delta R = 0.4$ of a jet, it will be treated as a potential signal
		      electron, i.e. it will be selected if its $E_T$ justifies its being found close
		      to the jet, but not closer to what its $E_T$ warrents.}, to increase the
	      acceptance of these electrons.

	\item The remaining muons and jets are subject to a removal scheme as follows.
	      If a non $b$-jet having fewer than three inner detector tracks is found within
	      a distance $\Delta R = 0.2$ of a muon, the jet is likely to come from
	      high-$p_T$ muon bremsstrahlung and thus is removed. Then muons with $p_T < 50$
	      GeV found within $\Delta R = 0.4$ of jets are removed, to suppress non-prompt
	      muons originating from jets. Muons having $p_T > 50$ GeV are, as in the case of
	      high $E_T$ electrons, subject to the boosted distance $\Delta R = \min(0.4,
		      0.04+10~\text{GeV}) / p_T$.

\end{itemize}


\paragraph{Boosted Overlap Removal Studies} The use of electrons inside jets
had been initiated by the $t\tbar$ resonance search~\cite{ttbarres} to increase
signal acceptance in scenarios that involve the decaying of
beyond-Standard-Model particles into Standard-Model top quarks. Following the
idea, the boosted distance $\Delta R = \min(0.4, 0.04+10~\text{GeV}) / p_T$ was
introduced into the current analysis for the muons before being subsequently
adopted for the electrons in a later version of the analysis. The studies for
muons was carried out on three samples, a $t\tbar$ sample, a Gtt sample where
the mass of the gluino is $1300$ GeV and that of the neutralino is $900$ GeV,
and another Gtt sample where the mass of the gluino is $1600$ GeV and that of
the neutralino is $100$ GeV. The last sample, due to large mass difference
between the gluino and the neutralino, will be called a boosted sample, since
it is expected to be a source of boosted top quarks. Figure~\ref{f:mindrs},
which plots the distance $\Delta R$ between the truth level muons and the
closest jets, shows that indeed a large fraction of potential signal muons are
found below $\Delta R = 0.4$ in the boosted sample. The other signal sample
shows a smaller but still considerable fraction of potential signal muons below
$\Delta R = 0.4$, most likely due to random overlap between the muons and the
top quarks (there are four top quarks as compared to two in the $t\tbar$
sample, the latter displays instead a mild peak in the region $\Delta R >
	0.4$).

\begin{figure}[H]
	\includegraphics[width=12cm]{figures/mindrs}
	\centering

	\caption{$\Delta R$ between the truth-level muons and the closest jets. The
		boosted sample where mass of the gluino is $1600$ GeV and that of the
		neutralino is $100$ GeV shows a high peak at low $\Delta R$. The other signal
		sample also displays but not as high. $t\tbar$ sample (shown as
		\texttt{t\_tbar} in the figure), exhibits a mild peak around $\Delta R =
			1.0$.}

	\label{f:mindrs}
\end{figure}

Subsequent studies on the possibility of applying the $p_T$-dependent overlap
removal distance on the muons was performed where the parameters of the formula
are varied. Figure~\ref{f:sa} show the significances that resulted from
different choices of the paramater of overlap removal distance. The
significance in which a fixed $\Delta R = 0.4$ is used is $0.87$, whereas the
significances achieved with $p_T$-dependent overlap removal could reach as high
as $1.37$, representing a possible gain factor of $1.5$. The studies was
performed on data that corresponds to ({\color{pink} ?? luminosity}).

\begin{figure}[H]
	\includegraphics[width=12cm]{figures/sA}
	\centering

	\caption{$p_T$-dependent overlap removal studies for the muons. Each set of
		parameters in the formula is called a combination. The maximal gain
		achievable a is $1.5$.}

	\label{f:sa}
\end{figure}



\section{Event Selection}\label{mbselection}

This section discusses the discriminating variables (Section~\ref{mb:disv}) and
the preselection criteria (Section~\ref{mb:pres}). In the latter section we
also discuss the modelling of the data. Finally, Section~\ref{mb:opt} discusses
the optimization of some important variables.

\subsection{Discriminating Variables}\label{mb:disv}

The following list of variables is found to be discriminating between signal
and Standard Model backgrounds:

\begin{itemize}[label=\ding{109}]

	\item The effective mass $m_{\text{eff}}$, defined as the sum of missing
	      transverse energy plus the transverse momenta of jets and leptons in the event:

	      $$m_{\text{eff}} = \sum_i p_T^{\text{jet}_i} + \sum_j p_T^{l_j} + E_{\text{T}}^{\text{miss}}$$

	      $m_{\text{eff}}$ is typically much higher for signal events than for background events.


	\item The transverse mass $m_{\text{T}}$, defined by

	      $$m_{\text{T}} = \bigg(2p_T^l E_{\text{T}}^{\text{miss}} \big(1 - \cos(\Delta\phi) \big) \bigg)^{1/2}$$

	      where $\Delta \phi$ is the angle between missing transverse momentum of the
	      event and the transverse momentum of the leading lepton.

	      In background events, such as in semileptonic $t\tbar$ and $W$+jets, where
	      there is one $W$ that decays leptonically, this variable reaches a maximum at
	      the $W$ mass. It is expected to be higher for signal events where there is more
	      missing transverse energy due to the neutralinos.


	\item The transverse mass $m_{\text{T,min}}^{\text{b-jets}}$ defined by

	      $$m_{\text{T,min}}^{\text{b-jets}} = \text{min}_{i\leq 3} \bigg(2
		      p_T^{\text{b-jets}_i} E_{\text{T}}^{\text{miss}} \big( 1 -
		      \cos(\Delta\phi)\big) \bigg)^{1/2}$$

	      where $\Delta \phi$ is the angle between the missing transverse momentum and
	      the $i$-th b-jet.

	      In background $t\tbar$ events where a top quark decays leptonically, this
	      variable reaches a maximum at the top quark mass. It is expected to be higher
	      for signal events as there is more missing transverse energy due to the
	      neutralinos.

	\item The total jet mass $M_J^{\sum}$, defined by

	      $$M_J^{\sum} = \sum_{i\leq 4}m_{J, i} $$

	      where $m_{J, i}$ is the mass of the $i$-th large-radius re-clustered jet in the
	      event. It is higher for signal events, because there are as many as four
	      hadronically decaying top quarks, whereas the background is dominated by
	      $t\tbar$ events where one or both of the tops decay leptonically.

\end{itemize}


\subsection{Preselection and Modelling of the Data}\label{mb:pres}

\paragraph{Preselection} The preselection requirements include
$E_{\text{T}}^{\text{miss}} > 200$ GeV, in addition to the
$E_{\text{T}}^{\text{miss}}$ trigger requirement, and at least four jets of
which at least two must be identified as $b$-jets.

\paragraph{Modelling of the Data} In the preselection sample, correction
factors need to be extracted to account for shape discrepancies between data
and the expected background for $m_{\text{eff}}$. Thus, background-dominated
regions are defined by requiring exactly two $b$-jets and
$m_{\text{T,min}}^{\text{b-jets}} < 140$ GeV, in which the correction factors
are taken to be the ratio of the number of observed events to the predicted
number of background events in a given $m_{\text{eff}}$ bin. The correction
factors range from $0.7$ to $1.1$; they are also taken as an uncertainty for
both background and signal events.

Figure~\ref{f:fig_04abcdef} show a number of variables after preselection,
including the number of jets, the number of $b$-jets,
$E_{\text{T}}^{\text{miss}}$, $m_{\text{eff}}$, $M_J^{\sum}$, and
$m_{\text{T}}$. The uncertainty shown include the statistical and experimental
systematic uncertainties (Section~\ref{mb:sysun}), but exclude the theoretical
uncertainties in the background modelling.

\begin{center}

	\begin{figure}[H]
		\includegraphics[width=7cm]{figures/fig_04a}
		\includegraphics[width=7cm]{figures/fig_04b}

		\includegraphics[width=7cm]{figures/fig_04c}
		\includegraphics[width=7cm]{figures/fig_04d}


		\includegraphics[width=7cm]{figures/fig_04e}
		\includegraphics[width=7cm]{figures/fig_04f}

		\caption{The distributions of the number of jets, the number of $b$-jets,
		$E_{\text{T}}^{\text{miss}}$, $m_{\text{eff}}$, $M_J^{\sum}$, and
		$m_{\text{T}}$ after the preselection requirements. The uncertainty
		includes both statistical and experimental systematatic uncertainties
		(defined in Section~\ref{mb:sysun}). The last bin includes overflow events.
		The ratio of data to background prediction is also shown below each
		figure.}

		\label{f:fig_04abcdef}

	\end{figure}

\end{center}

\subsection{Optimization of Discriminating Variables}\label{mb:opt}

An optimization studies was performed to optimize the selections for the
leptonic channel of the analysis ({\color{pink} at ?? luminosity}). The
hadronic channel was optimized separately and will not be discussed in the
current section.

The number of jets (Section~\ref{mbobjs}), $N_{\text{jet}}$, missing transverse
energy, $E_{\text{T}}^{\text{miss}}$, and $m_{\text{eff}}$ are important
variables that help with separating signal events and background events. In the
current analysis, a studies was performed to decide on the optimal values of
these variables (and other potentially discriminating variables also). To this
end, four samples were selected; they may be put into three groups:

\begin{itemize}

	\item One sample where the mass of the gluino is $1900$ GeV and that of the
	      neutralino is $200$ GeV. This will be referred to as the boosted sample, since
	      it is expected to be a source of boosted top quarks.

	\item One sample where the mass of the gluino is $1900$ GeV and that of the
	      neutralino is $1000$ GeV; this is a less boosted sample.

	\item Two samples where there are small mass differences between the gluino and
	      the neutralino. One having $1200$ GeV and $800$ GeV, one having $1500$ GeV and
	      $1000$ GeV, which will be called compressed samples one and two respectively.

\end{itemize}

The optimization proceeds with different sets of values of potentially
discriminating variables, among them including sets in which

\begin{itemize}

	\item $N_{\text{jet}} \geq 6$, or $N_{\text{jet}} \geq 7$, or $N_{\text{jet}}
		      \geq 8$, or $N_{\text{jet}} \geq 9$, or $N_{\text{jet}} \geq 10$;

	\item $E_{\text{T}}^{\text{miss}} > 200$ GeV, or $E_{\text{T}}^{\text{miss}} >
		      300$ GeV, or $E_{\text{T}}^{\text{miss}} > 400$ GeV, or
	      $E_{\text{T}}^{\text{miss}} > 500$ GeV, or $E_{\text{T}}^{\text{miss}} > 600$
	      GeV;

	\item $m_{\text{eff}}$ is allowed to varied from $500$ GeV to $3500$ GeV, in
	      steps of $200$ GeV.

\end{itemize}

All possible combinations of the values of the variables are tested. The
optimization finds that higher jets improve the search significance, especially
for samples where there are small mass differences between the gluino and the
neutralino (the compressed samples). This is consistent with the fact that a
small mass difference makes kinematic properties of the signal quite
indistinguisable from those of the background, and consequently we have to push
to the higher $N_{\text{jets}}$ regime. Figure~\ref{f:mbnjetsop} shows the
significance as the number of jets is allowed to increase while all other
variables are kept fixed, for the compressed sample number two. The red point
indicates the best significance, and the numbers that show up below the points
are the signal, the signal uncertainty, the background, and the background
uncertainty, in that order. Whenever the calculations of the significance is no
longer meaningful, such as when the number of unweighted events in the CR
region is too small, the significance is set to $-1$.

\begin{figure}[H]
	\includegraphics[width=10cm]{figures/leptons_njets_d370142.pdf}
	\centering

	\caption{The significance with respect to $N_{\text{jets}}$ for the
		compressed sample number two, where the masses are $1500$ GeV and $1000$ GeV.
		The red point indicates the best significance, and the numbers that show up
		below the points are the signal, the signal uncertainty, the background, and
		the background uncertainty, in that order.}

	\label{f:mbnjetsop}
\end{figure}

The optimization also suggests that higher $E_{\text{T}}^{\text{miss}}$ would
help when more data becomes available in future LHC run.
Figure~\ref{f:mbmetsop} shows the significance as $E_{\text{T}}^{\text{miss}}$
is allowed to increase while all other variables are kept fixed, for the
boosted sample where the masses are $1900$ GeV and $200$ GeV.

\begin{figure}[H]
	\includegraphics[width=10cm]{figures/leptons_met_d370170.pdf}
	\centering

	\caption{The significance with respect to $E_{\text{T}}^{\text{miss}}$ for
		the boosed sample with masses $1900$ GeV and $200$ GeV. The red point
		indicates the best significance, and the numbers that show up below the
		points are the signal, the signal uncertainty, the background, and the
		background uncertainty, in that order.}

	\label{f:mbmetsop}
\end{figure}

Finally, the optimization also suggests higher $m_{\text{eff}}$ would help when
more data becomes available in future LHC run. Figure~\ref{f:mbmeffsop} shows
the significance as $m_{\text{eff}}$ is allowed to increase while all other
variables are kept fixed, for the boosted sample.

The results of the optimization were used to inform the optimal criteria for
the signal regions defined in Section~\ref{s:mbstra}.

\begin{figure}[H]
	\includegraphics[width=10cm]{figures/leptons_meff_d370170.pdf}
	\centering

	\caption{The significance with respect to $m_{\text{eff}}$ for the boosted
		sample, where the masses are $1900$ GeV and $200$ GeV. The red point
		indicates the best significance, and the numbers that show up below the
		points are the signal, the signal uncertainty, the background, and the
		background uncertainty, in that order.}

	\label{f:mbmeffsop}
\end{figure}


\section{Analysis and Results}\label{mbbganal}

A significant part of the analysis is the work of background estimation; it is
discussed in Section~\ref{s:mbbg}. There are two search strategies, discussed
in Section~\ref{s:mbstra}. Section~\ref{mb:sysun} discusses the evaluation of
systematic uncertainties, and Section~\ref{mbresul} discusses the results of
the search.

\subsection{Background Estimation}\label{s:mbbg}

Each signal region that is defined is contaminated with Standard Model
backgrounds, the dominant source of which is $t\tbar$ plus jets, which is
estimated using a normalization factor. To this end, in addition to each signal
region (SR), a control region (CR), orthogonal to the signal region but is
otherwise comparable with it in terms of background composition and kinematics,
is defined. Signal contamination in the control region is suppressed by
inverting or relaxing some kinematic variables. The normalization factor is
then verified in validation regions (VRs), designed to be similar to the signal
region in terms of background composition.

The remaining backgrounds are made up of single-top, $W$+jets, $Z$+jets,
$t\tbar$+$W/Z/h$, $t\tbar t\tbar$ and diboson events. They are estimated from
simulations, which are normalized to the best available theoretical cross
sections. The multijet background was found to be negligible, but is still
estimated using a procedure described in \cite{Aad:2012fqa}.

\subsection{Analysis Strategy}\label{s:mbstra}

Physics objects in the final state may fall into different kinematic ranges.
These objects consist of those coming from Standard Model events as well as
those coming from the hypothetical SUSY events. Naturally, not all kinematic
ranges will be equal in terms of the relative distributions of the two kinds of
events. The signal and background samples allow us to optimize, i.e. to arrive
at one or more sets of kinematic ranges where we will have the best chances to
assess whether or not an excess of events, relative to Standard Model
distribution, is seen. This assessment can be based solely from judging the
excess of events and accordingly can be said to be model-independent. Since
there is no reason that these sets of kinematic ranges should not overlap with
each other, we may in fact design overlapping kinematic regions, to be able to
focus best on the question if an excess of events is seen at all, in which case
we may claim to have seen a signal, or to rule out the existence of any
beyond-the-Standard-Model signal, if the number of events seen in fact does not
deviate in any statistically significant way from the Standard Model
prediction. This strategy is called the cut-and-count strategy.

On the other hand, if we optimize under the constraint that the signal regions
must be non-overlapping with each other, then this constraint could mean that
the resulting signal regions might not be as performant as those in the
cut-and-count method, in terms of probing the existence of a signal.
Nevertheless, a combination of these non-overlapping signal regions would take
into account many non-overlapping kinematic ranges at the same time, and
accordingly would allow us to better assess if the specific model we are using
should be ruled out. This strategy is called the multi-bin analysis.

Both strategies are followed in the current analysis for the hadronic channel
as well as for the leptonic channel. However, only the leptonic channel will be
discussed in the following.

\paragraph{Cut-and-Count Analysis} In this analysis strategy the signal points
are grouped into three classes. Thus there are three signal regions together
with their corresponding control and validation regions. The common selections
include $\geq 1$ lepton, $p_T^{\text{jet}} > 30$ GeV, and $N_{b\text{-jets}}
	\geq 3$. The choice of values of other discriminating variables make the
classes different from one another. The definitions of the regions are shown in
Table~\ref{tab:GttEvsel}.

\begin{itemize}[label=\ding{111}]

	\item Region B, where B stands for boosted, is optimized for signals having a
	      large mass difference between the gluino and the neutralino ($\geq 1.5$ TeV).

	\item Region C, where C stands for compressed, is optimized for signals where
	      the mass difference is small ($\leq 300$ GeV).

	\item Region M, where M stands for moderate, is the region where the mass
	      difference is in between those of boosted and compressed regions.

\end{itemize}

As is shown in the table, the selections on $m_{\text{eff}}$,
$E_{\text{T}}^{\text{miss}}$, and $M_{\text{J}}^{\sum}$ are lower in B than in
C, to improve signal acceptance in the latter. The increase in background as a
result of the lower cuts is managed by making tighter selections on the number
of jets, the number of $b$-jets, or $m_{\text{T, min}}^{b\text{-jets}}$.

The CRs are defined in the low $m_{\text{T}}$ region to remove overlaps with
the SRs. The $m_{\text{T, min}}^{b\text{-jets}}$ cut is removed, and cuts on
other variables are lowered to make sure that each CR would have $\geq 10$
events, in order that the normalization of the $t\tbar$ background would be
determined with sufficient statistical accuracy. On the other hand, there are
two types of VRs, VR-$m_{\text{T}}$ to validate background prediction in high
$m_{\text{T}}$ region, and VR-$m_{\text{T, min}}^{b\text{-jets}}$ in the high
$m_{\text{T, min}}^{b\text{-jets}}$ region. These VRs are ensured to be
kinematically close to the SRs and the CRs by the cut on $N_{\text{jets}}$,
which at the same time ensures their non-overlapping. Cuts on other variables
are also used to keep the VRs non-overlapping with their corresponding SRs,
specifically $M_{\text{J}}^{\sum}$ or $m_{\text{T, min}}^{b\text{-jets}}$ in
VR-$m_{\text{T}}$ and $m_T$ in VR-$m_{\text{T, min}}^{b\text{-jets}}$.

\begin{table}[H]
	\centering
	\renewcommand{\arraystretch}{1.5}
	\begin{tabular}{c c c c c c c c}
		\toprule
		\multicolumn{8}{c}{\textbf{ Gtt 1-lepton}}\\
		\multicolumn{8}{c}{Criteria common to all regions: $\geq 1$ signal lepton , $p_T^{\text{jet}} >  30$ GeV, $N_{b\text{-jets}} \geq 3$} \\ \midrule
		Targeted kinematics & Type                                   & $N_{\text{jet}}$ & $m_{\text{T}}$ & $m_{\text{T, min}}^{b\text{-jets}}$ & $E_{\text{T}}^{\text{miss}}$ & $m_{\text{eff}}^{\text{incl}}$ & $M_{\text{J}^{\sum}}$ \\ \midrule
		\multirow{4}{*}{\begin{minipage}{3cm}\centering Region B\\
				(Boosted, Large $\Delta m$) \end{minipage}}
		                    & SR                                     & $\geq 5$         & $> 150$        & $> 120 $                            & $> 500 $                     & $> 2200 $                      & $> 200$               \\
		                    & CR                                     & $= 5$            & $< 150$        & $-$                                 & $> 300 $                     & $> 1700 $                      & $> 150$               \\
		                    & VR-$m_T$                               & $\geq 5$         & $> 150$        & $-$                                 & $> 300 $                     & $> 1600 $                      & $< 200$               \\
		                    & VR-$m_{\text{T, min}}^{b\text{-jets}}$ & $> 5$            & $< 150$        & $> 120 $                            & $> 400 $                     & $> 1400 $                      & $> 200$               \\\midrule
		\multirow{4}{*}{\begin{minipage}{3cm}\centering Region M\\
				(Moderate $\Delta m$) \end{minipage}}
		                    & SR                                     & $\geq 6$         & $> 150$        & $> 160 $                            & $> 450 $                     & $> 1800 $                      & $> 200$               \\
		                    & CR                                     & $= 6$            & $< 150$        & $-$                                 & $> 400 $                     & $> 1500 $                      & $> 100$               \\
		                    & VR-$m_T$                               & $\geq 6$         & $> 200$        & $-$                                 & $> 250 $                     & $> 1200 $                      & $< 100$               \\
		                    & VR-$m_{\text{T, min}}^{b\text{-jets}}$ & $> 6$            & $< 150$        & $> 140 $                            & $> 350 $                     & $> 1200 $                      & $> 150$               \\\midrule
		\multirow{4}{*}{\begin{minipage}{3cm}\centering Region C\\
				(Compressed, small $\Delta m$) \end{minipage}}
		                    & SR                                     & $\geq 7$         & $> 150$        & $> 160 $                            & $> 350 $                     & $> 1000 $                      & $-$                   \\
		                    & CR                                     & $= 7$            & $< 150$        & $-$                                 & $> 350 $                     & $> 1000 $                      & $-$                   \\
		                    & VR-$m_T$                               & $\geq 7$         & $> 150$        & $< 160 $                            & $> 300 $                     & $> 1000 $                      & $-$                   \\
		                    & VR-$m_{\text{T, min}}^{b\text{-jets}}$ & $> 7$            & $< 150$        & $> 160 $                            & $> 300 $                     & $> 1000 $                      & $-$                   \\
	\end{tabular}
	\caption{Definitions of the Gtt SRs, CRs and VRs of the cut-and-count analysis.
		The jet $p_T$ requirement is also applied to
		$b$-tagged jets.}
	\label{tab:GttEvsel}
\end{table}

\paragraph{Multi-bin Analysis} In this analysis strategy a number of
non-overlapping regions are defined using $N_{\text{jet}}$ and
$m_{\text{eff}}$. The regions are shown schematically in
Figure~\ref{f:fig_05b}. In each region signal models having a specified range
of mass difference are used to optimize all remaining kinematic variables.

\begin{figure}[H]
	\includegraphics[width=10cm]{figures/fig_05b}
	\centering

	\caption{Schematic illustration of the regions in the multibin analysis. This
		is a two-dimensional illustration in the variables $N_{\text{jet}}$ and
		$m_{\text{eff}}$.}

	\label{f:fig_05b}
\end{figure}

The definitions of the regions are shown in Table~\ref{tab:multibin_Hn}, which
shows high-$N_{\text{jet}}$ SRs, CRs, and VRs, and Table~\ref{tab:multibin_In},
which shows intermediate-$N_{\text{jet}}$ SRs, CRs, and VRs. The low
$m_{\text{eff}}$ regions are designed for signals with low mass difference,
while the high $m_{\text{eff}}$ for boosted events. For each SR, the CR is
obtained by keeping most kinematic variables close while inverting the
$m_{\text{T}}$ cut, so that there would be no overlapping with the SR. The VRs
are obtained with cuts on $E_{\text{T}}^{\text{miss}}$ and $m_{\text{T,
				min}}^{b\text{-jets}}$.


\begin{landscape}
	\begin{table}[H]
		\centering
		\renewcommand{\arraystretch}{1.5}

		\begin{tabular}{c c c c c c c c c}
			\toprule
			\multicolumn{9}{c}{\textbf{ High-$N_{\text{jet}}$ regions}}\\
			\multicolumn{9}{c}{Criteria common to all regions: $N_{b\text{-jets}} \geq 3$, $p_T^{\text{jet}} > 30$~GeV } \\
			\midrule
			Targeted & Type  & $N_{\text{lepton}}$ & $m_{\text{T}}$ & $N_{\text{jet}}$ & $m_{\text{T, min}}^{b\text{-jets}}$        & $M_J^{\sum}$ & $E_{\text{T}}^{\text{miss}}$                          & $m_{\text{eff}}$ \\
			\midrule
			\multirow{3}{*}{\begin{minipage}{3cm}\centering High-$m_{\text{eff}}$\ \\ (HH) \\ (Large $\Delta m$) \end{minipage}}
			         & SR-1L & $\ge 1$             & $> 150 $       & $\ge 6$          & $> 120$                                    & $>200$       & $> 500 $                                              & $> 2300$         \\
			         & CR    & $\ge 1$             & $< 150$        & $\ge 6$          & $> 60 $                                    & $>150$       & $> 300 $                                              & $> 2100$         \\
			         & VR-1L & $\ge 1$             & $> 150$        & $\ge 6$          & $<140$ if $m_{\text{eff}}>2300$            & $-$          & $< 500$                                               & $> 2100$         \\
			\midrule
			\multirow{3}{*}{\begin{minipage}{3cm}\centering Intermediate-$m_{\text{eff}}$\ \\ (HI) \\ (Intermediate $\Delta m$) \end{minipage}}
			         & SR-1L & $\ge 1$             & $> 150 $       & $\ge 8$          & $> 140$                                    & $>150$       & $> 300 $                                              & $[1800, 2300]$   \\
			         & CR    & $\ge 1$             & $< 150$        & $\ge 8$          & $> 60$                                     & $>150$       & $> 200 $                                              & $[1700, 2100]$   \\
			         & VR-1L & $\ge 1$             & $> 150$        & $\ge 8$          & $<140$ if $E_{\text{T}}^{\text{miss}}>300$ & $-$          & $< 300 $ if $m_{\text{T, min}}^{b\text{-jets}} > 140$ & $[1600, 2100]$   \\
			\midrule
			\multirow{3}{*}{\begin{minipage}{3cm}\centering Low-$m_{\text{eff}}$\ \\ (HL) \\ (Small $\Delta m$) \end{minipage}}
			         & SR-1L & $\ge 1$             & $> 150 $       & $\ge 8$          & $> 140$                                    & $-$          & $> 300 $                                              & $[900, 1800]$    \\
			         & CR    & $\ge 1$             & $< 150$        & $\ge 8$          & $> 130$                                    & $-$          & $> 250 $                                              & $[900, 1700]$    \\
			         & VR-1L & $\ge 1$             & $> 150$        & $\ge 8$          & $<140$                                     & $-$          & $> 225 $                                              & $[900, 1650]$    \\
			\bottomrule
		\end{tabular}
		\caption{Definition of the high-$N_{\text{jet}}$ SRs, CRs and VRs of the multi-bin analysis.}
		\label{tab:multibin_Hn}
	\end{table}
\end{landscape}


\begin{landscape}
	\begin{table}[t]
		\small
		\centering
		\renewcommand{\arraystretch}{1.5}

		\begin{tabular}{c c c c c c c c c}
			\toprule
			\multicolumn{9}{c}{\textbf{ Intermediate-$N_{\text{jet}}$ regions}}\\
			\multicolumn{9}{c}{Criteria common to all regions: $N_{b\text{-jets}} \geq 3$, $p_T^{\text{jet}} > 30$~GeV } \\
			\midrule
			Targeted & Type  & $N_{\text{lepton}}$ & $m_{\text{T}}$ & $N_{\text{jet}}$ & $m_{\text{T, min}}^{b\text{-jets}}$ & $M_J^{\sum}$ & $E_{\text{T}}^{\text{miss}}$ & $m_{\text{eff}}$ \\
			\midrule
			\multirow{3}{*}{\begin{minipage}{3cm}\centering Intermediate-$m_{\text{eff}}$ \ \\ (II) \\ (Intermediate $\Delta m$) \end{minipage}}
			         & SR-1L & $\ge 1$             & $> 150 $       & $[6,7]$          & $> 140$                             & $>150$       & $> 300 $                     & $[1600,2300]$    \\
			         & CR    & $\ge 1$             & $< 150$        & $[6,7]$          & $> 110 $                            & $>150$       & $> 200 $                     & $[1600,2100]$    \\
			         & VR-1L & $\ge 1$             & $> 150$        & $[6,7]$          & $<140$                              & $-$          & $> 225 $                     & $[1450,2000]$    \\
			\midrule
			\multirow{3}{*}{\begin{minipage}{3cm}\centering Low-$m_{\text{eff}}$ \ \\ (IL) \\ (Low $\Delta m$) \end{minipage}}
			         & SR-1L & $\ge 1$             & $> 150 $       & $[6,7]$          & $> 140$                             & $-$          & $> 300 $                     & $[800,1600]$     \\
			         & CR    & $\ge 1$             & $< 150$        & $[6,7]$          & $> 130 $                            & $-$          & $> 300 $                     & $[800,1600]$     \\
			         & VR-1L & $\ge 1$             & $> 150$        & $[6,7]$          & $<140$                              & $-$          & $> 300 $                     & $[800,1450]$     \\
			\bottomrule
		\end{tabular}
		\caption{Definition of the intermediate-$N_{\text{jet}}$ SRs, CRs and VRs of the multi-bin analysis.}
		\label{tab:multibin_In}
	\end{table}
\end{landscape}


\subsection{Systematic Uncertainties}\label{mb:sysun}

The systematic uncertainties on background estimation come from the
extrapolation of the $t\tbar$ normalization from the CRs to the SRs, as well as
from MC estimations of the minor backgrounds. The total systematic
uncertainties vary from $20\%$ to $80\%$; they are shown in
Figure~\ref{f:fig_06}.

\begin{figure}[H]
	\includegraphics[width=10cm]{figures/fig_06a}
	\includegraphics[width=10cm]{figures/fig_06b}
	\centering
	\caption{Systematic uncertainties for the cut-and-count analysis (top) and multi-bin
		analysis (bottom).}
	\label{f:fig_06}
\end{figure}

Among the detector-related uncertainties, the largest contributions come from
jet energy scale (JES), jet energy resolution (JER), and the $b$-tagging
efficiencies and mistagging rates. The JES uncertainties are derived from
$\sqrt{s}=13$ TeV data and simulations, whereas JER from $8$ TeV data using
simulations~\cite{ATL-PHYS-PUB-2015-015}. These uncertainties are measured for
small $R$-jets and are propagated to re-clustered large $R$-jets. The jet mass
scale and resolution uncertainties make a negligible contribution to
re-clustered jet mass. JES uncertainties contribute between $4\%$ and $35\%$ to
background estimation, and JER uncertainties up to $26\%$.

The $b$-tagging and mistagging rate uncertainties contribute between $3\%$ to
$24\%$ to background estimation. Lepton reconstructions and energy measurement
make a negligible contribution.

As $t\tbar$ normalization takes place in the CRs, uncertainties due to $t\tbar$
simulation only make contributions to the extrapolation from the CRs to the SRs
and the VRs. The theoretical uncertainty on the $t\tbar$ background is sum in
quadrature of the following sources:

\begin{itemize}[label=\ding{111}]

	\item Hadronization and parton showering model uncertainties are estimated with
	      a \POWHEG~sample, showered by \MYHERWIG++ v2.7.1 with the UEEE5
	      underlying-event.

	\item Uncertainties due to the simulation of initial- and final-state radiation
	      are estimated using \POWHEG~samples, showered with \PYTHIA~v6.428. The
	      renormalization and factorization scales are set to twice and then half of
	      their nominal values, so that radiation in the events is increased and
	      decreased respectively. The uncertainty in each case is taken to be the
	      difference between the obtained value and the nominal value.

	\item The uncertainty due to the choice of event generator is estimated by
	      comparing background predictions in \MGMCatNLO~and \POWHEG~samples, both
	      showered with \MYHERWIG++ v2.7.1.

\end{itemize}

An additional uncertainty is assigned to $t\tbar$ heavy-flavour jets. It was
found from simulation studies that each set of SR, CR, and VR had the same
fractions of these events. Thus the uncertainties are similar among the
regions, and $t\tbar$ normalization based on the predictions in the CR largely
cancel out these uncertanties. The residual uncertainty is taken as the
difference between the $t\tbar$ nominal prediction and that obtained after
varying the cross-section of $t\tbar$ events with additional heavy-flavor jets
by $30\%$~\cite{TOPQ-2014-10}. It contributes up to $8\%$ to the total $t\tbar$
background uncertainty (background expectation ranges from $5\%$ to $76\%$ in
the regions). The statistical uncertainty of the CRs is included in the
systematic uncertainties and varies from $10\%$ to $30\%$.

The single-top simulation suffers from inteference between $t\tbar$ and $Wt$
processes. This uncertainty is estimated using $WWbb$ events, generated using
\MGMCatNLO, where a comparison is made with the sum of $t\tbar$ and $Wt$
processes. Also, uncertainties due to initial- and final-state radiation are
estimated using \PYTHIA~v6.428, as in the case of $t\tbar$ uncertainties.
Moreover, an additional $5\%$ uncertainty is included in the cross-section of
single-top processes~\cite{TOPQ-2014-10}. The total uncertainty for the
single-top process contributes to a change of the overall background of up to
$11\%$ in the regions.

Uncertainties in the $W/Z$+jets backgrounds are estimated by varying various
parameter scales, and make a contribution up to $50\%$ in the regions.

Finally, the uncertainties in the cross-sections of signal processes are
determined from an envelope of different cross-section predictions. A
systematic uncertainty is also assigned to the kinematic correction described
in Section~\ref{mb:pres}; the total size of the correction is used as an
uncertainty.

\subsection{Results}\label{mbresul}

In each SR, the expected SM background is determined with a profile likelihood
fit \cite{Cowan:2010js} implemented in the HistFitter framework \cite{HFpaper},
which will be referred to as a background-only fit. The fit uses as inputs the
number of events predicted by simulation in each region, plus the number of
events predicted in the associated CR. It is constrained by the number of
observed events in the CR and outputs a $t\tbar$ normalization factor that is
applied to the number of $t\tbar$ events predicted by simulation in the SR. The
number of observed and predicted events are modelled as Poisson distributions,
and the systematic uncertainties as Gaussian distributions having widths that
correspond to the sizes of the uncertainties, treated as correlated where
appropriate. The likelihood function is the product of the various
distributions.

Figure~\ref{f:fig_07ab} shows the values of the normalization factors resulting
from the fit, the expected numbers of background events and observed data in
all the CRs for the cut-and-count and multi-bin analyses.

Figure~\ref{f:fig_08ab} shows the results of the fit to the CRs, extrapolated
to the VRs for the cut-and-count and multi-bin analyses. The background
predicted by the fit is compared to the data in the upper panel. The figure
also shows in the lower panel the pull, which is the difference between the
observed number of events and the predicted background divided by the total
uncertainty.

Figure~\ref{f:fig_09ab} shows the SRs for the cut-and-count and multi-bin
analyses. The pull is shown in the lower panel. No significant excess relative
to the predicted background is seen. The background is dominated by $t\tbar$ in
all SRs.

Table~\ref{tab:yield_discovery} shows the observed number of events and
predicted number of background events from the background fit for the
cut-and-count analysis. In general, the central value of the fitted background
is larger than the MC-only prediction. This is in part due to an
underestimation of the cross-section of $t\tbar+ \geq 1b$ and $t\tbar+ \geq 1c$
processes in the simulation.

\begin{figure}[H]
	\includegraphics[width=10cm]{figures/fig_07a}
	\includegraphics[width=10cm]{figures/fig_07b}
	\centering

	\caption{Pre-fit events in CRs and the related $t\tbar$ normalization factors
		for the cut-and-count analysis (top) and multi-bin analysis (bottom). The
		upper panel shows the observed number of events and the predicted background
		before the fit. The background $t\tbar+X$ include $t\tbar W/Z$, $t\tbar H$,
		and $t\tbar t\tbar$ events. The multijet background is negligible. All
		uncertainties described in Section~\ref{mb:sysun} are included in the
		uncertainty band. The $t\tbar$ normalization is obtained from the fit and is
		shown in the bottom panel.}

	\label{f:fig_07ab}
\end{figure}



\begin{figure}[H]
	\includegraphics[width=10cm]{figures/fig_08a_mb}
	\includegraphics[width=10cm]{figures/fig_08b_mb}
	\centering

	\caption{Background fit extrapolated to the VRs of the cut-and-count analysis
		(top) and the multi-bin analysis (bottom). The $t\tbar$ normalization is
		obtained from the fit to the CRs shown in Figure~\ref{f:fig_07ab}. The upper
		panel shows the observed number of events and the predicted background. The
		background $t\tbar+X$ include $t\tbar W/Z$, $t\tbar H$, and $t\tbar t\tbar$
		events. The lower panel shows the pulls in each VR. The last row displays the
		total background prediction when the $t\tbar$ normalization is obtained from a
		theoretical calculation~\cite{Czakon:2011xx}.}

	\label{f:fig_08ab}
\end{figure}

\begin{table*}[htbp]
	\centering
	\begin{tabular}{lccc}
		\toprule
		& \multicolumn{3}{c}{SR-Gtt-1L} \\
		\midrule
		Targeted kinematics & B                 & M                 & C                 \\[-0.05cm]
		\midrule
		Observed events     & 0                 & 1                 & 2                 \\
		\midrule
		Fitted background   & 0.5 $\pm$ 0.4     & 0.7 $\pm$ 0.4     & 2.1 $\pm$ 1.0     \\
		\midrule
		$t\tbar$\           & 0.4 $\pm$ 0.4     & 0.5 $\pm$ 0.4     & 1.2 $\pm$ 0.8     \\
		Single-top          & 0.04 $\pm$ 0.05   & 0.03 $\pm$ 0.06   & 0.35 $\pm$ 0.28   \\
		$t\tbar+X$          & 0.08 $\pm$ 0.05   & 0.09 $\pm$ 0.06   & 0.50 $\pm$ 0.28   \\
		$Z$+jets            & 0.049 $\pm$ 0.023 & 0.050 $\pm$ 0.023 & $<0.01$           \\
		$W$+jets            & $<0.01$           & $<0.01$           & 0.024 $\pm$ 0.026 \\
		Diboson             & $<0.01$           & $<0.01$           & $<0.01$           \\
		\midrule
		MC-only background  & 0.43              & 0.45              & 1.9               \\
		\bottomrule
	\end{tabular}

	\caption{Results of the background-only fit extrapolated to the Gtt
		1-lepton SRs in the cut-and-count analysis, for the total background
		prediction and breakdown of the main background sources. The uncertainties
		shown include all systematic uncertainties. The data in the SRs are not
		included in the fit. The background $t\tbar+X$ include $t\tbar W/Z$,
		$t\tbar H$, and $t\tbar t\tbar$ events. The row MC-only background provides
		the total background prediction when the $t\tbar$ normalisation is obtained
		from a theoretical calculation~\cite{Czakon:2011xx}. }

	\label{tab:yield_discovery}

\end{table*}

\begin{figure}[H]
	\includegraphics[width=10cm]{figures/fig_09a}
	\includegraphics[width=10cm]{figures/fig_09b}
	\centering
	\caption{}
	\label{f:fig_09ab}
\end{figure}

%%%%%%%%%%%%%%%%%%%%%%%%%%%%%%%%%%%%%%%%%%%%%%%%%%%%%%%%%%%%%%%%%%%%%%%%%%%%%%%%%%%%%%%%%%%%

\section{Interpretation}\label{mbinter}

As no discovery can be claimed, one-sided upper limits at $95\%$ confidence
level (CL) are derived from the data. Section~\ref{mb:midel} discusses
model-independent exclusion limits and Section~\ref{mb:midel} discusses
model-dependent exclusion limits.


\subsection{Model-independent Exclusion Limits}\label{mb:midel}

For each SR, model-independent limits on the number of beyond-the-SM events are
derived. Pseudoexperiments in the CL$_{\text{s}}$
prescription~\cite{Read:2002hq} are used, neglecting possible signal
contamination in the CR. Here, only the single-bin regions in the cut-and-count
are used. Table~\ref{mod-ind-lim} shows the results. It includes the visible
beyond the Standard Model cross-section ($\sigma_{\text{vis}}^{95}$) obtained
by dividing the observed upper limits on the number of beyond the Standard
Model events with the integrated luminosity, as well as the $p_0$-values, which
represent the probability that the SM background alone would fluctuate to the
observed number of events or higher.

\begin{table}[H]
	\centering
	\small
	\begin{tabular*}{0.6\textwidth}{@{\extracolsep{\fill}}lcccc}
		\noalign{\smallskip}\toprule\noalign{\smallskip}
		Signal channel         & $p_0$ (Z)            & $\sigma^{95}_{\text{vis}}$ [fb]  &  $S_{\text{obs}}^{95}$  & $S_{\text{exp}}^{95}$   \\
		\noalign{\smallskip}\midrule \noalign{\smallskip}
		SR-Gtt-1L-B & $ 0.50~(0.00) $ &  $0.08$ &  $3.0$ & $ { 3.0 }^{ +1.0 }_{ -0.0 }$ \\[1mm]
		SR-Gtt-1L-M & $ 0.34~(0.42)$ &  $0.11$ &  $3.9$ & $ { 3.6 }^{ +1.1 }_{ -0.4 }$ \\[1mm]
		SR-Gtt-1L-C & $ 0.50~(0.00)$ &  $0.13$ &  $4.8$ & $ { 4.7 }^{ +1.8 }_{ -0.9 }$ \\[1mm]
		\noalign{\smallskip}\midrule\noalign{\smallskip}
	\end{tabular*}
	\caption{The $p_0$-values and $Z$ (the number of equivalent Gaussian standard deviations),
	the 95$\%$ CL upper limits on the visible cross-section
	($\sigma^{95}_{\text{vis}}$),
	and the observed and
	expected 95$\%$ CL upper limits on the number of BSM events ($S_{\text{obs}}^{95}$
	and $S_{\text{exp}}^{95}$). The maximum
	allowed $p_0$-value
	is truncated at 0.5.}
	\label{mod-ind-lim}
\end{table}

\subsection{Model-dependent Exclusion Limits}\label{mb:mdel}

The regions in the multi-bin analysis, from both leptonic hadronic channels,
are statistically combined to set model-dependent upper limits, using the
CL$_{\text{s}}$ prescription in the asymptotic approximation
\cite{Cowan:2010js}. The expected and observed limits are found to be
compatible with the CL$_{\text{s}}$ calculated from pseudoexperiments.

Figure~\ref{f:fig_10a} shows the $95\%$ CL observed and expected exclusion
limits in the neutralino and the gluino mass plane. The $\pm 1
	\sigma^{\text{SUSY}}_{\text{theory}}$ limit lines are obtained by changing the
SUSY cross-section by one standard deviation up and down
(Section~\ref{mbdatasm}). The yellow band around the expected limit shows the
$\pm 1\sigma$ uncertainty, including all statistical and systematatic
uncertainties, except the theoretical uncertainties in the SUSY cross-section.
The current search shows an improvement, compared to the previous
result~\cite{Aad:2016eki}, of $450$ GeV in gluino mass sensitivity, assuming
massless neutralino. Gluinos having masses below $1.97$ TeV are excluded at
$95\%$ CL for neutralino masses lower than $300$ GeV. The red line shows the
observed limit; at high gluino mass it is weaker than the expected limits, due
to a mild excess observed in the region SR-1L-HI (and a mild excess in a region
in the hadronic channel) of the multi-bin analysis.


\begin{figure}[H]
	\includegraphics[width=10cm]{figures/fig_10a}
	\centering

	\caption{Exclusion limits in the context of the multi-bin analysis. The dashed line shows
		the $95\%$ Cl expected limit, and the solid bold line the $95\%$ Cl observed limit. The shaded
		bands around the expected limits show the impact of experimental and background uncertainties.
		The dotted lines show the impact on the observed limit of the variation of the nominal signal
		cross-section by $\pm 1 \sigma$ of its theoretical uncertainty. Also shown are the
		$95\%$ CL expected and observed limits from the ATLAS search based on 2015 data \cite{Aad:2016eki}.}

	\label{f:fig_10a}
\end{figure}



\section{Conclusions}\label{mb:conc}

The analysis in this chapter presents a search for pair-produced gluinos, using
$\sqrt{s}=13$ TeV data collected at the LHC in the 2015-2016 data taking period
that correponds to an integrated luminosity of $36.1~\text{fb}^{-1}$. The
signal model discussed is Gtt, of which the leptonic final state involves at
least one lepton, large $E_{T}^{\text{miss}}$, and multiple jets among which at
least three must be $b$-jets. Several signal regions are defined to
accommondate alternative ranges of mass differences between the gluino and the
neutralino. Two analysis strategies are followed, the cut-and-count strategy in
which possibly overlapping signal regions are optimized for discovery, and the
multi-bin strategy in which non-overlapping signal regions are optimized for
model-dependent exclusions. The dominant source of background is $t\tbar+$jets,
whose normalization factors are obtained in dedicated control regions. No
excess relative to the Standard Model background can be claimed.
Model-independent limits are set on the visible cross-section for new physics
processes. The multibin regions in the leptonic channel are combined with those
in the hadronic channel to set model-dependent limits on gluino and neutralino
masses. For neutralino masses below approximately $300$ GeV, gluino masses of
less than $1.97$ TeV are excluded at the $95\%$ CL, which is an improvement
compared to the exclusion limits obtained with the 2015 dataset alone.

