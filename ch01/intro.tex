
For centuries philosophers and scientists, among them physicists, have pursued
the idea that there is a simplicity underneath the apparent complexity of
natural phenomena. This quest for simplicity, which to many is also a quest for
beauty, have led physicists to contemplate the universe at ever smaller and
ever larger scales. As they progress, physicists have become more and more
convinced that their quest is fruitful, that a satisfactory picture of the
physical world is attainable, even though major paradigm shifts have occurred
many times over.

In the search for a theoretical understanding of physical phenomena at ever
smaller scales, particle physicists have been guided by the idea that matters
are made up from a small number of elementary particles, that these particles
interact through certain fundamental forces, and that a knowledge of the
elementary particles and their fundamental interactions is equivalent to a full
understanding of the physical world.

This simple yet profound idea, that all one needs to know is a knowledge of the
elementary particles and their fundamental interactions, has been implemented
quantitatively. In the second half of the $20$th century, particle physicists
have been able to construct a theoretical framework, called the Standard Model,
in which the elementary particles and their interactions are identified and
classified, and on the basis of which many calculations have been carried out
with outstanding precision. In this respect, a famous example is the
calculation of the magnetic moment of the electron, where the agreement between
theoretical calculation and experimental measurement has reached the level of
ten decimal figures~\cite{electronmoment}, among the most precise in physics.

According to the Standard Model, there is a small number of elementary
particles that are classified into bosons and fermions. Fermions, which are the
constituents of matter, carry spin $1/2$ and, depending on their properties,
interact with each other by one or more fundamental forces by exchanging
bosons. The bosons, which have integer spins, are thus the force carriers, and
some of them may carry charges themselves, which make some of them capable of
interacting with each other. Particle physicists have, to a great extent,
succeeded in proposing and clarifying, both qualitatively and quantitatively,
these elementary interactions. More remarkably, they have found in the process
that the unifying principle of the Standard Model is the principle of symmetry.
The entire theoretical framework of the Standard Model is constrained by
spacetime symmetry, the Poincar\'{e} group, and local gauge symmetries, the
groups $U(1)$, $SU(2)$, and $SU(3)$, underlie its different components.

To particle physicists, the Standard Model, even though a fantastic achievement
of 20th-century physics and ranks among the greatest intellectual achievements
of all time, is not absolutely satisfactory. First of all, presently we know
that there are four fundamental forces that exist in nature. The
electromagnetic force takes place between particles that carry electric
charges. Subatomic forces, including the weak and the strong forces, take place
between particles that carry the weak and the strong charges, respectively. The
electromagnetic and the weak forces have been unified into a single force
called the electroweak force, and together with the strong force, make up the
three fundamental forces in the Standard Model. Gravity, a fundamental force
that takes place between any two particles that carry masses, is however not
part of the Standard Model. In this sense, the Standard Model is not seen as a
complete physical theory of nature.

There are other problems with the Standard Model as well. Neutrino masses, the
hierarchy problem, the nature of dark matter, the unification of the
electromagnetic, weak, and strong forces, are problems believed to lie beyond
the scope of the Standard Model.

Among the many ideas that have been proposed, supersymmetry is perhaps the one
that stands out and one that is most actively pursued. It is a theoretically
consistent framework that started with the question whether or not it would be
possible to extend spacetime symmetry. Indeed, gauge symmetries are not related
to spacetime symmetry, in the sense the commutators between the generators of
the gauge groups and those of the Poincar\'{e} groups all vanish. Spacetime
symmetry may be extended, provided, along with adding new generators to the
spacetime generators, we consider their anti-commutators. Supersymmetry unifies
bosons and fermions, thereby in a sense further simplifying of our picture of
the physical world. It, however, adds some complexity into our physical picture
of nature with a considerable increase of the number of elementary particles
and their interactions. At the same time, supersymmetry provides solutions to a
number of open questions that have been raised, such as the hierarchy problem,
dark matter, and the unification of the three fundamental forces of the
Standard Model.


For any physical theory, the ultimate test is experiments. Since its operation
in 2009, the Large Hadron Collider based at CERN, or the LHC as it is often
called, has given physicists opportunities to verify if supersymmetry is indeed
a symmetry of nature. A great number of physicists is participating in this
process, which is still going on. This they do by analyzing the data that have
been collected and are being collected at the LHC, searching for signs that
supersymmetry exists.

The LHC centre-of-mass energy makes it possible to probe a number of physics
models that extend the Standard Model. Many of these models predict unstable
hypothetical particles that decay into Standard Model particles such as top
quarks and weak bosons which, being unstable themselves, decay either
leptonically or hadronically. Leptonic processes, expected in a fraction of the
total number of interactions, produce electrons and muons. Thus electrons and
muons are important physics objects in the search for new physics, all the more
because leptonic processes are often not plagued with hadronic backgrounds
which are so numerous in high-energy interactions. On the other hand, pure
Standard Model processes, in particular the strong and weak interactions, also
take place abundantly at the LHC, and many of these interactions produce again
particles such as the top quarks and the weak bosons. Again, leptonic processes
are to be expected in a fraction of the total number of interactions, and thus
electrons and muons are important physics objects that need to be accurately
reconstructed and calibrated , in order that Standard Model background can be
accounted for as reliably as possible.


This thesis is one among many works that have been carried out by experimental
physicists at CERN as the search for supersymmetry continues. The common theme
is electrons, specifically improving the selection of signal electrons in SUSY
searches\footnote{The works are applicable to other beyond-the-Standard-Model
	searches.}. The contexts of the work are as follow.


\begin{itemize}[label=\ding{111}]

	\item In some SUSY searches, the final state consists of a pair of same-sign
	      leptons, where the leptons are electrons and muons. In general, SUSY cross
	      sections are much smaller than the Standard Model background cross section and,
	      as we continue to push to unexplored phase space, we often have to deal with
	      situations that involve a very small signal on top of a very large background.
	      Correct determination of the charges of the leptons is very important, as
	      charge mis-identification rates occur on the order of $\text{O}(1\%)$, while
	      Standard Model processes that provide opposite-sign dileptons (dominantly $Z\to
		      e^+e^-$ bosons) occur approximately $10^3$ times more commonly than genuine
	      Standard Model sources of same-sign leptons (dominantly $WZ$ production), and
	      as a result, opposite-sign sources of dileptons can constitute a large
	      background in these searches. At ATLAS, electron charge is determined in the
	      Inner Detector which is embedded in a solenoidal magnetic field. This
	      determination is not always correct, however, due to the apparent straightness
	      of a track or bremsstrahlung, and as a result the sign of the charge might be
	      mis-measured, or mis-identified. In this thesis, the estimation of the rate of
	      charge mis-identification by a likelihood function is described in
	      chapter~\ref{c:cid}.

	\item Many SUSY searches target strongly-interacting processes, as these have
	      relatively high cross sections. Processes that involve pair productions of
	      gluinos, which are hypothetical partners of the Standard Model gluons, is an
	      example. These super-particles are also highly motivated as they are expected
	      by naturalness to have a mass around the TeV scale~\cite{Barbieri:1987fn}. This
	      thesis describes, in chapter~\ref{c:susys}, the search in which the final state
	      consists of large missing transverse momentum due to the neutralino, as well as
	      multiple jets, where at least three of the jets must be $b$-jets. The focus is
	      on the leptonic channel\footnote{The hadronic channel which requires zero
		      lepton is also part of the analysis; however, the chapter only focuses on the
		      leptonic channel as it is directly related the work done in this thesis.} where
	      electrons and muons are involved, and we describe in some detail a new scheme
	      of overlap removal between jets and muons that was introduced into the
	      analysis, to maximize signal acceptance in the presence of
	      dileptonically-decaying boosted top quarks.

	      In the same chapter we also discuss the optimization of some important
	      discriminating variables, the result of which was used in the design of the
	      signal regions of the analysis.

	\item Starting from Run 2, the LHC centre-of-mass was upgraded to $13$ TeV. In
	      a number of SUSY searches that involve leptons in the final state, more
	      electrons were found to be within $\Delta R = 0.4$ of jets. The SUSY search
	      that involves pair productions of gluinos discussed above is an example, in
	      which considerable signal acceptance was gained when electrons within $\Delta R
		      = 0.4$ were selected. At ATLAS, only standard electrons are calibrated, and it
	      is important to make sure that this calibration, in particular for the
	      identification efficiencies, remains valid for electrons found within $\Delta R
		      = 0.4$ of jets. This thesis develops a method for measuring the identification
	      efficiencies for electrons found inside jets and performs the initial
	      measurement. The measurement uses a dilepton ($e\mu$) $t\tbar$ sample enriched
	      in boosted top quarks. The top quark decays almost all of the time into a $W$
	      and $b$-quark, and the electrons in the sample are located inside the
	      $b$-quarks. The work is described in chapter~\ref{c:eid}.

	      Prior to the work initiated in this thesis, no attempt had been made to measure
	      the identification efficiencies for these electrons.

\end{itemize}


The remaining chapters are organized as follow. Chapter~\ref{c:sm} discusses
briefly the symmetry principles that underlie the Standard Model, as well as
its particle contents and forces. In the same chapter we give also a general
discussion of the shortcomings of the Standard Model as well as a discussion of
the basic ideas of supersymmetry. Chaper~\ref{c:detector} describes the LHC,
including a discussion of the LHC accelerator and the ATLAS detector. All the
works in this thesis are associated with the ATLAS experiment.
Chapter~\ref{c:ereid}, on the other hand, gives an introduction of electron
reconstruction and identification at ATLAS.
