\chapter{IN-JET ELECTRON IDENTIFICATION EFFICIENCIES}\label{c:eid}

In early 2015 the LHC restarted after two years of shutdown, beginning what is
referred to as Run 2. The new cent-of-mass energy was $13$ TeV, in place of
the previous $8$ TeV. The higher energy opens up unexplored parameter space and
allows further probe of supersymmetry (SUSY) and other
beyond-the-Standard-Model processes~\cite{multib-c7, stop-c7, ttresonance-c7,
	vlq-c7}. At ATLAS, many SUSY searches involve supersymmetric particles that
decay into the Standard Model top quarks and we expect, with higher
centre-of-mass energy, higher sensitivity to more massive supersymmetric
particles and consequently an increase in the production of high $p_T$ top
quarks. Since the top quarks also decay --- essentially all the time --- into a
$W$ boson and a $b$ quark, we in turn expect boosted decay topology, in other
words the daughter particles of the top quarks, which include the daughter
particles of the $W$ boson and the $b$ quark, to stay close to each other
(Figure~\ref{f:boosttop}).



This chapter describes the work to measure the identification efficiencies for
electrons that are found inside $\Delta R = 0.4$ of high $p_T$ jets, which will
also be called in-jet electrons, using $t\tbar$ events. Prior to the work in
this chapter, there was no attempt to measure the identification efficiencies
for in-jet electrons in a $t\tbar$ topology. The chapter will be organized as
follows. In section~\ref{s:eidmot} we motivate the need for the measurement of
the identification efficiencies for in-jet electrons. Section~\ref{s:eidmet}
describes the method used to perform the measurements and presents the measured
efficiencies. Section~\ref{s:eidcon} presents some conclusions.

The data used for this chapter was collected in the period 2015-2016 at $13$
TeV center-of-mass and corresponded to an integrated luminosity of
$36.5~\text{fb}^{-1}$.

\section{Motivation}\label{s:eidmot}

Prior to Run 2, ATLAS center-of-mass energy 7-8 GeV allows limited sensitivity
to high mass resonances. Because many beyond-Standard-Model particles are
predicted to decay into the Standard Model top quarks, the limited sensitivity
reduces the chance in which we could expect boosted top quark decays. Such
decays, however, are expected to become significant as the centre-of-mass of
the LHC reached $13$ TeV starting from Run 2. In a boosted top quark decay
scenario, the produced particles, which in this case are the daughters of the
$W$ and the $b$ quark that come from the top quark, are found close to each
other~\cite{hadronic01-ch7, hadronic02-ch7, bdis-figs}.
Figure~\ref{f:drwbtoppt} shows the angular distance $\Delta R$
(Formula~\ref{eq:angulardr}) between the $W$'s and the $b$-quarks as a function
of the top $p_T$, in the context of a hypothetical particle $Z'$ with mass
$m_{Z'}=1.6$ TeV~\cite{bdis-figs} that decays into a $t\bar{t}$ pair. Also
shown in the same figure is the separation between the light quarks of the
subsequent hadronic decay of the $W$ boson. As can be seen, the angular
distance decreases as the top quark $p_T$ increases, and at high top quark
$p_T$ a non-negligible fraction of the distances becomes very small.


\vspace{3mm}

\begin{figure}[H]
	\includegraphics[width=12cm]{figures/boost}
	\centering

	\caption{An illustration of low $p_T$ top quark decay (left) and boosted top
		decay (right) of a high $p_T$ top quark. In the case of high $p_T$ top quark
		decay the daughter particles of the top quark, which include the daughter
		particles of the $W$ and the $b$ quark, are expected to be found close to each
		other~\cite{boostedtop-fig}.}

	\label{f:boosttop}
\end{figure}

\vspace{3mm}

Leptonic boosted top quark decay is also an important channel in searches for
beyond-Standard-Model particles that decay into the Standard-Model top quarks.
Table~\ref{t:injetfraction} shows a measurement of the fraction of in-jet
electrons over signal electrons as a function of the top quark $p_T$, at
truth-level. The measurement used PowhegPythia $t\bar{t}$ events simulated at
$13$ TeV centre-of-mass energy (Chapter~\ref{c:susys}, Section~\ref{mbdatasm}),
where dilepton events consisting of a muon and an electron were selected. The
selections made use of the $p_T$-dependent overlap removal

\begin{equation}\label{c7overlapr}
	\Delta R <  \text{min (0.4, 0.04 + 10 GeV} / p_T),
\end{equation}

where the overlap removal is required to keep the overlapping $b$-jets
(Chapter~\ref{c:susys}, Section~\ref{mbobjs}). As is shown, more and more
electrons are found inside jets as the top quark $p_T$ increases. The number of
in-jet electrons becomes quite significant from $500$ GeV, being approximately
$25\%$ there and reaching nearly $40\%$ at $650$ GeV. If the top quark $p_T$ is
allowed to go up to $1$ TeV, the figure is $64\%$. This result supports the
fact that different ATLAS analyses searching for heavy beyond-Standard-Model
particles decaying into lighter sparticles, such as the gluinos and stops that
decay into neutralinos, in which the final state involve the Standard Model top
quarks, were able to increase signal acceptances considerably if in-jet
electrons were selected~\cite{multib-c7, stop-c7}.

This chapter develops a method and performs the initial measurements for the
identification efficiencies of electrons found inside $\Delta R=0.4$ of jets.
The measurements for electrons outside jets are done by the ATLAS Egamma
group~\cite{atlaselcid, eleffme}.

\begin{figure}[H]

	\begin{subfigure}{0.5\textwidth}
		\includegraphics[width=7.5cm]{figures/fig_01a}
		\caption{$t\to Wb$}
		\label{drwbtoppta}
	\end{subfigure}
	\begin{subfigure}{0.5\textwidth}
		\includegraphics[width=7.5cm]{figures/fig_01b}
		\caption{$W\to q\bar{q}$}
		\label{drwbtopptb}
	\end{subfigure}

	\centering
	\caption{(\ref{drwbtoppta})~The angular distance $\Delta R$ between
		the $W$'s and the $b$ quarks as a function of the top quark $p_T$ simulated
		PYTHIA~\cite{sjostrand:ch7}, in the context of a hypothetical particle $Z'$
		($m_{Z'}=1.6$ TeV) that decays into a $t\bar{t}$ pair. At high top quark $p_T$
		a non-negligible fraction of the distances is seen to be very
		small.~(\ref{drwbtopptb})~The angular distance between two light quarks from
		$t\to Wb$ decay as a function of the $p_T$ of the $W$ boson~\cite{bdis-figs}.}
	\label{f:drwbtoppt}
\end{figure}


\renewcommand{\arraystretch}{1.15}
\begin{table}
	\centering
	\begin{tabular}{||c c||}
		\hline
		Top quark $p_T$ (GeV) & Fraction  \\ [0.5ex]
		\hline\hline
		\toprule
		$\leq 300$            & $ 8.4\%$  \\
		\hline
		$\leq 425$            & $ 17.2\%$ \\
		\hline
		$\leq 500$            & $ 24.0\%$ \\
		\hline
		$ \leq 650$           & $ 39.0\%$ \\
		\hline
		$ \leq 750$           & $ 49.0\%$ \\
		\hline
		$ \leq 800$           & $ 53.0\%$ \\
		\hline
		$ \leq 900$           & $ 59.0\%$ \\
		\hline
		$ \leq 1000$          & $ 64.0\%$ \\
		\hline
	\end{tabular}

	\caption{The fraction of in-jet electrons over the number of signal
		electrons, both at truth-level, as a function of the top quark $p_T$. The
		fraction increases and becomes very significant at high top quark $p_T$.}

	\label{t:injetfraction}
\end{table}
\renewcommand{\arraystretch}{1.0}




%%%%%%%%%%%%%%%%%%%%%%%%%%%%%%%%%%%%%%%%%%%%%%%%%%%%%%%%%%%%
\section{Method}\label{s:eidmet}

The method that is used to measure the identification efficiencies for in-jet
electrons is discussed in detail in Section~\ref{ss:eidtp},~\ref{ss:eidsp}, and
~\ref{ss:eidsr}. Background estimations is described in Section~\ref{s:eidbe}.

\subsection{Boosted Dilepton $e\mu$ Events}\label{ss:eidtp}

In order to measure the identification efficiencies for in-jet electrons, a
sample of reconstructed electrons (Chapter~\ref{c:susys}, Section~\ref{mbobjs})
inside high-$p_T$ jets was obtained by selecting boosted $t\bar{t}$ dilepton
($e\mu$) events. This is expected to result not only in a very pure $t\tbar$
sample, but also in a topology close to that of many SUSY and other
beyond-Standard-Model searches. In contrast, the standard method for measuring
electron identification efficiencies, the tag-and-probe method supported by the
Egamma group at ATLAS~\cite{eleffme}, makes use of $Z\to e^+e^-$ events for
high-energy electrons ($E_{\text{T}} > 10$ GeV). Even though a clean sample of
electrons may be obtained in a relatively straightforward way by selecting
events around the $Z$ mass peak, we expect, if electrons inside high $p_T$ jets
are required, a sample unrepresentative of events with a boosted topology and
limited in statistics.

Thus, a sample of electrons was obtained by selecting a hard muon and a
reconstructed electron with only a $p_T$ requirement applied, which will also
be referred to below simply as reconstructed electrons. The efficiency at a
particular identification operating point (Section~\ref{ss:elop}) is defined by
the ratio

$$
	\text{ID efficiency} = \frac{\text{The number of identified
			electrons}}{\text{The number of reconstructed electrons}}
$$

Both the numerator and the denominator are contaminated with background
electrons which require a careful estimate (Section~\ref{s:eidbe}),
particularly because background electrons are expected to reside primarily
inside jets.

%%%%%%%%%%%%%%%%%%%%%%%%%%%%%%%%%%%%%%%%%%%%%%%%%%%%%%%%%%%%
\subsection{Data and Monte Carlo Samples}\label{ss:eidsp}

The data used for this chapter was collected in the period 2015-2016 and
corresponds to an integrated luminosity of $36.5~\text{fb}^{-1}$. The following
simulation samples (Chapter~\ref{c:susys}, Section~\ref{mbdatasm}),
at $13$~TeV centre-of-mass energy, are used:


\begin{itemize}

	\item $t\bar{t}$ events from the Powheg+Pythia generator. As a hard muon will
	      be required in the sample in which the identification efficiencies are measured
	      (Section~\ref{ss:eidsr}), these events naturally partition into either a
	      dileptonic set ($e\mu$) when truth-level electrons are present inside
	      high-$p_T$ jets, or a semileptonic set otherwise. The latter, with jets from
	      the fully hadronic decay of one of the top quarks constituting a source of
	      background electrons, is expected to be a dominant background.

	\item $W$+jets, which will be used as a background. As the $W$ boson may
	      produce a hard muon, the presence of jets make these events a source of
	      background events to signal dilepton $e\mu$ $t\tbar$ events.

	\item Single top events, which include the $Wt$ production as well as the
	      $s$-channel and $t$-channel productions. The $Wt$ production is treated as a
	      source of signal electrons, since it contains a pair of $W$ bosons that can
	      decay to a prompt $e\mu$ pair, whereas the remaining two productions each
	      contains only one $W$ boson and as a result cannot produce a prompt $e\mu$
	      pair.

\end{itemize}
%%%%%%%%%%%%%%%%%%%%%%%%%%%%%%%%%%%%%%%%%%%%%%%%%%%%%%%%%%%%
\subsection{Signal Region}\label{ss:eidsr}

The kinematic region in which the measurement of the identification
efficiencies is performed is called the signal region. It is defined after the
following preliminary selections, which are called the pre-selection cuts and
aimed at isolating dilepton $e\mu$ $t\tbar$ events, are applied.

\paragraph{Pre-selection}

\begin{itemize}
	\item One primary vertex

	\item Muon trigger. The following triggers were used for the periods $2015$
	      and $2016$:

	      \begin{itemize}
		      \item $2015$: \texttt{HLT_mu26_imedium || HLT_mu40}
		      \item $2016$:  \texttt{HLT_mu26_ivarmedium || HLT_mu50}
	      \end{itemize}

	\item $p_T$-dependent overlap removal, where the overlapping $b$-jets are kept
	      (Formula~\ref{c7overlapr}).

	\item Events with bad or cosmic muons are removed. Highly energetic jets could
	      reach the muon spectrometer and create hits in the latter, or jet tracks in the
	      inner detector could be erroneously matched to muon spectrometer segments, both
	      of which cases are sources of bad muons. Events with these muon candidates,
	      along with those having muons from cosmic rays, are rejected.

	\item Exactly one identified muon and $\geq 1$ electrons inside jets are required
	      for each event, where

	      \begin{itemize}

		      \item The muon is required to have $p_T > 30$ GeV, $d_0 / \sigma(d_0) < 3.0$,
		            and $z_0 < 0.5$ in terms of the transverse impact parameter and the
		            longitudinal impact parameter. It must also have $\text{ptvarcone}30 / p_T <
			            0.06$, where $\text{ptvarcone}30$ is defined as the scalar sum of the momenta
		            of the tracks with $p_T > 1$ GeV in the cone with $\Delta R <
			            \text{min}(10\text{GeV}/p_T, 0.3)$, and must be a muon that has been triggered.

		      \item The electrons must have $p_T \geq 30$ GeV, which is a common cut in most
		            analyses where in-jet electrons are used, and must overlap within $\Delta R <
			            0.4$ with some jets. There could be more than one electron present in the
		            event, however the leading $p_T$ electron will be used,

	      \end{itemize}

	\item $\geq 1$ $b$-tagged jet, instead of exactly $2$ $b$-tagged jets as is
	      usually expected in $t\tbar$ events, since we are selecting events with
	      electrons inside jets and the $b$-tagging efficiency may suffer because the
	      tracks of the electron, which is expected to originate from the interaction
	      point, may confuse the $b$-tagging algorithm.

\end{itemize}

These cuts result in a set of 3183 events with one hard muon and at least one
electron candidate found inside some jet. In the following, we discuss several
variables that have been found to be discriminating, along with their
distribution plots. Simulation shows an expected $814.2$ dilepton events and
$178.5$ single top $Wt$ production events. On the other hand, the prominent
source of background comes from semileptonic events, predicted to be $2010.8$,
whereas $W$+jets and single top $s$-channel and $t$-channel constitute two
small sources of background, predicted to be $315.5$ and $19.1$ respectively.

\begin{itemize}

	\item The mass of the large radius jet that overlaps with the probe electron,
				shown in Figure~\ref{f:premrjet} and denoted $m_{\text{rjet}}^{\text{el}}$. 
				The large radius jet is reclustered from the
	      small radius jets present in the events (Chapter~\ref{c:susys},
	      Section~\ref{mbobjs}), and accordingly in semileptonic events it is expect to be
	      more massive, as it picks up the masses of the jets from the hadronic decay of
	      one of the top quarks. In dileptonic events, on the other hand, there are fewer jets
	      due to leptonic decays of both of the top quarks, and in addition the
	      neutrino that accompanies the electron may reduce the visible mass of the
	      reconstructed large radius jet. As is shown in the figure, the higher mass
	      region is dominated by background events.


	\item The number of jets, which is shown in Figure~\ref{f:prenjets} and denoted 
	      $\text{N}_{\text{jet}}$. Three jets
	      are expected from a fully hadronic decaying top quark, as compared to only one
	      jet from a semileptonic decay, and as a result semileptonic events, in which
	      one top quark decays hadronically and one decays semileptonically, is expected
	      to have a greater number of jets than dileptonic events, where both jets decay
	      hadronically. In the figure, the semileptonic distribution is seen to be higher
	      everywhere.


	\item The sum of the transverse momenta of all jets, shown in
	      Figure~\ref{f:prehtjet}. As above, a larger number of jets is expected in
	      semileptonic events due to the fully hadronic decay of one of the top quarks,
	      and in dileptonic events fewer jets are expected because of leptonic decays of
	      both of the top quarks. Consequently a sum over all transverse momenta of the
	      jets is expected to lead to a discriminating distribution. As is seen in the
	      figure, the semileptonic distribution is higher everywhere.


	\item The transverse momenta of the jet closest to the probe, which is shown in
	      Figure~\ref{f:preptcjet}. This variable allows the removal of low $p_T$ jets
	      overlapping with background electrons.


	\item The fraction of the transverse momentum of the probe electron over that
	      of the closest jet (Figure~\ref{f:preptfractioncjet}). We expect real electrons
	      from the $W$'s produced from the top quarks to have higher $p_T$ than
	      background electrons. In the figure, the low $p_T$ region can be seen to be
	      dominated by semileptonic events.

\end{itemize}

\begin{figure}[H]
	\includegraphics[width=10cm]{figures/nvariables_pt40_probes_pre_m0_pt100_mar05_loose_real_loose_fake_m_rjet_overlap_el_mar22}
	\centering
	\caption{$m_{\text{rjet}}^{\text{el}}$. The semileptonic contribution is higher everywhere, especially on the right side of the distribution where
		there is little signal contamination.}
	\label{f:premrjet}
\end{figure}

\begin{figure}[H]
	\includegraphics[width=10cm]{figures/nvariables_pt40_probes_pre_m0_pt100_mar05_loose_real_loose_fake_njets_njets_mar22}
	\centering
	\caption{$\text{N}_{\text{jet}}$. The semileptonic contribution is higher because of the hadronic decay of one of the tops.}
	\label{f:prenjets}
\end{figure}

\begin{figure}[H]
	\includegraphics[width=10cm]{figures/nvariables_pt40_probes_pre_m0_pt100_mar05_loose_real_loose_fake_ht_jet_ht_jet_mar22}
	\centering
	\caption{The sum of the transverse momenta of all jets.}
	\label{f:prehtjet}
\end{figure}


\begin{figure}[H]
	\includegraphics[width=10cm]{figures/nvariables_pt40_probes_pre_m0_pt100_mar05_loose_real_loose_fake_pt_c_jet_pt_c_jet_mar22}
	\centering
	\caption{$p_T$ of the jet closest to the probe electron. }
	\label{f:preptcjet}
\end{figure}


\begin{figure}[H]
	\includegraphics[width=10cm]{figures/nvariables_pt40_probes_pre_m0_pt100_mar05_loose_real_loose_fake_pt_fraction_c_jet_pt_fraction_c_jet_mar22}
	\centering
	\caption{Fraction of the $p_T$ of the probe electron over that of the closest jet. The lower $p_T$ region is dominated by semileptonic events.}
	\label{f:preptfractioncjet}
\end{figure}

\paragraph{Further cuts to arrive at the signal region}\label{p:eidfurthercuts}

Of all the discriminating variables shown above, $m_{\text{rjet}}^{\text{el}}$
seems to be the most discriminating variable. In addition, its distribution
shows two distinct regions, one abundant in signal electrons and one largely
dominated by background electrons. As will be discussed later in the chapter,
the region $< 60$ GeV will define the signal region where the identification
efficiencies are measured, and the region $> 60$ GeV will define the control
region for background estimation. With this in mind, we decided to apply cuts
on the other discriminating variables to further remove the undesired
background, while leaving $m_{\text{rjet}}^{\text{el}}$ untouched.

The cuts are as follows:

\begin{itemize}
	\item Missing transverse momentum $E_{\text{T}}^{\text{miss}} > 25$ GeV, 
	to ensure that the QCD multi-jet background is negligible.

	\item The number of jets $< 5$ and sum of $p_T$ of jets $< 700$ GeV, to remove
	      semileptonic events (Figure~\ref{f:prenjets} and~\ref{f:prehtjet}).


	\item $p_T$ of jet closest to the probe is between $150$ GeV and $500$ GeV, to
	      remove semileptonic events (Figure~\ref{f:preptcjet}) and at the
	      same time make sure that boosted $t\tbar$ dilepton events are selected.


	\item $p_T(\text{probe})  / p_T(\text{closest jet}) > 0.16 $ (Figure~\ref{f:preptfractioncjet}).

\end{itemize}

The resulting distribution $m_{\text{rjet}}^{\text{el}}$ is shown in
Figure~\ref{f:crmrjet}. There are $1102$ events, of which $734$ are in the
signal region $< 60$ GeV and $368$ in the background-dominated region $\geq 60$
GeV. In the signal region, simulation shows an expected $484.5$ dilepton events
and $29.65$ single top $Wt$ production events, whereas for the background
semileptonic events, $W$+jets, and single top $s$-channel and $t$-channel are
predicted to be $229.6$, $4.1$, and $91.6$ respectively.


\begin{figure}[H]
	\includegraphics[width=10cm]{figures/n_pt40_probes_cr1_m0_pt100_mar05_loose_real_loose_fake_m_rjet_overlap_el_mar22}
	\centering

	\caption{$m_{\text{rjet}}^{\text{el}}$ after the pre-selection cuts. The
		region $< 60$ GeV will define the signal region, and the region $\geq 60$ GeV
		will define the control region for background estimation.}

	\label{f:crmrjet}
\end{figure}

%%%%%%%%%%%%%%%%%%%%%%%%%%%%%%%%%%%%%%%%%%%%%%%%%%%%%%%%%%%%
\subsection{Background Estimation}\label{s:eidbe}

The identification efficiency for electrons inside jets depends on the
particular operating point (Loose, Medium, or Tight) at which the measurement
is carried out. Such an efficiency, which will be denoted $\epsilon$, is the
ratio of a numerator and a denominator (Section~\ref{s:eidmet}), both of which
are expected to be contaminated with background electrons that need to be
estimated. If $P$ denotes the number of electron candidates passing a
particular ID operating point, $B_P$ the number of background electrons passing
the operating point, $N$ the total number of reconstructed electron candidates
in the sample, and $B_N$ the number of background electrons present in the
sample, the efficiency $\epsilon$ may be written as

\begin{equation}\label{eqn:effo}
	\epsilon = \frac{P-B_P}{N-B_N}
\end{equation}

Because analyses using in-jet electrons all use the Medium or Tight operating
point, these are the only two points which will be measured in this chapter.
Accordingly, a Medium or Tight ID selection will be applied on the sample
representing the denominator, giving in each case the required numerator.
Background estimations will consist of estimating the term $B_P$ separately for
Medium and Tight in the numerator, and estimating the common term $B_N$ in the
denominator.

\paragraph{Estimating $B_P$} Since we expect background electrons to rarely
pass the Medium or Tight ID points, we expect in turn the term $B_P$ to be very
small in either case. Thus $B_P$ is taken directly from simulation, and the
measurements are not expected to be affected significantly.

Figure~\ref{f:eidmedium} shows the $m_{\text{rjet}}^{\text{el}}$ distributions
for electrons that pass the Medium and Tight selections. The distributions are
obtained by applying a Medium or Tight ID selection in addition to the
selections that define the signal region (Section~\ref{p:eidfurthercuts}). The
number of background electrons predicted by the simulation can be seen to be
indeed small in each case, accounting for only $0.3\%$ of the total number in
the Medium case and $0.1\%$ in the Tight case.


\begin{figure}[H]
	\includegraphics[width=7.5cm]{figures/pt40_probes_cr1_medium_m0_pt100_mar05_medium_real_medium_fake_m_rjet_overlap_el_mar22}
	\includegraphics[width=7.5cm]{figures/pt40_probes_cr1_tight_m0_pt100_mar05_tight_real_tight_fake_m_rjet_overlap_el_mar22}

	\centering

	\caption{The distribution of $m_{\text{rjet}}^{\text{el}}$ for electrons passing
		the Medium (left) and Tight (right) operating points. Background electrons figure $0.3\%$ and
		$0.1\%$ respectively.}

	\label{f:eidmedium}

\end{figure}

%%%%%%%%%%%%%%%%%%%%%%%%%%%%%%%%%%%%%%%%%%%%%%%%%%%

\paragraph{Estimating $B_N$} The term $B_N$ represents background contamination
from fake electrons found in $N$ (Formula~\ref{eqn:effo}). Since $N$ contains
only reconstructed in-jet electrons with no ID applied, estimating $B_N$ is
expected to be the most challenging part of the measurements.

The method employed for estimating $B_N$ in the following makes use the set of
electrons that fail the Loose ID selection, which will be called antiloose
electrons hereafter. These electrons are made up of two parts, one in the
signal region ($\leq 60$ GeV) and one in the background-dominated region ($>
	60$ GeV, Figure~\ref{f:crmrjet}). The part in the background-dominated region
will be used to obtain a normalization factor, which will then be applied to
the part in the signal region to estimate the number of background electrons.
In what follows, the set of antiloose electrons will also be referred to as the
fake electron template. Its part in the signal region will be denoted by $T$,
and that in the background-dominated region will be denoted by $T_>$.

In order to check if the set of antiloose electrons would be a suitable
distribution, the set of background electrons in $N$, namely $B_N$, is plotted
against the former and shown in Figure~\ref{f:shapebeforeafter}, both
normalized to unity. As is seen in the figure, the antiloose selection is
expected to be effective for classifying background electrons in the sample. On
the other hand, Figure~\ref{f:antiloosemrjet} shows the composition of
antiloose electrons in the $m_{\text{rjet}}^{\text{el}}$ distribution.
Simulation predicts about $10\%$ of signal electron contamination, but
otherwise the distribution is made up of mostly background electrons
dominated by semileptonic $t\tbar$.


\begin{figure}[H]
	\includegraphics[width=10cm]{figures/w_allbn_probes_m_rjet_overlap_el_m0_pt100_antiloose_m_rjet_overlap_el_mar22}
	\centering
	\caption{The distribution $m_{\text{rjet}}^{\text{el}}$ of $B_N$ against that
		of $T$, normalized to unity. $T$ describes very well $B_N$ and therefore it is
		reasonable to estimate $B_N$ using $T$.}
	\label{f:shapebeforeafter}
\end{figure}

\begin{figure}[H]
	\includegraphics[width=10cm]{figures/probe_pt40_probes_contam_m0_pt100_mar05_antiloose_m_rjet_overlap_el_mar22}
	\centering
	\caption{The distribution $m_{\text{rjet}}^{\text{el}}$ for electrons that fail
		the Loose ID point, also called antiloose electrons.}
	\label{f:antiloosemrjet}
\end{figure}

Background estimation using antiloose electrons proceeds in detail as follows:

\begin{enumerate}

	\item First, $T$ and $T_>$ are obtained by selecting antiloose electrons. Thus the method
	      is data-driven, $T$ and $T_>$ from simulations are not used.

	\item In addition to $N$, the set of reconstructed electron candidates in the
	      signal region, there is also the set of reconstructed electron candidates in the
	      background-dominated region, which will be denoted $N_>$.

	      Signal contamination is substracted from $T_>$, the resulting set of which is
	      denoted $\overline{T}_>$, and signal contamination is subtracted from $N_>$, where
	      the resulting set is denoted by $\overline{N}_>$. Then $\overline{T}_>$ is normalized to
	      $\overline{N}_>$, to obtain a normalization factor.

	\item Signal contamination is subtracted from $T$, the resulting set of which is
	      denoted $\overline{T}$, and the normalization factor is applied to $\overline{T}$,
	      to obtain the number of background electrons in the signal region.

\end{enumerate}

In other words, the background to be estimated in the signal region, $B_N$, is
measured according to

\begin{equation}\label{eid:bne}
	B_N =  \overline{T} \times \frac{\overline{N}_>}{\overline{T}_>}
\end{equation}

The following section discusses signal contamination subtractions in $T$,
$T_>$, and $N_>$, and the measurements of the idenfitication efficiencies.

\subsection{The Measurements of the Identification Efficiency}\label{s:meffs}

The idenfitication efficiency $\epsilon$ shown in Formula~\ref{eqn:effo}, where
$B_P$ is taken from simulation and $B_N$ is evaluated according to
Formula~\ref{eid:bne}, is


\begin{equation}\label{eid:efff}
	\epsilon = \frac{P-B_P}{N - \overline{T} \times \frac{\overline{N}_>}{\overline{T}_>}}.
\end{equation}


$\overline{T}$ is the set of antiloose electrons in the signal region, $T$,
minus signal contamination, and $\overline{T}_>$ is the corresponding quantity
in the background-dominated region. As there is an expected of $10\%$ of signal
contamination in the set of antiloose electrons
(Figure~\ref{f:shapebeforeafter}), $\overline{T}$ and $\overline{T}_>$ will be
obtained by subtracting signal contamination as predicted by simulations from
$T$ and $T_>$ respectively.

On the other hand, signal contamination in $N_>$ (Figure~\ref{f:crmrjet}), from
which $\overline{N}_>$ is obtained, is larger. In fact, as has been mentioned
at the end of Section~\ref{ss:eidsr}, there are $368$ events in the
background-dominated region, of which simulation predicts signal contamination,
made up of dilepton events and single top $Wt$ production events, to be $60.5 +
20.4 = 80.9$ events. In order to reduce the contribution from the estimation of
this signal contamination to the uncertainty in the efficiency we will use a
data-driven approach. According to Figure~\ref{f:eidmedium}, the number of
background electrons after a Medium or Tight ID selection is expected to be
negligible. We expect as a result $P$, and the corresponding quantity $P_>$ in
the background-dominated region, to be relatively free of background electrons.
Thus $P_>$ could be used to represent signal contamination in $N_>$, provided
the corresponding identification efficiency is properly taken into account. In
other words,

%
$$\overline{N}_> = N_> - P_> / \epsilon $$
%

where the efficiency in~\ref{eid:efff}, which is being measured, is used again.
The efficiency will be evaluated iteratively, until the change from one
iteration to the next is less than $0.5\%$. The value of $0.5\%$ will be taken
as the uncertainty due to signal contamination subtraction in $N_>$.

The efficiencies, as well as the total statistical and systematic uncertainties
(Section~\ref{s:eidunc}), are $\mathbf{0.870\pm 0.017 \pm 0.031}$ for Medium
and $\mathbf{0.784 \pm 0.019 \pm 0.020}$ for Tight. As is seen, the efficiency
is higher for Medium than for Tight, consistent with expectation. The
statistical uncertainties are slightly larger for Tight, also consistent with
expectation, as the stats for Tight is slightly less than that for Medium. The
relevant quantities in Formula~\ref{eid:efff} that are used to compute the
efficiencies in data are listed in Table~\ref{t:inteffqe}.

\renewcommand{\arraystretch}{1.15}
\begin{table}[H]
	\centering
	\begin{tabularx}{0.6\textwidth}{| c | *{2}{Y|} }
		\cline{2-3}
		\multicolumn{1}{c |}{} & MEDIUM & TIGHT   \\
		\hline\hline
		\toprule
		~~~~~~$P$~~~~~~        & $ 392$ & $ 356$  \\
		\hline
		$B_P$                  & $1.47$ & $ 0.40$ \\
		\hline
		$N$                    & $734$  & $ 734$  \\
		\hline
		$\thead{\overline{N_>} \\ }$ & $368$ &  $ 368$ \\
		\hline
		$P_>$                  & $49$   & $ 48$   \\
		\hline
		$\thead{\overline{T} \\ }$  & \multicolumn{2}{c |}{$267.35$} \\
		\hline
		$\thead{\overline{T_>} \\ }$  &  \multicolumn{2}{c |}{$292.52$} \\
		\hline
		\toprule
	\end{tabularx}
	\caption{The relevant quantities for computing the efficiencies according to
		Formula~\ref{eid:efff}.}
	\label{t:inteffqe}
\end{table}
\renewcommand{\arraystretch}{1}

The efficiencies and statistical uncertainties in simulation for the Medium and
Tight operating points are also computed and are $\mathbf{0.871\pm 0.010}$ and
$\mathbf{0.807\pm 0.011}$ respectively. Thus Medium in data and in simulation
agree, while there is a deviation of about $2\%$ for Tight, possibly revealing
the difficulty of modeling accurately electrons inside jets for the latter
operating point.


The next section discusses in detail the treatment of statistical and
systematic uncertainties.

%%%%%%%%%%%%%%%%%%%%%%%%%%%%%%%%%%%%%%%%%%%%%%%%%%%%%%%%%%%%
\subsection{Uncertainties}\label{s:eidunc}

The measurement of the identification efficiency is accompanied by statistical
and systematic uncertainties. The identification efficiencies, the statistical
uncertainties, and the systematic uncertainties have been quoted in
Section~\ref{s:meffs}, they are $\mathbf{0.870\pm 0.017 \pm 0.031}$ and
$\mathbf{0.784 \pm 0.019 \pm 0.020}$ for Medium and Tight respectively. Thus
the statistical uncertainty is approximately $2\%$ for Medium and $2.4\%$ for
Tight, and the systematic uncertainty is higher, approximately $3.6\%$ and $2.6\%$
respectively. This section discusses in detail the treatment of the statistical
and systematic uncertainties, which are listed in~Table~\ref{t:alluncertainties}
at the end of this section. 

\paragraph{Statistical Uncertainties} According to Formula~\ref{eid:efff}, the
efficiency is measured according to the formula


$$
	\epsilon = \frac{P-B_P}{N - \overline{T} \times \frac{\overline{N}_>}{\overline{T}_>}}
$$

where

\begin{itemize}
	\item $P$ is the number of electrons that pass Medium or Tight.

	\item $B_P$ is background contamination due to fake electrons in $P$.

	\item $N$ is the set of reconstructed electron candidates, and $\overline{N}_>$
	      the corresponding quantity in the background-dominated region minus signal
	      contamination.

	\item $\overline{T}$ is the set of antiloose electrons minus signal
	      contamination, and $\overline{T}_>$ the corresponding quantity in the
	      background-dominated region.

\end{itemize}

Since $N$ contains $P$, and $\overline{N}_>$ contains $\overline{T}_>$, the
quantities in the formula are not all independent. We may remove the
correlation between $N$ and $P$ by writing $N = P + F$, where $F$ is the set of
electrons that fail a particular ID point. Then

$$
	\epsilon = \frac{P-B_P}{P + F - \overline{T} \times \frac{\overline{N}_>}{\overline{T}_>}}
$$


The correlation between $\overline{N}_>$ and $\overline{T}_>$ remains, and
moreover $F$ and $\overline{T}$ are also correlated, because in the Medium case
or in the Tight case, $F$ represents electrons failing Medium or Tight
respectively, and since $\overline{T}$ represents electrons failing Loose
(minus signal contamination), in each case $\overline{T}$ is a subset of $F$
and there is accordingly a correlation.

In order to remove all the correlations and write the efficiency completely in
terms of statistically independent quantities we will first multiply both the
numerator and the denominator by $\overline{T}_>$, to write

$$
	\epsilon = \frac{(P-B_P)\overline{T}_>}{P\overline{T}_> + F\overline{T}_> - \overline{T}\times \overline{N}_>}.
$$

Then we will add and subtract $\overline{T} \times \overline{T}_>$, to have


\begin{equation*}
	\begin{split}
		\epsilon & = \frac{(P-B_P)\overline{T}_>}{P\overline{T}_> + F\overline{T}_> - \overline{T}\times \overline{T}_>
			+ \overline{T}\times \overline{T}_> - \overline{T}\times \overline{N}_>}  \\
		&  =  \frac{(P-B_P)\overline{T}_>}{P\overline{T}_> + (F - \overline{T})\overline{T}_>
			- (\overline{N}_> - \overline{T}_>)\overline{T}}
	\end{split}
\end{equation*}

The difference $F - \overline{T}$ represents the set of electrons that fail
Medium or Tight but pass the Loose identification, and the difference
$\overline{N}_> - \overline{T}_>$ represents the set of electrons that pass the
Loose identification. If we treat each of the differences as a single term, and
set $S = F - \overline{T}$ and $\overline{R}_> = \overline{N}_> -
	\overline{T}_>$ respectively, the efficiency becomes

\begin{equation}\label{eq:effco}
	\epsilon = \frac{(P-B_P)\overline{T}_>}{P\overline{T}_> + S\overline{T}_>
		- \overline{R}_>\times \overline{T}}
\end{equation}

which is now a function of six independent quantities, $\epsilon=\epsilon(P,
	B_P, \overline{T}_>, S, \overline{R}_>, T)$. The statistical uncertainty of the
efficiency then follows the standard error propagation formula,

\begin{equation}\label{eq:statprop}
	\Delta \epsilon^2 = \bigg(\frac{\partial \epsilon}{\partial P}\bigg)^2\Delta P^2 + \cdots +
	\bigg(\frac{\partial \epsilon}{\partial T}\bigg)^2\Delta T^2
\end{equation}


Let $A$ denote the numerator in Formula~\ref{eq:effco} and $B$ the denominator.
The terms in the formula above are then

$$\frac{\partial \epsilon}{\partial P} = \frac{B \overline{T}_> - A \overline{T}_> }{B^2},
	\quad
	\frac{\partial \epsilon}{\partial B_P} = \frac{- B \overline{T}_> }{B^2},
	\quad
	\frac{\partial \epsilon}{\partial \overline{T}_>} = \frac{B (P - B_P) - A(P + S) }{B^2},
$$

$$\frac{\partial \epsilon}{\partial S} = \frac{- A \overline{T}_> }{B^2},
	\quad
	\frac{\partial \epsilon}{\partial \overline{R}_>} = \frac{AT }{B^2},
	\quad
	\frac{\partial \epsilon}{\partial \overline{T}} = \frac{A \overline{R}_> }{B^2}.
$$

Since $P$ and $S$ are the only terms in the signal region not used for
background estimation, the statistical uncertainty of the efficiency is taken
from the contributions of these two terms. For both operating points, the
contribution from $S$ is the major one; the contribution from $P$ is small ($<
	0.5\%$ from the total $2\%$ for Medium and $2.4\%$ for Tight).

The contributions to the uncertainty from other terms, which are used for
background estimation, are taken as contributions to the total systematic
uncertainty.

%%%%%%%%%%%%%%%%%%%%%%%%%%%%%%%%%%%%%%%%%%%%%%%%%%%%%%%%%%%%%%%%%%%%%%%%%%%%%%%%%%%%%%
%%%%%%%%%%%%%%%%%%%%%%%%%%%%%%%%%%%%%%%%%%%%%%%%%%%%%%%%%%%%%%%%%%%%%%%%%%%%%%%%%%%%%%
%%%%%%%%%%%%%%%%%%%%%%%%%%%%%%%%%%%%%%%%%%%%%%%%%%%%%%%%%%%%%%%%%%%%%%%%%%%%%%%%%%%%%%

\paragraph{Systematic Uncertainties} Contributions from different sources to
the total systematic uncertainty ($3.6\%$ for Medium and $2.5\%$ for Tight),
which are discussed below, are added in quadrature.

\begin{itemize}

	\item The variation of the signal region. In addition to defining the signal
	      region at $\leq 60$ GeV, we may define it at $\leq 50$ or $\leq 80$ GeV, the
	      asymmetry because of the fact that signal distributions on both sides of the
	      point $60$ GeV are not equal in equal intervals. The contribution to the total
	      systematic uncertainty is $\mathbf{0.022}$ (approximately $2.5\%$)
	      for Medium and $\mathbf{0.010}$ (approximately $1.3\%$) for Tight.


	\item The variation of the term $B_P$, which is taken from simulation and
	      represents background contamination in $P$. A $50\%$ variation is used for a
	      conservative estimate of the contribution of this term, which has been seen to
	      be negligible for both Medium and Tight ($<0.2\%$ in both cases).


	\item The simultaneous variations, either up or down, of the signal
	      contaminations in $T$ and $T_>$, the subtractions of which from both terms
	      give $\overline{T}$ and $\overline{T}_>$. A $25\%$ variation is used for a
	      conservative estimate of these contributions, which are $\mathbf{0.017}$ (approximately
	      $2\%$) in Medium and $\mathbf{0.015}$ (approximately $1.9\%$) in Tight.

	\item The change of the template $T$, from the distribution of antiloose
	      electrons to the distribution of antiloose electrons in events with exactly 2
	      $b$-jets. The contributions to the total systematic uncertainty are
	      $\mathbf{0.008}$ (approximately $0.9\%$) for Medium and $\mathbf{0.007}$
	      (approximately $0.9\%$) for Tight.


	\item The statistical uncertainties from the counting of $\overline{T}_>$,
	      $\overline{R}_>$, and $\overline{T}$ in Formula~\ref{eq:effco}. They are
	      $0.002$ (approximately $0.2\%$), $0.008$ (approximately $0.9\%$), and $0.002$
	      (approximately $0.2\%$) respectively for Medium and $0.001$ (approximately
	      $0.1\%$), $0.006$ (approximately $0.8\%$), and $0.001$ (approximately $0.1\%$)
	      respectively for Tight.


\end{itemize}


\renewcommand{\arraystretch}{1.15}
\begin{table}[H]
	\centering
	\begin{tabularx}{0.6\textwidth}{| c | *{2}{Y|} }
		\cline{2-3}
		\multicolumn{1}{c |}{} & MEDIUM & TIGHT   \\
		\hline\hline
		\toprule
				& \multicolumn{2}{ c |}{ Systematic Uncertainties }\\
		\hline
		$S$                    & $0.022$  & $ 0.010$  \\
		\hline
		$B_P$                  & $0.002$ & $ 0.000$ \\
		\hline
		$T$                    & $0.017$  & $ 0.015$  \\
		\hline
		& \multicolumn{2}{ c |}{ Statistical Uncertainties }\\
		\hline
		                      & $0.010$   & $ 0.011$   \\
		\hline
		\toprule
	\end{tabularx}
	\caption{The statistical and systematic uncertainties for the Medium and Tight 
					 operating point.}
	\label{t:alluncertainties}
\end{table}
\renewcommand{\arraystretch}{1}




%%%%%%%%%%%%%%%%%%%%%%%%%%%%%%%%%%%%%%%%%%%%%%%%%%%%%%%%%%%%%%%%%%%%%%%%%%%%%%%%%%%%%%%%%%%%
%%%%%%%%%%%%%%%%%%%%%%%%%%%%%%%%%%%%%%%%%%%%%%%%%%%%%%%%%%%%%%%%%%%%%%%%%%%%%%%%%%%%%%%%%%%%
%%%%%%%%%%%%%%%%%%%%%%%%%%%%%%%%%%%%%%%%%%%%%%%%%%%%%%%%%%%%%%%%%%%%%%%%%%%%%%%%%%%%%%%%%%%%
%%%%%%%%%%%%%%%%%%%%%%%%%%%%%%%%%%%%%%%%%%%%%%%%%%%%%%%%%%%%%%%%%%%%%%%%%%%%%%%%%%%%%%%%%%%%

\subsection{Efficiencies as Functions of the Properties of the Electron and of the
	Overlapping Jet}

In addition to the integrated efficiencies, the efficiencies as functions of
the properties of the electron and of the overlapping jet are also measured.
The measurements include the following variables (Figure~\ref{f:bins01}
and~\ref{f:bins02}).

\begin{itemize}

	\item $p_T$ of the probe, in five bins: 30-60 GeV, 60-80 GeV, 80-110 GeV,
	      110-140 GeV, and $> 140$ GeV.

	\item $\abs{\eta}$ of the probe, in five bins: 0.0-0.3, 0.3-0.6, 0.6-0.9,
	      0.9-1.3, and $> 1.3$.

	\item $\Delta R $ between the probe and the closest overlapping jet, in five
	      bins: 0.0-0.15, 0.15-0.19, 0.19-0.23, 0.23-0.27, and 0.27-0.4.

	\item $p_T$ of the closest overlapping jet, in five bins: 150-220 GeV, 220-280
	      GeV, 280-340 GeV, 340-400 GeV, and 400-500 GeV.

\end{itemize}

\begin{figure}[H]
	\begin{subfigure}{0.5\textwidth}
		\includegraphics[width=8cm]{figures/nvariables_pt40_probes_cr1_m0_pt100_mar05_loose_real_loose_fake_dr_el_cjet_dr_el_cjet_mar22}
		\label{bfptprobe}
		\caption{$\Delta R $ between the probe and the closest jet}
	\end{subfigure}
	\begin{subfigure}{0.5\textwidth}
		\includegraphics[width=8cm]{figures/nvariables_pt40_probes_cr1_m0_pt100_mar05_loose_real_loose_fake_el_eta_el_eta_mar22}
		\caption{$\abs{\eta}$ of the probe}
	\end{subfigure}

	\centering

	\caption{The distributions of $\Delta R $ between the probe and the closest
		overlapping jet and $\abs{\eta}$ of the probe.}

	\label{f:bins01}
\end{figure}

\begin{figure}
	\begin{subfigure}{0.5\textwidth}
		\includegraphics[width=8cm]{figures/nvariables_pt40_probes_cr1_m0_pt100_mar05_loose_real_loose_fake_el_pt_el_pt_mar22}
		\caption{$p_T$ of the probe}
	\end{subfigure}
	\begin{subfigure}{0.5\textwidth}
		\includegraphics[width=8cm]{figures/nvariables_pt40_probes_cr1_m0_pt100_mar05_loose_real_loose_fake_pt_c_jet_pt_c_jet_mar22}
		\caption{$p_T$ of the closest overlapping jet}
	\end{subfigure}

	\centering

	\caption{The distributions of $p_T$ of the probe and $p_T$ of the closest
		overlapping jet.}

	\label{f:bins02}
\end{figure}

The efficiencies, for the Medium and Tight operating points, as a function of
the $p_T$ and $\abs{\eta}$ of the probes are shown in Figure~\ref{f:binspt01}.
Also shown are the efficiencies for standard electrons
(~\cite{atlaselcid,eleffme}), which, as can be seen, agree with those for
in-jet electrons within the error bars. As a function of the $p_T$ of the
probe, the efficiencies increase as $p_T$ increases. On the other hand, no
obvious dependency is seen in the case of $\abs{\eta}$.


The efficiencies as a function of the $\Delta R$ between the probe and the
closest overlapping jet, and as a function of the $p_T$ shown in
Figure~\ref{f:binspt02}. In the latter case the efficiencies are higher for
lower $p_T$.


\begin{figure}[H]
	\begin{subfigure}{0.5\textwidth}
		\includegraphics[width=8cm]{figures/sep04_bins_probe_pt_medium_tight_dt_mc_nov26}
	\end{subfigure}
	\begin{subfigure}{0.5\textwidth}
		\includegraphics[width=8cm]{figures/sep04_bins_probe_eta_medium_tight_dt_mc_nov26}
	\end{subfigure}

	\centering
	\caption{The efficiencies in $p_T$ of the probe as well as in $\abs{\eta}$ of the
		probe. Also shown are the efficiencies for standard electrons and the associated 
		uncertainties (which are very small and therefore are barely visible).}
	\label{f:binspt01}
\end{figure}

\begin{figure}[H]
	\begin{subfigure}{0.5\textwidth}
		\includegraphics[width=8cm]{figures/sep04_bins_dr_medium_tight_dt_mc}
	\end{subfigure}
	\begin{subfigure}{0.5\textwidth}
		\includegraphics[width=8cm]{figures/sep04_bins_jetpt_medium_tight_dt_mc}
	\end{subfigure}

	\centering
	\caption{The efficiencies in $\Delta R$ between the probe and the closest jet,
		as well as in $p_T$ of the closest overlapping jet.}
	\label{f:binspt02}
\end{figure}


\section{Conclusions}\label{s:eidcon}

This chapter describes the work to measure the identification efficiencies for
in-jet electrons. It was the first attempt to perform such a measurement since
Run 2 began, and the first ever using dilepton $t\tbar$ events. The measurement
used the data collected in the period 2015-2016, at $13$ TeV center-of-mass and
totaled $36.5~\text{fb}^{-1}$ in integrated luminosity. A sample of electrons
for the measurements was obtained by selecting boosted $t\bar{t}$ dilepton
($e\mu$) events. Background estimations used both simulations and data, and the
efficiencies were evaluated iteratively. The efficiencies were measured for the
Medium and Tight operating points, both on data and simulation. The
efficiencies as functions of the properties of the electrons and of the
overlapping jets also measured. In all of the results, the efficiencies
predicted by simulation agree with those obtained from the measurements on
data.


