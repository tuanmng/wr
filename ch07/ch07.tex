\chapter{IN-JET ELECTRON IDENTIFICATION EFFICIENCIES}\label{c:eid}

In early 2015 the LHC restarted after two years of shutdown, beginning what is
referred to as Run 2. The new centre-of-mass energy was $13$ TeV, in place of
the previous $8$ TeV. The higher energy opens up unexplored parameter space and
allows further probe of supersymmetry (SUSY) and other
beyond-the-Standard-Model processes~\cite{multib-c7, stop-c7, ttresonance-c7,
	vlq-c7}. At ATLAS, many SUSY searches involve supersymmetric particles that
decay into the Standard Model top quarks and we expect, with higher
centre-of-mass energy, higher sensitivity to more massive supersymmetric
particles and consequently an increase in the production of high $p_T$ top
quarks. Since the top quarks also decay --- essentially most of the time ---
into a $W$ boson and a $b$ quark, we in turn expect boosted decay topology, in
other words the daughter particles of the top quarks, which include the
daughter particles of the $W$ boson and the $b$ quark, tend to stay close to
each other (Figure~\ref{f:boosttop}).


This chapter describes the work to measure the identification efficiencies for
electrons that are found inside $\Delta R = 0.4$ of high $p_T$ jets, which will
also be called in-jet electrons in the chapter, using $t\tbar$ events. Prior to
the work in this chapter, there was no attempt to measure the identification
efficiencies for in-jet electrons in a $t\tbar$ topology. The chapter will be
organized as follows. In section~\ref{s:eidmot} we motivate the need for the
measurement of the identification efficiencies for in-jet electrons.
Section~\ref{s:eidmet} describes the method used to perform the measurements.
Section~\ref{s:eideff} presents the measured efficiencies for in-jet electrons
and section~\ref{s:eidcon} gives some conclusions.

The data used for this chapter was collected in the period 2015-2016 at $13$
TeV center-of-mass and correponded to an integrated luminosity of
$36.5~\text{fb}^{-1}$.

\section{Motivation}\label{s:eidmot}

Prior to Run 2, ATLAS center-of-mass energy 7-8 GeV allows limited sensitivity
to high mass resonances. Because many beyond-Standard-Model particles are
predicted to decay into the Standard Model top quarks, the limited sensitivity
reduces the chance in which we could expect boosted top quark decays. Such
decays, however, are expected to become significant as the centre-of-mass of
the LHC reached $13$ TeV starting from Run 2. In a boosted top quark decay
scenario, the produced particles, which in this case are the daughters of the
$W$ and the $b$ quark that come from the top quark, are found close to each
other~\cite{hadronic01-ch7, hadronic02-ch7, bdis-figs}.
Figure~\ref{f:drwbtoppt} shows the angular distance $\Delta R$
(Formula~\ref{eq:angulardr}) between the $W$'s and the $b$-quarks as a function
of the top $p_T$, in the context of a hypothetical particle $Z'$ at mass
$m_{Z'}=1.6$ TeV~\cite{bdis-figs} that decays into a $t\bar{t}$ pair. Also
shown in the same figure is the separation between the light quarks of the
subsequent hadronic decay of the $W$ boson. As can be seen, the angular
distance decreases as the top quark $p_T$ increases, and at high top quark
$p_T$ a non-negligible fraction of the distances becomes very small.


% At ATLAS, prior to Run 2, electrons found outside $\Delta R = 0.4$ of jets were
% used by most analyses. Indeed, although there were some attempts to select
% signal electrons inside jets~\cite{?}, at 7-8 TeV such electrons can hardly be
% expected; objects inside jets that are identified as electrons are mostly
% background electrons that are either hadrons faking jets or real electrons
% coming from heavy-flavour jet decays. As a result, most analyses rejected
% electrons inside $\Delta R = 0.4$ of jets and only worked with electrons
% outside jets, of which different efficiencies, among them identification
% efficiencies, are measured and provided by the ATLAS Egamma
% group~\cite{atlaselcid}.



\vspace{3mm}

\begin{figure}[H]
	\includegraphics[width=12cm]{figures/boost}
	\centering

	\caption{An illustration of low $p_T$ top quark decay (left) and boosted top
		decay (right) of a high $p_T$ top quark. In the case of high $p_T$ top quark
		decay the daughter particles of the top quark, which include the daughter
		particles of the $W$ and the $b$ quark, are expected to be found close to each
		other~\cite{boostedtop-fig}.}

	\label{f:boosttop}
\end{figure}

\vspace{3mm}

Leptonic boosted top quark decay is also an important
channel in searches for beyond-Standard-Model particles that decay into the
Standard-Model top quarks. Table~\ref{t:injetfraction} shows the fraction of
in-jet electrons over signal electrons as a function of the top quark $p_T$, at
truth-level. The measurement used PowhegPythia $t\bar{t}$ events simulated at
$13$ TeV centre-of-mass energy (Chapter~\ref{c:susys}, Section~\ref{mbdatasm}),
where dilepton events consisting of a muon and an electron were selected. The
selections made use of the $p_T$-dependent overlap removal

\begin{equation}\label{c7overlapr}
	\Delta R <  \text{min (0.4, 0.04 + 10 GeV} / p_T)
\end{equation}

where the overlap removal is also required to keep the overlapping $b$-jets
(Chapter~\ref{c:susys}, Section~\ref{mbobjs}). As is shown, more and more
electrons are found inside jets as the top quark $p_T$ increases. The number of
in-jet electrons becomes quite significant from $500$ GeV, being approximately
$25\%$ there and reaching nearly $40\%$ at $650$ GeV. If the top quark $p_T$ is
allowed to go up to $1$ TeV, the figure is $64\%$. This result supports the
fact that different ATLAS analyses searching for heavy beyond-Standard-Model
particles decaying into top quarks was able to increase signal acceptances
considerably when in-jet electrons were selected.

This chapter develops a method and performs the initial measurements for the
identification efficiencies of electrons found inside $\Delta R=0.4$ of jets.
The measurements for electrons outside jets are done by the ATLAS Egamma
group~\cite{atlaselcid, eleffme}.

\begin{figure}[H]

	\begin{subfigure}{0.5\textwidth}
		\includegraphics[width=7.5cm]{figures/fig_01a}
		\caption{$t\to Wb$}
		\label{drwbtoppta}
	\end{subfigure}
	\begin{subfigure}{0.5\textwidth}
		\includegraphics[width=7.5cm]{figures/fig_01b}
		\caption{$W\to q\bar{q}$}
		\label{drwbtopptb}
	\end{subfigure}

	\centering
	\caption{(\ref{drwbtoppta})~The angular distance $\Delta R$ between
		the $W$'s and the $b$ quarks as a function of the top quark $p_T$ simulated
		PYTHIA~\cite{sjostrand:ch7}, in the context of a hypothetical particle $Z'$
		($m_{Z'}=1.6$ TeV) that decays into a $t\bar{t}$ pair. At high top quark $p_T$
		a non-negligible fraction of the distances is seen to be very
		small.~(\ref{drwbtopptb})~The angular distance between two light quarks from
		$t\to Wb$ decay as a function of the $p_T$ of the $W$ boson~\cite{bdis-figs}.}
	\label{f:drwbtoppt}
\end{figure}


\renewcommand{\arraystretch}{1.15}
\begin{table}
	\centering
	\begin{tabular}{||c c||}
		\hline
		Top quark $p_T$ (GeV) & Fraction  \\ [0.5ex]
		\hline\hline
		\toprule
		$\leq 300$            & $ 8.4\%$  \\
		\hline
		$\leq 425$            & $ 17.2\%$ \\
		\hline
		$\leq 500$            & $ 24.0\%$ \\
		\hline
		$ \leq 650$           & $ 39.0\%$ \\
		\hline
		$ \leq 750$           & $ 49.0\%$ \\
		\hline
		$ \leq 800$           & $ 53.0\%$ \\
		\hline
		$ \leq 900$           & $ 59.0\%$ \\
		\hline
		$ \leq 1000$          & $ 64.0\%$ \\
		\hline
	\end{tabular}

	\caption{The fraction of in-jet electrons over the number of signal
		electrons, both at truth-level, as a function of the top quark $p_T$. The
		fraction increases and becomes very significant at high top quark $p_T$.}

	\label{t:injetfraction}
\end{table}
\renewcommand{\arraystretch}{1.0}




%%%%%%%%%%%%%%%%%%%%%%%%%%%%%%%%%%%%%%%%%%%%%%%%%%%%%%%%%%%%
\section{Method}\label{s:eidmet}

The method followed to measure the identification efficiencies for in-jet
electrons is discussed in detail in Section~\ref{ss:eidtp},~\ref{ss:eidsp}, and
~\ref{ss:eidsr}. Background estimations will be described in
Section~\ref{s:eidbe}.

\subsection{Boosted Dilepton $e\mu$ Events}\label{ss:eidtp}

In order to measure the identification efficiencies for in-jet electrons, a
sample of reconstructed electrons (Chapter~\ref{c:susys}, Section~\ref{mbobjs})
inside high-$p_T$ jets was obtained by selecting boosted $t\bar{t}$ dilepton
($e\mu$) events. This is expected to result in not only a very pure $t\tbar$
sample, but also a topology close to that of many SUSY and other
beyond-Standard-Model searches. In contrast, the standard method for measuring
electron identification efficiencies, the tag-and-probe method supported by the
Egamma group at ATLAS~\cite{eleffme}, makes use of $Z\to e^+e^-$ events for
high-energy electrons ($E_{\text{T}} > 10$ GeV). Even though a clean sample of
electrons may be obtained in a relatively straightforward way by selecting
events around the $Z$ mass peak, we expect, if electrons inside high $p_T$ jets
are required, a sample unrepresentative of events with a boosted topology and
limited in statistics.

Given sample of reconstructed electrons, the efficiency at a particular
identification operating point (section~\ref{ss:elop}) is defined by the ratio

$$
	\text{ID efficiency} = \frac{\text{The number of identified
			electrons}}{\text{The number of reconstructed electrons}}
$$

Both the numerator and the denominator are contaminated with background
electrons, which will need to be estimated carefully (Section~\ref{s:eidbe}),
particularly because background electrons are expected to reside primarily
inside jets.

%%%%%%%%%%%%%%%%%%%%%%%%%%%%%%%%%%%%%%%%%%%%%%%%%%%%%%%%%%%%
\subsection{Data and Monte Carlo Samples}\label{ss:eidsp}

The data used for this chapter was collected in the period 2015-2016 and
corresponds to an integrated luminosity of $36.5~\text{fb}^{-1}$. The following
simulation samples, all at $13$~TeV centre-of-mass energy, are used
(Chapter~\ref{c:susys}, Section~\ref{mbdatasm}):

\begin{itemize}

	\item $t\bar{t}$ events from the Powheg+Pythia generator. As a hard muon will
	      be required in the sample in which the identification efficiencies are measured
	      (Section~\ref{ss:eidsr}), these events naturally partition into either a
	      dileptonic set ($e\mu$) when truth-level electrons are present inside
	      high-$p_T$ jets, or a semileptonic set otherwise. The latter, with jets from
	      the fully hadronic decay of one of the tops constituting a source of background
	      electrons, is expected to be a dominant background.

	\item $W$+jets, which will be used as a background. As the $W$ boson may
	      produce a hard muon, the presence of jets make these events a source of
	      background events to signal dilepton $e\mu$ $t\tbar$ events.

	\item Single top events, which include the $Wt$ production as well as the
	      $s$-channel and $t$-channel productions. The $Wt$ production is treated as a
	      source of signal electrons, since it contains a pair of $W$ bosons that can
	      decay to a prompt $e\mu$ pair, whereas the remaining two productions each
	      contains only one $W$ boson and as a result cannot produce a prompt $e\mu$
	      pair.

\end{itemize}
%%%%%%%%%%%%%%%%%%%%%%%%%%%%%%%%%%%%%%%%%%%%%%%%%%%%%%%%%%%%
\subsection{Signal Region}\label{ss:eidsr}

The kinematic region in which the measurement of the identification
efficiencies is performed is called the signal region. It is defined after the
following preliminary selections, which are called the pre-selection cuts and
aimed at isolating $t\tbar$ $e\mu$ events, are applied.

\paragraph{Pre-selection}

\begin{itemize}
	\item One primary vertex

	\item Muon trigger. The following triggers were used for the periods $2015$
	      and $2016$:

	      \begin{itemize}
		      \item $2015$: \texttt{HLT_mu26_imedium || HLT_mu40}
		      \item $2016$:  \texttt{HLT_mu26_ivarmedium || HLT_mu50}
	      \end{itemize}

	\item $p_T$-dependent overlap removal, where the overlapping $b$-jets are kept
	      (Formula~\ref{c7overlapr}).

	\item Events with bad or cosmic muons are removed. Highly energetic jets could
	      reach the muon spectrometer and create hits in the latter, or jet tracks in the
	      inner detector could be erroneously matched to muon spectrometer segments, both
	      of which cases are sources of bad muons. Events with these muon candidates,
	      along with those with muons coming from cosmic rays, are not accepted.

	\item Exactly one tagged muon and $\geq 1$ electrons inside jets are required
	      for each event, where

	      \begin{itemize}

		      \item The muon must have $p_T > 30$ GeV, $d_0 / \sigma(d_0) < 3.0$, and
		            $z_0 < 0.5$ in terms of the transverse impact paramater and the longitudinal
		            impact parameter. It must also have $\text{ptvarcone}30 / p_T < 0.06$, and be
		            trigger-matched.

		      \item The electrons must have $p_T \geq 30$ GeV, which is a common cut in most
		            analyses where in-jet electrons are used, and must overlap within $\Delta R <
			            0.4$ with some jets. Only the leading $p_T$ electron will be used, even though
		            there may be more than one electron present in the event.

	      \end{itemize}

	\item $\geq 1$ $b$-tagged jet, instead of exactly $2$ $b$-tagged jets as is
	      usually expected in $t\tbar$ events, since we are selecting events with
	      electrons inside jets and the $b$-tagging efficiency may suffer because the
	      tracks of the electron, which is expected to originate from the interaction
	      point, may confuse the $b$-tagging algorithm.

\end{itemize}

These cuts result in a set of 3183 events with one hard muon and at least one
electron found inside some jet. In the following, we discuss several variables
that were found to be discriminating, along with their distribution plots. The
prominent source of background events is semileptonic events, whereas $W$+jets
and single top $s$-channel and $t$-channel constitute two small sources of
background, predicted by simulations to be $315.483$ and $19.120$ respectively.

\begin{itemize}

	\item The mass of the large radius jet that overlaps with the probe electron,
	      shown in Figure~\ref{f:premrjet}. The large radius jet is reclustered from the
	      small radius jets present in the events (Chapter~\ref{c:susys},
	      Section~\ref{mbobjs}), and accordingly in semileptonic events it is expect to be
	      more massive, as it picks up the masses of the jets from the hadronic decay of
	      one of the tops. In dileptonic events, on the other hand, there are fewer jets
	      due to leptonic decays of both of the top quarks, and in addition the
	      neutrio that accompanies the electron may reduce the visible mass of the
	      reconstructed large radius jet. As is shown in the figure, the higher mass
	      region is dominated by background events.


	\item The number of jets, which is shown in Figure~\ref{f:prenjets}. Three jets
	      are expected from a fully hadronic decaying top quark, as compared to only
	      one jet from a semileptonic decay, and as a result semileptonic events, in
	      which one top quark decays hadronically and one decays semileptonically, is
	      expected to have more jets than dileptonic events, where both jets decay
	      hadronically. In the figure, the semileptonic distribution is seen to be higher
	      everywhere.


	\item The sum of the transverse momenta of all jets, shown in
	      Figure~\ref{f:prehtjet}. As above, a larger number of jets is expected in
	      semileptonic events due to the fully hadronic decay of one of the top quarks,
	      and in dileptonic events fewer jets are expected because of leptonic decays of
	      both of the top quarks. Consequently a sum over all transverse momenta of the
	      jets is expected to lead to a discriminating distribution. As is seen in the
	      figure, the semileptonic distribution is higher everywhere.


	\item The transverse momenta of the jet closest to the probe, which is shown in
	      Figure~\ref{f:preptcjet}. This variable allows the removal of low $p_T$ jets
	      overlapping with background electrons.


	\item The fraction of the transverse momentum of the probe electron over that
	      of the closest jet (Figure~\ref{f:preptfractioncjet}). We expect real electrons
	      from the $W$'s produced from the top quarks to have higher $p_T$ than
	      background electrons. In the figure, the low $p_T$ region can be seen to be
	      dominated by semileptonic events.

\end{itemize}

\begin{figure}[H]
	\includegraphics[width=10cm]{figures/nvariables_pt40_probes_pre_m0_pt100_mar05_loose_real_loose_fake_m_rjet_overlap_el_mar22}
	\centering
	\caption{$m_{\text{rjet}}^{\text{el}}$. The semileptonic contribution is higher everywhere, especially on the right side of the distribution where
		there is little signal contamination.}
	\label{f:premrjet}
\end{figure}

\begin{figure}[H]
	\includegraphics[width=10cm]{figures/nvariables_pt40_probes_pre_m0_pt100_mar05_loose_real_loose_fake_njets_njets_mar22}
	\centering
	\caption{The number of jets. The semileptonic contribution is higher because of the hadronic decay of one of the tops.}
	\label{f:prenjets}
\end{figure}

\begin{figure}[H]
	\includegraphics[width=10cm]{figures/nvariables_pt40_probes_pre_m0_pt100_mar05_loose_real_loose_fake_ht_jet_ht_jet_mar22}
	\centering
	\caption{The sum of the transverse momenta of all jets.}
	\label{f:prehtjet}
\end{figure}


\begin{figure}[H]
	\includegraphics[width=10cm]{figures/nvariables_pt40_probes_pre_m0_pt100_mar05_loose_real_loose_fake_pt_c_jet_pt_c_jet_mar22}
	\centering
	\caption{$p_T$ of the jet closest to the probe electron. }
	\label{f:preptcjet}
\end{figure}


\begin{figure}[H]
	\includegraphics[width=10cm]{figures/nvariables_pt40_probes_pre_m0_pt100_mar05_loose_real_loose_fake_pt_fraction_c_jet_pt_fraction_c_jet_mar22}
	\centering
	\caption{Fraction of the $p_T$ of the probe electron over that of the closest jet. The lower $p_T$ region is dominated by semileptonic events.}
	\label{f:preptfractioncjet}
\end{figure}

\paragraph{Further cuts to arrive at the signal region}\label{p:eidfurthercuts}

Of all the discriminating variables shown above, $m_{\text{rjet}}^{\text{el}}$
seems to be the most discriminating variable. In addition, its distribution
shows two distinct regions, one abundant in signal electrons and one largely
dominated by background electrons. As will be discussed later in the chapter,
the region $< 60$ GeV will define the signal region where the identification
efficiencies are measured, while the region $> 60$ GeV will define the control
region for background estimation. With this in mind, we decided to apply cuts
on the other discriminating variables to further remove the undesired
background, while leaving $m_{\text{rjet}}^{\text{el}}$ untouched.

The cuts are as follows:

\begin{itemize}
	\item MET $> 25$ GeV, to ensure that the QCD multi-jet background is negligible.

	\item The number of jets $< 5$ and sum of $p_T$ of jets $< 700$ GeV, to remove
	      semileptonic events (Figure~\ref{f:prenjets} and~\ref{f:prehtjet}).


	\item $p_T$ of jet closest to the probe is between $150$ GeV and $500$ GeV, to
	      remove semileptonic events (Figure~\ref{f:preptcjet}) and at the
	      same time make sure that boosted $t\tbar$ dilepton events are selected.


	\item $p_T(\text{probe})  / p_T(\text{closest jet}) > 0.16 $ (Figure~\ref{f:preptfractioncjet}).

\end{itemize}

The resulting distrution $m_{\text{rjet}}^{\text{el}}$ is shown in Figure~\ref{f:crmrjet}.

\begin{figure}[H]
	\includegraphics[width=10cm]{figures/n_pt40_probes_cr1_m0_pt100_mar05_loose_real_loose_fake_m_rjet_overlap_el_mar22}
	\centering

	\caption{$m_{\text{rjet}}^{\text{el}}$ after the pre-selection cuts. The
		region $< 60$ GeV will define the signal region, and the region $\geq 60$ GeV
		will define the control region for background estimation.}

	\label{f:crmrjet}
\end{figure}

%%%%%%%%%%%%%%%%%%%%%%%%%%%%%%%%%%%%%%%%%%%%%%%%%%%%%%%%%%%%
\subsection{Background Estimation}\label{s:eidbe}

The identification efficiency for electrons inside jets depends on the
particular operating point (Loose, Medium, or Tight) at which the measurement
is carried out. Such an efficiency, which will be denoted $\epsilon$, is the
ratio of a numerator and a denominator (Section~\ref{s:eidmet}), both of which
are expected to be contaminated with background electrons that need to be
estimated. If $P$ denotes the number of electron candidates passing a
particular ID operating point, $B_P$ the number of background electrons passing
the operating point, $N$ the total number of reconstructed electron candidates
in the sample, and $B_N$ the number of background electrons present in the
sample, the efficiency $\epsilon$ may be written as

\begin{equation}\label{eqn:effo}
	\epsilon = \frac{P-B_P}{N-B_N}
\end{equation}

Because analyses using in-jet electrons all use the Medium or Tight operating
point, these are the only two points that will be measured in this chapter.
Accordingly, a Medium or Tight ID selection will be applied on the sample
representing the denominator, giving in each case the required numerator.
Background estimations will consist of estimating the term $B_P$ separately for
Medium and Tight in the numerator, and estimating the common term $B_N$ in the
denominator.

\paragraph{Estimating $B_P$} Since we expect background electrons to rarely
pass the Medium or Tight ID points, we expect in turn the term $B_P$ to be very
small in either case. Thus $B_P$ is taken directly from simulation, and the
measurements are not expected to be affected significantly.

In Figure~\ref{f:eidmedium} is shown the $m_{\text{rjet}}^{\text{el}}$
distributions for electrons that pass the Medium and Tight selections. It is
obtained by applying a Medium or Tight ID selection in addition to the
selections that define the signal region (Section~\ref{p:eidfurthercuts}). The
number of background electrons predicted by the simulation can be seen to be
indeed small in each case, accouting for only $0.3\%$ of the total number in
the Medium case and $0.1\%$ in the Tight case.


\begin{figure}[H]
	\includegraphics[width=7.5cm]{figures/pt40_probes_cr1_medium_m0_pt100_mar05_medium_real_medium_fake_m_rjet_overlap_el_mar22}
	\includegraphics[width=7.5cm]{figures/pt40_probes_cr1_tight_m0_pt100_mar05_tight_real_tight_fake_m_rjet_overlap_el_mar22}

	\centering

	\caption{The distribution of $m_{\text{rjet}}^{\text{el}}$ for electrons passing
		the Medium (left) and Tight (right) operating points. Background electrons figure $0.3\%$ and
		$0.1\%$ respectively.}

	\label{f:eidmedium}

\end{figure}

%%%%%%%%%%%%%%%%%%%%%%%%%%%%%%%%%%%%%%%%%%%%%%%%%%%

\paragraph{Estimating $B_N$} The term $B_N$ represents background contamination
from fake electrons found in $N$ (Formula~\ref{eqn:effo}). Since $N$ contains
only reconstructed in-jet electrons with no ID applied, estimating $B_N$ is
expected to be the most challenging part of the measurements.

The method employed for estimating $B_N$ in the following makes use the set of
electrons that fail the Loose ID selection, which will be called antiloose
electrons hereafter. These electrons are made up of two parts, one in the
signal region ($\leq 60$ GeV) and one in the background-dominated region ($>
	60$ GeV, Figure~\ref{f:crmrjet}). The part in the background-dominated region
will be used to obtain a normalization factor, which will then be applied on
the part in the signal region to estimate the number of background electrons.
In what follows, the set of antiloose electrons will also be referred to as the
fake electron template. Its part in the signal region will be denoted by $T$,
and that in the background-dominated region will be denoted by $T_>$.

In order to check if the set of antiloose electrons would be a suitable
distribution, the set of background electrons in $N$, namely $B_N$, is plotted
against the former and shown in Figure~\ref{f:shapebeforeafter}, both
normalized to unity. As is seen in the figure, the antiloose selection is
expected to be effective for classifying background electrons in the sample. On
the other hand, Figure~\ref{f:antiloosemrjet} shows the composition of
antiloose electrons in the $m_{\text{rjet}}^{\text{el}}$ distribution.
Simulation predicts about $10\%$ of signal electron contamination, but
otherwise the distribution is made up of mostly background electrons and
dominated by semileptonic $t\tbar$.


\begin{figure}[H]
	\includegraphics[width=10cm]{figures/w_allbn_probes_m_rjet_overlap_el_m0_pt100_antiloose_m_rjet_overlap_el_mar22}
	\centering
	\caption{The distribution $m_{\text{rjet}}^{\text{el}}$ of $B_N$ against that
		of $T$, normalized to unity. $T$ describes very well $B_N$ and therefore it is
		reasonable to estimate $B_N$ using $T$.}
	\label{f:shapebeforeafter}
\end{figure}

\begin{figure}[H]
	\includegraphics[width=10cm]{figures/probe_pt40_probes_contam_m0_pt100_mar05_antiloose_m_rjet_overlap_el_mar22}
	\centering
	\caption{The distribution $m_{\text{rjet}}^{\text{el}}$ for electrons that fail
		the Loose ID point, also called antiloose electrons.}
	\label{f:antiloosemrjet}
\end{figure}

Background estimation using antiloose electrons proceeds in detail as follows:

\begin{enumerate}

	\item Obtaining $T$ and $T_>$ by selecting antiloose electrons. Thus the method
	      is data-driven, $T$ and $T_>$ from simulations are not used.

	\item In addition to $N$, the set of reconstructed electron candidates in the
	      signal region, there is also the set of reconstructed electron candidates in the
	      background-dominated region, which will be denoted $N_>$.

	      Subtracting signal contamination from $T_>$, the resulting set of which is
	      denoted $\overline{T}_>$, and subtract signal contamination from $N_>$, and
	      denote the resulting set by $\overline{N}_>$. Normalizing $\overline{T}_>$ to
	      $\overline{N}_>$ to obtain a normalization factor.

	\item Subtracting signal contamination from $T$, the resulting set of which is
	      denoted $\overline{T}$, and applying the normalization factor to $\overline{T}$
	      to obtain the number of background electrons in the signal region.

\end{enumerate}

In other words, the background to be estimated in the signal region, $B_N$, is

\begin{equation}\label{eid:bne}
	B_N =  \overline{T} \times \frac{\overline{N}_>}{\overline{T}_>}
\end{equation}

The following section discusses signal contamination subtractions in $T$,
$T_>$, and $N_>$, and the measurements of the idenfitication efficiencies.

\subsection{The Measurements of the Identification Efficiency}\label{s:meffs}

The idenfitication efficiency $\epsilon$ (Formula~\ref{eqn:effo}), in which
$B_P$ is taken from simulation and $B_N$ evaluated according to
Formula~\ref{eid:bne}, is


\begin{equation}\label{eid:efff}
	\epsilon = \frac{P-B_P}{N - \overline{T} \times \frac{\overline{N}_>}{\overline{T}_>}}.
\end{equation}


$\overline{T}$ is the set of antiloose electrons in the signal region, $T$,
minus signal contamination, and $\overline{T}_>$ is the corresponding quantity
in the background-dominated region. As there is an expected of $10\%$ of signal
contamination in the set of antiloose electrons
(Figure~\ref{f:shapebeforeafter}), $\overline{T}$ and $\overline{T}_>$ will be
obtained by subtracting signal contamination as predicted by simulations from
$T$ and $T_>$ respectively.

On the other hand, signal contamination in $N_>$ (Figure~\ref{f:crmrjet}), from
which $\overline{N}_>$ is obtained, is larger, and to reduce the contribution
from the estimation of this signal contamination to the uncertainty in the
efficiency we will use a data-driven approach. According to
Figure~\ref{f:eidmedium}, the number of background electrons after a Medium or
Tight ID selection is negligible. We expect as a result $P$, and the
corresponding quantity $P_>$ in the background-dominated region, to be
relatively free of background electrons. Thus $P_>$ could be used to represent
signal contamination in $N_>$, provided the corresponding identification
efficiency is properly taken into account. In other words,

%
$$\overline{N}_> = N_> - P_> / \epsilon $$
%

where the efficiency in~\ref{eid:efff}, which is being measured, is used again.
The efficiency will be evaluated iteratively, until the change from one
iteration to the next is less than $0.5\%$. The value of $0.5\%$ will be taken
as the uncertainty due to signal contamination subtraction in $N_>$.

The next section discusses in detail the treatment of statistical and
systematic uncertainties.

%%%%%%%%%%%%%%%%%%%%%%%%%%%%%%%%%%%%%%%%%%%%%%%%%%%%%%%%%%%%
\subsection{Uncertainties}\label{s:eidunc}

The measurement of the identification efficiency is accompanied by statistical
and systematic uncertainties, both of which are discussed in the following.

\paragraph{Statistical Uncertainties} According to Formula~\ref{eid:efff}, the
efficiency will be measured according to the formula


$$
	\epsilon = \frac{P-B_P}{N - \overline{T} \times \frac{\overline{N}_>}{\overline{T}_>}}
$$

where

\begin{itemize}
	\item $P$ is the number of electrons that pass Medium or Tight.

	\item $B_P$ is background contamination due to fake electrons in $P$.

	\item $N$ is the set of reconstructed electron candidates, and $\overline{N}_>$
	      the corresponding quantity in the background-dominated region minus signal
	      contamination.

	\item $\overline{T}$ is the set of antiloose electrons minus signal
	      contamination, and $\overline{T}_>$ the corresponding quantity in the
	      background-dominated region.

\end{itemize}

Since $N$ contains $P$, and $\overline{N}_>$ contains $\overline{T}_>$, the
quantities in the formula are not all independent. We may remove the
correlation between $N$ and $P$ by writing $N = P + F$, where $F$ is the set of
electrons that fail a particular ID point. Then

$$
	\epsilon = \frac{P-B_P}{P + F - \overline{T} \times \frac{\overline{N}_>}{\overline{T}_>}}
$$


The correlation between $\overline{N}_>$ and $\overline{T}_>$ remains, and
moreover $F$ and $\overline{T}$ are also correlated, because in the Medium case
or in the Tight case, $F$ represents electrons failing Medium or Tight
respectively, and since $\overline{T}$ represents electrons failing Loose
(minus signal contamination), in each case $\overline{T}$ is a subset of $F$
and there is accordingly a correlation.

In order to remove all the correlations and write the efficiency completely in
terms of statistically independent quantities we will first multiply both the
numerator and the denominator by $\overline{T}_>$, to write

$$
	\epsilon = \frac{(P-B_P)\overline{T}_>}{P\overline{T}_> + F\overline{T}_> - \overline{T}\times \overline{N}_>}.
$$

Then we will add and subtract $\overline{T} \times \overline{T}_>$, to have


\begin{equation*}
	\begin{split}
		\epsilon & = \frac{(P-B_P)\overline{T}_>}{P\overline{T}_> + F\overline{T}_> - \overline{T}\times \overline{T}_>
			+ \overline{T}\times \overline{T}_> - \overline{T}\times \overline{N}_>}  \\
		&  =  \frac{(P-B_P)\overline{T}_>}{P\overline{T}_> + (F - \overline{T})\overline{T}_>
			- (\overline{N}_> - \overline{T}_>)\overline{T}}
	\end{split}
\end{equation*}

The difference $F - \overline{T}$ represents the set of electrons that fail
Medium or Tight but pass the Loose identification, and the difference
$\overline{N}_> - \overline{T}_>$ represents the set of electrons that pass the
Loose identification. If we treat each of the differences as a single term, and
set $S = F - \overline{T}$ and $\overline{R}_> = \overline{N}_> -
	\overline{T}_>$ respectively, the efficiency becomes

\begin{equation}\label{eq:effco}
	\epsilon = \frac{(P-B_P)\overline{T}_>}{P\overline{T}_> + S\overline{T}_>
		- \overline{R}_>\times \overline{T}}
\end{equation}

which is now a function of six independent quantities, $\epsilon=\epsilon(P,
	B_P, \overline{T}_>, S, \overline{R}_>, T)$. The statistical uncertainty of the
efficiency then follows the standard error propagation formula,

\begin{equation}\label{eq:statprop}
	\Delta \epsilon^2 = \bigg(\frac{\partial \epsilon}{\partial P}\bigg)^2\Delta P^2 + \cdots +
	\bigg(\frac{\partial \epsilon}{\partial T}\bigg)^2\Delta T^2
\end{equation}


Let $A$ denote the numerator in Formula~\ref{eq:effco} and $B$ the denominator.
The terms in the formula above are then

$$\frac{\partial \epsilon}{\partial P} = \frac{B \overline{T}_> - A \overline{T}_> }{B^2},
	\quad
	\frac{\partial \epsilon}{\partial B_P} = \frac{- B \overline{T}_> }{B^2},
	\quad
	\frac{\partial \epsilon}{\partial \overline{T}_>} = \frac{B (P - B_P) - A(P + S) }{B^2},
$$

$$\frac{\partial \epsilon}{\partial S} = \frac{- A \overline{T}_> }{B^2},
	\quad
	\frac{\partial \epsilon}{\partial \overline{R}_>} = \frac{AT }{B^2},
	\quad
	\frac{\partial \epsilon}{\partial \overline{T}} = \frac{A \overline{R}_> }{B^2}.
$$

Among the terms, only $P$ and $S$ are present in the signal region that are not
used for background estimation, and as a result the statistical uncertainty of
the efficiency will be taken from the contributions of these two terms. The
contributions to the uncertainty from other terms, which are used for
background estimation, will be taken as contributions to the total systematic
uncertainty.

%%%%%%%%%%%%%%%%%%%%%%%%%%%%%%%%%%%%%%%%%%%%%%%%%%%%%%%%%%%%%%%%%%%%%%%%%%%%%%%%%%%%%%
%%%%%%%%%%%%%%%%%%%%%%%%%%%%%%%%%%%%%%%%%%%%%%%%%%%%%%%%%%%%%%%%%%%%%%%%%%%%%%%%%%%%%%
%%%%%%%%%%%%%%%%%%%%%%%%%%%%%%%%%%%%%%%%%%%%%%%%%%%%%%%%%%%%%%%%%%%%%%%%%%%%%%%%%%%%%%

\paragraph{Systematic Uncertainties} Contributions to the total systematic
uncertainty from different sources will be added in quadrature. The sources are
listed below.

\begin{itemize}

	\item The variation of the signal region, i.e. instead of marking the signal
	      region at $60$ GeV, we may mark it at $50$ or $80$ GeV, the asymmetry is due to
	      the fact that signal distributions on both sides of the $60$ GeV mark are not
	      equal in equal intervals.


	\item The term $B_P$ which represents the background contamination in $P$ is
	      taken from simulation and may be varied up and down. Here a $50\%$ variation
	      which will be used represents a conservative estimate of the contribution of
	      this term.


	\item The uncertainty due to signal contamination subtraction from $T$ and
	      $T_>$, from which result $\overline{T}$ and $\overline{T}_>$, are obtained
	      by conservatively varying the signal contamination $25\%$ up and down.

	\item The template $T$, which is the distribution of antiloose electrons,
	      is replaced by the distribution of antiloose electrons in events with exactly 2
	      $b$-jets.


	\item In addition, the statistical uncertainties from the counting of
	      $\overline{T}_>$, $\overline{R}_>$, and $\overline{T}$ in
	      Formula~\ref{eq:effco} are treated as contributions to the total systematic
	      uncertainty as well.


\end{itemize}


%%%%%%%%%%%%%%%%%%%%%%%%%%%%%%%%%%%%%%%%%%%%%%%%%%%%%%%%%%%%%%%%%%%%%%%%%%%%%%%%%%%%%%%%%%%%
%%%%%%%%%%%%%%%%%%%%%%%%%%%%%%%%%%%%%%%%%%%%%%%%%%%%%%%%%%%%%%%%%%%%%%%%%%%%%%%%%%%%%%%%%%%%
%%%%%%%%%%%%%%%%%%%%%%%%%%%%%%%%%%%%%%%%%%%%%%%%%%%%%%%%%%%%%%%%%%%%%%%%%%%%%%%%%%%%%%%%%%%%
%%%%%%%%%%%%%%%%%%%%%%%%%%%%%%%%%%%%%%%%%%%%%%%%%%%%%%%%%%%%%%%%%%%%%%%%%%%%%%%%%%%%%%%%%%%%

\section{Identification Efficiencies}\label{s:eideff}

In this section the integrated efficiencies as well as the binned efficiencies
for the Medium and Tight operating points are presented, along with the
associated uncertainties.


\subsection{Integrated Efficiencies}

The identification efficiencies for electrons inside jets are measured for the
Medium and Tight operating points; they are evaluated iteratively according to
Formula~\ref{eid:efff}, which is

$$
	\epsilon = \frac{P-B_P}{N - \overline{T} \times \frac{\overline{N}_>}{\overline{T}_>}}.
$$

where

$$\overline{N}_> = N_> - P_> / \epsilon $$

The integrated efficiencies are presented in the following, along with the
associated uncertainties.

\subsubsection{The efficiencies}

The efficiencies, as well as the total statistical and systematic
uncertainties, are $\mathbf{0.870\pm 0.017 \pm 0.031}$ for Medium and
$\mathbf{0.784 \pm 0.019 \pm 0.020}$ for Tight. As is seen, the efficiency is
higher for Medium than for Tight, consistent with expectation. The statistical
uncertainties are slightly larger for Tight, also consistent with expectation,
as the stats for Tight is slightly less than that for Medium. The relevant
quantities in Formula~\ref{eid:efff} that are used to compute the efficiencies
in data are listed in Table~\ref{t:inteffq}.

\renewcommand{\arraystretch}{1.15}
\begin{table}[H]
	\centering
	\begin{tabularx}{0.6\textwidth}{| c | *{2}{Y|} }
		\cline{2-3}
		\multicolumn{1}{c |}{} & MEDIUM & TIGHT   \\[1.0ex]
		\hline\hline
		\toprule
		~~~~~~$P$~~~~~~        & $ 392$ & $ 356$  \\[1ex]
		\hline
		$B_P$                  & $1.47$ & $ 0.40$ \\[1ex]
		\hline
		$N$                    & $734$  & $ 734$  \\[1ex]
		\hline
		$\thead{\overline{N_>} \\ }$ & $368$ &  $ 368$ \\[1ex]
		\hline
		$P_>$                  & $49$   & $ 48$   \\[1ex]
		\hline
		$\thead{\overline{T} \\ }$  & \multicolumn{2}{c |}{$267.35$} \\[1ex]
		\hline
		$\thead{\overline{T_>} \\ }$  &  \multicolumn{2}{c |}{$292.52$} \\[1ex]
		\hline
		\toprule
	\end{tabularx}
	\caption{The relevant quantities for computing the efficiencies according to
		Formula~\ref{eid:efff}.}
	\label{t:inteffq}
\end{table}
\renewcommand{\arraystretch}{1}

The efficiencies and statistical uncertainties in simulation for the Medium and
Tight operating points are also computed and are $\mathbf{0.871\pm 0.010}$ and
$\mathbf{0.807\pm 0.011}$ respectively.

\subsubsection{Statistical Uncertainties}

As has been discussed in Section~\ref{s:eidunc}, when evaluating the
efficiencies in data, the quantities in the signal region that are not used for
background estimation are $P$ and $S$, and the statistical uncertainty of the
efficiency is taken from the contributions of these two terms, computed
according to Formula~\ref{eq:effco} and shown in Table~\ref{t:inteffss}. Of the
two, the contribution from $S$ is the dominant one; the contribution from $P$
is small ($< 0.5\%$).

\subsubsection{Systematic Uncertainties}

The total systematic uncertainty receives contributions from different sources,
as discussed in Section~\ref{s:eidunc}. The individual contributions are shown
below.

\paragraph{Contributions from $\overline{T}_>$, $\overline{R}_>$, and
	$\overline{T}$} The contributions to the total systematic uncertainty that come
from the counting of $\overline{T}_>$, $\overline{R}_>$, and $\overline{T}$ in
Formula~\ref{eq:effco} are listed in Table~\ref{t:statsources} below. They each
contributes $< 1\%$ to the total uncertainty. The relevant quantities used for
the calculations are listed in Table~\ref{t:syssourcesstats}. The term
$\overline{R}_>$ is computed as the difference $\overline{N}_> -\overline{T}_>$
(Section~\ref{s:eidunc}), where $\overline{N}_> = N_> - P_> /\epsilon $ as
discussed in Section~\ref{s:meffs}.

\renewcommand{\arraystretch}{1.20}
\begin{table}[H]
	\centering
	\begin{tabularx}{0.8\textwidth}{| c | *{2}{Y|} }
		\cline{2-3}
		\multicolumn{1}{c |}{}                          & MEDIUM      & TIGHT        \\[1.0ex]
		\hline\hline
		\toprule
		$\Delta \overline{R}_> = \sqrt{\overline{R}_>}$ & $\pm 0.008$ & $ \pm 0.006$ \\
		\hline
		$\Delta \overline{T} = \sqrt{\overline{T}}$     & $\pm 0.002$ & $ \pm 0.001$ \\
		\hline
		$\Delta \overline{T}_> = \sqrt{\overline{T}_>}$ & $\pm 0.002$ & $ \pm 0.001$ \\
		\hline
	\end{tabularx}
	\caption{Contributions to the total systematic uncertainty from the individual
		sources.}
	\label{t:statsources}
\end{table}
\renewcommand{\arraystretch}{1.0}


\begin{table}[H]
	\centering
	\begin{tabularx}{0.8\textwidth}{| c |*{2}{Y|} }
		\cline{2-3}
		\multicolumn{1}{c |}{}       & \thead{MEDIUM } & \thead{TIGHT } \\[1.0ex]
		\hline\hline
		\toprule
		~~~~~~$\overline{R}_>$~~~~~~ & 19.11           & 14.24          \\
		\hline
		$\overline{T} $              & 267.35          & 267.35         \\
		\hline
		$\overline{T}_>$             & 292.52          & 292.52         \\
		\hline
	\end{tabularx}
	\caption{The quantities $\overline{R}_>$, $\overline{T}_>$, and $\overline{T}$
		that are used to compute the contributions of these terms to the total
		systematic uncertainty according to Formula~\ref{eq:effco}. The term
		$\overline{R}_>$ is computed as the difference $\overline{N}_> -
			\overline{T}_>$ (Section~\ref{s:eidunc}).}
	\label{t:syssourcesstats}
\end{table}


%%%%%%%%%%%%%%%%%%%%%%%%%%%%%%%%%%%%%%%%%%%%%%
\paragraph{Contribution from $B_P$}

The contribution to the total systematic uncertainty from varying the term
$B_P$ (Formula~\ref{eq:effco}) is shown below in Table~\ref{t:syssourcesbp}.
The total contribution is taken to be the sum in quadrature of the two
variations. The countings of $B_P$ in each of the variations are shown in
Table~\ref{t:syssourcesb}. We see that the contribution from $B_P$ is very
small.

\begin{table}[H]
	\centering
	\begin{tabularx}{0.8\textwidth}{| c | *{2}{Y|} }
		\cline{2-3}
		\multicolumn{1}{c |}{}   & MEDIUM & TIGHT \\[1.0ex]
		\hline\hline
		\toprule
		$B_P\times 1.5$          & 0.002  & 0.000 \\
		\hline
		$B_P\div 1.5$            & 0.001  & 0.000 \\
		\hline
		\toprule
		$B_P$ total contribution & 0.002  & 0.000 \\
		\hline
	\end{tabularx}
	\caption{Contributions to the total systematic uncertainty from the term $B_P$.}
	\label{t:syssourcesbp}
\end{table}

\begin{table}[H]
	\centering
	\begin{tabularx}{0.8\textwidth}{| l |*{2}{Y|} }
		\cline{2-3}
		\multicolumn{1}{c |}{} & \thead{MEDIUM } & \thead{TIGHT } \\[1.0ex]
		\hline\hline
		\toprule
		Semileptonic $t\tbar$  & 1.13            & 0.40           \\
		\hline
		Background single top  & 0.34            & 0.00           \\
		\hline
		W+jets                 & 0.00            & 0.00           \\
		\toprule
		\hline
		$B_P\times 1.5$        & 2.2             & 0.61           \\
		\hline
		$B_P\div 1.5$          & 0.98            & 0.27           \\
		\hline
		%\toprule 
		%Uncertainty &  &   \\
		\hline
	\end{tabularx}
	\caption{Contributions to the systematic uncertinty from $B_P$.}
	\label{t:syssourcesb}
\end{table}

%%%%%%%%%%%%%%%%%%%%%%%%%%%%%%%%%%%%%%%%%%%%%%%%%

\paragraph{Contribution from signal contaminations in $T$ and $T_>$} The
contributions due to signal comtamination in the terms $T$ and $T_>$ are listed
in Table~\ref{t:syssourcestn}. These are a major sources of contribution to the
total systematic uncertainty. The signal contaminations in the two terms $T$
and $T_>$ are varied simultaneously, either up or down by $25\%$. The countings
obtained from the variations are shown in Table~\ref{t:syssourcest}.

\begin{table}[H]
	\centering
	\begin{tabularx}{0.8\textwidth}{| c | *{2}{Y|} }
		\cline{2-3}
		\multicolumn{1}{c |}{} & MEDIUM & TIGHT \\[1.0ex]
		\hline\hline
		\toprule
		$\times 1.25$          & 0.014  & 0.012 \\
		\hline
		$\div 1.25$            & 0.010  & 0.010 \\
		\toprule
		\hline
		Total contribution     & 0.017  & 0.015 \\
		\hline
	\end{tabularx}
	\caption{}
	\label{t:syssourcestn}
\end{table}



\begin{table}[H]
	\centering
	\begin{tabularx}{0.8\textwidth}{| c |*{2}{Y|} }
		\cline{2-3}
		\multicolumn{1}{c |}{}          & \thead{MEDIUM } & \thead{TIGHT } \\[1.0ex]
		\hline\hline
		\toprule
		$T$                             & 312             & 312            \\
		\hline
		Dilepton contamination          & 28.43           & 28.43          \\
		\hline
		Single top signal contamination & 16.23           & 16.23          \\
		\hline
		$\overline{T} \times 1.5$       & ?               & ?              \\
		\hline
		$\overline{T} \div 1.5$         & ?               & ?              \\
		\hline\hline
		\toprule
		$T_>$                           & 317             & 317            \\
		\hline
		Dilepton contamination          & 5.45            & 5.45           \\
		\hline
		Single top signal contamination & 19.03           & 19.03          \\
		\hline
		$\overline{T}_> \times 1.5$     & ?               & ?              \\
		\hline
		$\overline{T}_> \div 1.5$       & ?               & ?              \\
		\hline
	\end{tabularx}
	\caption{}
	\label{t:syssourcest}
\end{table}

%%%%%%%%%%%%%%%%%%%%%%%%%%%%%%%%%%%%%%%%%%%%

\paragraph{Contribution from changing the template to antiloose+2b} The
contribution to the total systematic uncertainty from the template change is
shown in Table~\ref{t:syssources2b}. The template change leads to changes in
quantities $\overline{T}$ and $\overline{T}_>$, which are listed in
Table~\ref{t:syssources2bc}.

\begin{table}[H]
	\centering
	\begin{tabularx}{0.8\textwidth}{| c | *{2}{Y|} }
		\cline{2-3}
		\multicolumn{1}{c |}{} & MEDIUM & TIGHT \\[1.0ex]
		\hline\hline
		\toprule
		antiloose+2b           & 0.008  & 0.007 \\
		\hline
	\end{tabularx}
	\caption{Contributions to the total systematic uncertainty from changing the template to
		antiloose+2b}
	\label{t:syssources2b}
\end{table}

\begin{table}[H]
	\centering
	\begin{tabularx}{0.8\textwidth}{| l |*{2}{Y|} }
		\cline{2-3}
		\multicolumn{1}{c |}{}       & \thead{MEDIUM } & \thead{TIGHT } \\[1.0ex]
		\hline\hline
		\toprule
		~~~~~~$\overline{T}$~~~~~~   & 106.03          & 106.03         \\
		\hline
		~~~~~~$\overline{T}_>$~~~~~~ & 114.49          & 114.49         \\
		\hline
	\end{tabularx}
	\caption{}
	\label{t:syssources2bc}
\end{table}

\paragraph{Contributions from re-marking the signal region} The contribution to
the total systematic uncertainty from re-marking the signal region is shown in
Table~\ref{t:syssourcessn}. It is another major contribution to the total
systematic uncertainty. The signal region is marked, in place of at $60$ GeV,
at 50 GeV and 80 GeV in turn. The total contribution is computed as a sum of
quadrature of the two individual contributions. The relevant quantities used
for the calculations of the efficiencies are listed in Table~\ref{t:inteffq50}
and~\ref{t:inteffq80}.

\begin{table}[H]
	\centering
	\begin{tabularx}{0.8\textwidth}{| c | *{2}{Y|} }
		\cline{2-3}
		\multicolumn{1}{c |}{} & MEDIUM & TIGHT \\[1.0ex]
		\hline\hline
		\toprule
		50 GeV                 & 0.010  & 0.004 \\
		\hline
		80 GeV                 & 0.020  & 0.001 \\
		\toprule
		\hline
		Total contribution     & 0.022  & 0.010 \\
		\hline
	\end{tabularx}
	\caption{Contributions to the total systematic uncertainty re-marking the signal
		region at 50 GeV and 80 GeV.}
	\label{t:syssourcessn}
\end{table}

\renewcommand{\arraystretch}{1.15}
\begin{table}[H]
	\centering
	\begin{tabularx}{0.8\textwidth}{| c | *{2}{Y|} }
		\cline{2-3}
		\multicolumn{1}{c |}{}  & MEDIUM & TIGHT   \\[1.0ex]
		\hline\hline
		\toprule
		~~~~~~~~~~$P$~~~~~~~~~~ & $ 350$ & $ 314$  \\
		\hline
		$B_P$                   & $1.47$ & $ 0.40$ \\
		\hline
		$N$                     & $593$  & $ 593$  \\
		\hline
		$\thead{\overline{N_>} \\ }$ & $509$ &  $ 368$ \\
		\hline
		$P_>$                   & $91$   & $ 90$   \\
		\hline
		$\thead{\overline{T} \\ }$  & $182.76$ &  $ 182.76$ \\
		\hline
		$\thead{\overline{T_>} \\ }$  & $377.11$ &  $ 377.11$ \\
		\hline
		\toprule
	\end{tabularx}

	\caption{The relevant quantities to compute the efficiencies for the Medium and
		Tight operating points when marking the signal region at 50 GeV.}

	\label{t:inteffq50}

\end{table}
\renewcommand{\arraystretch}{1.0}


\renewcommand{\arraystretch}{1.15}
\begin{table}[H]
	\centering
	\begin{tabularx}{0.8\textwidth}{| c | *{2}{Y|} }
		\cline{2-3}
		\multicolumn{1}{c |}{}  & MEDIUM & TIGHT   \\[1.0ex]
		\hline\hline
		\toprule
		~~~~~~~~~~$P$~~~~~~~~~~ & $ 420$ & $ 384$  \\
		\hline
		$B_P$                   & $1.56$ & $ 0.43$ \\
		\hline
		$N$                     & $734$  & $ 734$  \\
		\hline
		$\thead{\overline{N_>} \\ }$ & $238$ &  $ 238$ \\
		\hline
		$P_>$                   & $21$   & $ 20$   \\
		\hline
		$\thead{\overline{T} \\ }$  & $355.54$ &  $ 355.54$ \\
		\hline
		$\thead{\overline{T_>} \\ }$  & $204.33$ &  $ 204.33$ \\
		\hline
		\toprule
	\end{tabularx}

	\caption{The relevant quantities to compute the efficiencies for the Medium and
		Tight operating points when marking the signal region at 80 GeV.}

	\label{t:inteffq80}

\end{table}
\renewcommand{\arraystretch}{1.0}

\subsection{Efficiencies in Bins}

In addition to the integrated efficiencies, binned efficiencies are also
measured, to check the possible dependencies of the efficiencies on certain
variables. The variables and their associated binnings are:

\begin{itemize}
	\item $p_T$ of the probe, in five bins
	      \begin{itemize}
		      \item 30-60 GeV,
		      \item 60-80 GeV,
		      \item 80-110 GeV,
		      \item 110-140 GeV,
		      \item $> 140$ GeV.
	      \end{itemize}
	\item $\abs{\eta}$ of the probe, in five bins
	      \begin{itemize}
		      \item 0.0-0.3,
		      \item 0.3-0.6,
		      \item 0.6-0.9,
		      \item 0.9-1.3,
		      \item $> 1.3$.
	      \end{itemize}
	\item $\Delta R $ between the probe and the closest overlapping jet, in five bins
	      \begin{itemize}
		      \item 0.0-0.15,
		      \item 0.15-0.19,
		      \item 0.19-0.23,
		      \item 0.23-0.27,
		      \item 0.27-0.4.
	      \end{itemize}
	\item $p_T$ of the closest overlapping jet, in five bins
	      \begin{itemize}
		      \item 150-220 GeV,
		      \item 220-280 GeV,
		      \item 280-340 GeV,
		      \item 340-400 GeV,
		      \item 400-500 GeV.
	      \end{itemize}
\end{itemize}

The distributions of these variables are shown in Figure~\ref{f:bins01} and~\ref{f:bins02}.
The binned efficiencies are shown in Figure~\ref{f:binspt01} and~\ref{f:binspt02}.


\begin{figure}[H]
	\begin{subfigure}{0.5\textwidth}
		\includegraphics[width=8cm]{figures/nvariables_pt40_probes_cr1_m0_pt100_mar05_loose_real_loose_fake_dr_el_cjet_dr_el_cjet_mar22}
		\label{bfptprobe}
		\caption{$\Delta R $ }
	\end{subfigure}
	\begin{subfigure}{0.5\textwidth}
		\includegraphics[width=8cm]{figures/nvariables_pt40_probes_cr1_m0_pt100_mar05_loose_real_loose_fake_el_eta_el_eta_mar22}
		\caption{$\abs{\eta}$ of the probe}
	\end{subfigure}

	\centering

	\caption{The distributions of $\Delta R $ between the probe and the closest
		overlapping jet and $\abs{\eta}$ of the probe.}

	\label{f:bins01}
\end{figure}

\begin{figure}
	\begin{subfigure}{0.5\textwidth}
		\includegraphics[width=8cm]{figures/nvariables_pt40_probes_cr1_m0_pt100_mar05_loose_real_loose_fake_el_pt_el_pt_mar22}
		\caption{$p_T$ of the probe}
	\end{subfigure}
	\begin{subfigure}{0.5\textwidth}
		\includegraphics[width=8cm]{figures/nvariables_pt40_probes_cr1_m0_pt100_mar05_loose_real_loose_fake_pt_c_jet_pt_c_jet_mar22}
		\caption{$p_T$ of the closest overlapping jet}
	\end{subfigure}

	\centering

	\caption{The distributions of $p_T$ of the probe and $p_T$ of the closest
		overlapping jet.}

	\label{f:bins02}
\end{figure}


\begin{figure}[H]
	\begin{subfigure}{0.5\textwidth}
		\includegraphics[width=8cm]{figures/sep04_bins_probe_pt_medium_tight_dt_mc}
	\end{subfigure}
	\begin{subfigure}{0.5\textwidth}
		\includegraphics[width=8cm]{figures/sep04_bins_probe_eta_medium_tight_dt_mc}
	\end{subfigure}

	\centering
	\caption{The binned efficiencies in $p_T$ of the probe as well as in $\abs{eta}$ of the
		probe.}
	\label{f:binspt01}
\end{figure}

\begin{figure}
	\begin{subfigure}{0.5\textwidth}
		\includegraphics[width=8cm]{figures/sep04_bins_dr_medium_tight_dt_mc}
	\end{subfigure}
	\begin{subfigure}{0.5\textwidth}
		\includegraphics[width=8cm]{figures/sep04_bins_jetpt_medium_tight_dt_mc}
	\end{subfigure}

	\centering
	\caption{The binned efficiencies in $\Delta R$ between the probe and the closest jet,
		as well as in $p_T$ of the closest overlapping jet.}
	\label{f:binspt02}
\end{figure}


\section{Conclusions}\label{s:eidcon}

This chapter describes the work to measure the identification efficiencies for
in-jet electrons. It was the first attempt to perform such a measurement since
Run 2 began, and the first ever using dilepton $t\tbar$ events. The measurement
used the data collected in the period 2015-2016, at $13$ TeV center-of-mass and
totalled $36.5~\text{fb}^{-1}$ in integrated luminosity. A sample of electrons
for the measurements was obtained by selecting boosted $t\bar{t}$ dilepton
($e\mu$) events. Background estimations used both simulations and data, and the
efficiencies were evaluated iteratively. The efficiencies were measured for the
Medium and Tight operating points, both on data and simulation. The
efficiencies as functions of the properties of the electrons and of the
overlapping jets also measured. In all of the results, the efficiencies
predicted by simulation agree with those obtained from the measurements on
data.


