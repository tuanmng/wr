\chapter{IN-JET ELECTRON IDENTIFICATION EFFICIENCIES}\label{c:eid}

In early 2015 the LHC restarted after two years of shutdown, beginning what is
referred to as Run 2. The new centre-of-mass energy was $13$ TeV, in place of
the previous $8$ TeV. The higher energy opens up unexplored parameter space and
allows further probe of supersymmetry (SUSY) and other
beyond-the-Standard-Model processes~\cite{multib-c7, stop-c7, ttresonance-c7,
	vlq-c7}. At ATLAS, many SUSY searches involve supersymmetric particles that
decay into the standard model top quarks and we expect, with higher
centre-of-mass energy, more massive supersymmetric particles and consequently
an increase in the production of high $p_T$ top quarks. Since the top quarks
also decay, in fact most of the time into a $W$ boson and a $b$ quark, we in
turns expect boosted decay topology, in other words the daughter particles of
the top quarks, which include the daughter particles of the $W$ boson and the
$b$ quark, tend to stay close to each other (Figure~\ref{f:boosttop}). In the
case of leptonic top quark decays, the lepton tends to be very close to the
produced $b$-jet.


This chapter describes the work to measure the identification efficiencies for
electrons that are found inside $\Delta R = 0.4$ of high $p_T$ jets, which will
also be referred to in the chapter as in-jet electrons. Prior to the work in
this chapter, there was no attempt to measure the identification efficiencies
for in-jet electrons. The chapter will be organized as follows. In
section~\ref{s:eidmot} we motivate the need for the measurement of the
identification efficiencies for in-jet electrons. Section~\ref{s:eidmet}
describes the method used to perform the measurements. Section~\ref{s:eideff}
presents the measured efficiencies for in-jet electrons and
section~\ref{s:eidcon} gives some conclusions.

The data used for this chapter was collected in the period 2015-2016 at $13$
TeV center-of-mass and correponded to an integrated luminosity of
$36.47~\text{fb}^{-1}$.

\section{Motivation}\label{s:eidmot}

At ATLAS, prior to Run 2, electrons found outside $\Delta R = 0.4$ of jets were
used by most analyses. Indeed, although there were some attempts to select
signal electrons inside jets~\cite{?}, at 7-8 TeV such electrons can hardly be
expected; objects inside jets that are identified as electrons are mostly
background electrons that are either hadrons faking jets or real electrons
coming from heavy-flavour jet decays. As a result, most analyses rejected
electrons inside $\Delta R = 0.4$ of jets and only worked with electrons
outside jets, of which different efficiencies, among them identification
efficiencies, are measured and provided by the ATLAS Egamma
group~\cite{atlaselcid}.

\vspace{3mm}

\begin{figure}[H]
	\includegraphics[width=12cm]{figures/boost}
	\centering

	\caption{An illustration of low $p_T$ top quark decay (left) and boosted top
		decay (right) of a high $p_T$ top quark. In the case of high $p_T$ top quark
		decay the daughter particles of the top quark, which include the daughter
		particles of the $W$ and the $b$ quark, are expected to be found close to each
		other~\cite{boostedtop-fig}.}

	\label{f:boosttop}
\end{figure}

\vspace{3mm}

In Run 2, as the centre-of-mass of the LHC reached $13$ TeV, high $p_T$ top
quarks are expected to come from the decays of massive supersymmetric or other
beyond-the-Standard-Model particles. As we have already mentioned, a top quark
decays almost all of the time into a $W$ and a $b$ quark, and high $p_T$ top
quarks are expected to undergo boosted decays where the produced particles,
here the daughters of the $W$ and the $b$ quark, are found close to each other.
In particular, there are two possible scenarios: either the $W$ decays
hadronically, in which case the produced jets stay close to each other, or it
may decay leptonically, in which case the leptons are expected to be found near
the produced $b$-jets. Either scenario is important and the possibility of
encountering boosted topology was assessed by several ATLAS physics groups,
albeit most works focused on hadronic boosted top quark
decays~\cite{hadronic01-ch7, hadronic02-ch7, bdis-figs}.
Figure~\ref{f:drwbtoppt} shows the angular distance $\Delta R$
(Formula~\ref{eq:angulardr}) between the $W$'s and the $b$-quarks as a function
of the top $p_T$, in the context of a hypothetical particle $Z'$ at mass
$m_{Z'}=1.6$ TeV~\cite{bdis-figs} that decays into a $t\bar{t}$ pair. It also
shows the separation between the light quarks of the subsequent hadronic decay
of the $W$ boson. As can be seen, the angular distance decreases as the top
quark $p_T$ increases, and at high top quark $p_T$ a non-negligible fraction of
the distances becomes very small. In the case of leptonic top quark decays, we
should expect the decay products of the $W$'s to be found very near the
$b$-jets.

\begin{figure}[H]

	\begin{subfigure}{0.5\textwidth}
		\includegraphics[width=7.5cm]{figures/fig_01a}
		\caption{$t\to Wb$}
		\label{drwbtoppta}
	\end{subfigure}
	\begin{subfigure}{0.5\textwidth}
		\includegraphics[width=7.5cm]{figures/fig_01b}
		\caption{$W\to q\bar{q}$}
		\label{drwbtopptb}
	\end{subfigure}

	\centering
	\caption{(\ref{drwbtoppta})~The angular distance $\Delta R$ between
		the $W$'s and the $b$ quarks as a function of the top quark $p_T$ simulated
		PYTHIA~\cite{sjostrand:ch7}, in the context of a hypothetical particle $Z'$
		($m_{Z'}=1.6$ TeV) that decays into a $t\bar{t}$ pair. At high top quark $p_T$
		a non-negligible fraction of the distances is seen to be very
		small.~(\ref{drwbtopptb})~The angular distance between two light quarks from
		$t\to Wb$ decay as a function of the $p_T$ of the $W$ boson~\cite{bdis-figs}.}
	\label{f:drwbtoppt}
\end{figure}


Taking advantage of the new topology, several SUSY searches which expect
standard model top quarks from the decays of massive supersymmetric particles
began to include electrons inside $\Delta R = 0.4$ of jets in their analyses.
Considerable increases in signal acceptance were seen by different analysis
groups, including the search for the gluinos discussed in
Chapter~\ref{c:susys}, which in fact was the analysis that motivated the
measurements described in this chapter.


In the context of the work of this chapter, we also computed the fraction of
in-jet electrons over isolated electrons as a function of the top quark $p_T$,
at truth-level, in order to assess the significance of the number of electrons
that are found inside jets. The measurement used PowhegPythia $t\bar{t}$ events
simulated at $13$ TeV centre-of-mass energy (Chapter~\ref{c:susys},
Section~\ref{mbdatasm}), where dilepton events consisting of a muon and an
electron were selected. The selections made use of the $p_T$-dependent overlap
removal (Chapter~\ref{c:susys}, Section~\ref{mbobjs})

$$ \Delta R <  \text{min (0.4, 0.04 + 10 GeV} / p_T) $$

where the overlap removal is also required to keep the overlapping $b$-jets.
The formula takes advantage of the fact that the higher is the $p_T$ of an
electron or a muon, the closer it is to some jet. It thus ensures that such
electrons or muons that are found inside $\Delta R = 0.4$ are also selected,
provided they are not closer than what their $p_T$ warrant. On the other hand,
$b$-jets found within $\Delta R = 0.2$ of electrons or muons will not be
rejected as is usually done in a standard overlap removal procedure.

The results are shown in Table~\ref{t:injetfraction}, which lists several top
quark $p_T$ along with the number of in-jet electrons, the number of isolated
electrons, all at truth-level, and the fraction of the former over the latter.

\vspace{3mm}


\renewcommand{\arraystretch}{1.15}
\begin{table}
	\centering
	\begin{tabular}{||c c||}
		\hline
		Top quark $p_T$ (GeV) & Fraction  \\ [0.5ex]
		\hline\hline
		\toprule
		$\leq 300$            & $ 8.4\%$  \\
		\hline
		$\leq 425$            & $ 17.2\%$ \\
		\hline
		$\leq 500$            & $ 24.0\%$ \\
		\hline
		$ \leq 650$           & $ 39.0\%$ \\
		\hline
		$ \leq 750$           & $ 49.0\%$ \\
		\hline
		$ \leq 800$           & $ 53.0\%$ \\
		\hline
		$ \leq 900$           & $ 59.0\%$ \\
		\hline
		$ \leq 1000$          & $ 64.0\%$ \\
		\hline
	\end{tabular}

	\caption{The fraction of in-jet electrons over the number of isolated
		electrons, both at truth-level, as a function of the top quark $p_T$. The
		fraction increases and becomes very significant at high top quark $p_T$.}

	\label{t:injetfraction}
\end{table}
\renewcommand{\arraystretch}{1.0}

It is seen that more and more electrons are found inside jets as the top quark
$p_T$ increases. In fact, the number of in-jet electrons becomes quite
significant from $500$ GeV, being approximately $25\%$ there and reaching
nearly $40\%$ at $650$ GeV. If the top quark $p_T$ is allowed to go up to $1$
TeV, the figure is $64\%$. This result supports the fact that different ATLAS
analyses saw considerable increases in signal acceptance as they included
in-jet electrons in their selections.

This chapter develops a method and performs the initial measurements for the
identification efficiencies of electrons found inside $\Delta R=0.4$ of jets.
The measurements for electrons outside jets are done by the ATLAS Egamma
group~\cite{atlaselcid, eleffme}.


%%%%%%%%%%%%%%%%%%%%%%%%%%%%%%%%%%%%%%%%%%%%%%%%%%%%%%%%%%%%
\section{Method}\label{s:eidmet}

In general, to measure the identification (ID) efficiencies of electrons we
start from a sample of reconstructed electrons. The efficiency at a particular
identification operating point (section~\ref{ss:elop}) is defined by the ratio

$$
	\text{ID efficiency} = \frac{\text{The number of identified
			electrons}}{\text{The number of reconstructed electrons}}
$$

\vspace{3mm}

The number of identified electrons, the numerator, is contaminated with
background electrons, and so is the demoninator. As a result, part of the work
to obtain a good sample of electrons inside high $p_T$ jets includes at the
same time adequate background estimation methods, particularly because
background electrons are expected to reside primarily inside jets.


In this thesis, a sample of electrons was obtained by selecting boosted
$t\bar{t}$ dilepton ($e\mu$) events, described below in
Section~\ref{ss:eidtp},~\ref{ss:eidsp}, and ~\ref{ss:eidsr}. Background
estimations will be described in Section~\ref{s:eidbe}.

\subsection{Boosted Dilepton $e\mu$ Events}\label{ss:eidtp}

The standard method for measuring electron identification efficiencies is the
tag-and-probe method supported by the Egamma group at ATLAS~\cite{eleffme}.
Such a measurement uses $Z\to e^+e^-$ events for high-energy electrons
($E_{\text{T}} > 10$ GeV). The method is well-documented, and it is
straight-forward to obtain a clean sample of electrons. It is expected,
however, that because of the nature of the events, requiring electrons to be
inside high $p_T$ jets would lead to a sample unrepresentative of events with a
boosted topology, and moreover the resulting sample would be statistics-limited
also.


In this thesis, $t\bar{t}$ events at $13$ TeV centre-of-mass are used to select
electrons inside high $p_T$ jets. In particular, dilepton $e\mu$ events will be
selected; they are expected to be a source of reconstructed electrons that will
be used for the measurement of the identification efficiencies. The selections
will be such that the muon in an event will be a hard muon, while the electron
on the other hand will remain almost untouched, i.e. it will be a reconstructed
electron with a $p_T$ cut applied.

Compared to $Z\to ee$ events, such $t\bar{t}$ events are expected to provide
more statistics for electrons inside high $p_T$ jets. In addition, because
they directly involve the top quarks, they will bear close topology to many
SUSY and other beyond-the-Standard-Model searches.

The remaining detail of the method is described below, including a discussion
of the data as well as the simulations used, the definition of the signal
region, and background subtractions.


%%%%%%%%%%%%%%%%%%%%%%%%%%%%%%%%%%%%%%%%%%%%%%%%%%%%%%%%%%%%
\subsection{Data and Monte Carlo Samples}\label{ss:eidsp}

The data used for this chapter was collected in the period 2015-2016. The
integrated luminosity was $36.47~\text{fb}^{-1}$. The following simulation
samples, all at $13$ TeV centre-of-mass, are used (Chapter~\ref{c:susys},
Section~\ref{mbdatasm}):

\begin{itemize}[label=\ding{109}]

	\item PowhegPythia $t\bar{t}$. This set is used for selecting electrons inside
	      high $p_T$ jets. As will be discussed in Section~\ref{ss:eidsr}, $t\bar{t}$
	      events will be partitioned into a semileptonic set (a muon from one top quark
	      and all jets from the other top) and a dileptonic set ($e\mu$ events). The
	      former will constitute a background and the latter a source of signal events.
	      Indeed, in the semileptonic case, jets from one of the tops constitute a source
	      of background electrons

	\item W+jets. This is used as a background. Indeed, as a top quark almost
	      always decays into a $W$ and a $b$-jet, $W$+jets naturally constitutes a
	      background where the $W$ produces a hard muon and the jets are a source of
	      background electrons

	\item Single top. This includes the $Wt$ production as well as the $s$-channel
	      and $t$-channel productions. The $Wt$ production is a source of signal
	      electrons, while the other two productions provide a source of top+jets and are
	      therefore background, as the top quark may be a source of a hard muon and the
	      jets in the events are a source of background electrons.

\end{itemize}
%%%%%%%%%%%%%%%%%%%%%%%%%%%%%%%%%%%%%%%%%%%%%%%%%%%%%%%%%%%%
\subsection{Signal Region}\label{ss:eidsr}

The signal region will be the kinematic region in which the measurement of the
identification efficiencies is performed. Before defining the signal region, we
apply the following preliminary cuts, called pre-selection cuts:

\paragraph{Pre-selection}

\begin{itemize}[label=\ding{111}]
	\item One primary vertex

	\item Muon trigger. The following triggers were used for the periods $2015$
	      and $2016$:

	      \begin{itemize}
		      \item $2015$: \texttt{HLT_mu26_imedium || HLT_mu40}
		      \item $2016$:  \texttt{HLT_mu26_ivarmedium || HLT_mu50}
	      \end{itemize}

	\item $p_T$-dependent overlap removal, with the option of keeping overlapping
	      $b$-jets turned on (Section~\ref{s:eidmot})

	\item Veto events with bad or cosmic muons. Highly energetic jets could reach
	      the muon spectrometer and create hits in the latter, or jet tracks in the inner
	      detector could be erroneously matched to muon spectrometer segments, both cases
	      of which are sources of bad muons. Thus events with these muons, along with
	      those with muons coming from cosmic rays, are removed.

	\item Each event is required to have exactly one tagged muon and $\geq 1$
	      electrons inside jets, where

	      \begin{itemize}

		      \item The muon is required to have $p_T > 30$ GeV, $d_0 / \sigma(d_0) < 3.0$,
		            $z_0 < 0.5$ in terms of the transverse impact paramater and the longitudinal
		            impact parameter, and $\text{ptvarcone}30 / p_T < 0.06$, and be
		            trigger-matched.

		      \item The electrons have $p_T \geq 30$ GeV and must overlap with some jets,
		            i.e. they must be found inside $\Delta R < 0.4$ of jets. There could be more
		            than one electron in the event, however only the leading one in terms of $p_T$
		            will be used as the probe. The $p_T$ cut is applied because it is applied in
		            most analyses where in-jet electrons are actually used.

	      \end{itemize}

	\item $\geq 1$ $b$-tagged jet, instead of exactly $2$ $b$-tagged jets as is
	      usually expected in $t\tbar$ events, since events with electrons inside jets
	      are being selected and consequently $b$-tagging efficiency may suffer because
	      the tracks of the electrons may confuse the $b$-tagging algorithm.

\end{itemize}

After these cuts, we arrive at a set of 3183 events with one hard muon and at
least one electron found inside some jet. Several variables that were found to
be discriminating are listed and discussed below, along with plots of their
distributions. It is seen in the plots that the prominent source of background
events is semileptonic events, whereas $W$+jets and single top $s$-channel and
$t$-channel constitute two small sources of background, predicted by
simulations to be $315.483$ and $19.120$ respectively.

\begin{itemize}[label=\ding{109}]

	\item The mass of the large radius jet that overlaps with the probe electron,
	      shown in Figure~\ref{f:premrjet}. The large radius jet is reclustered from the
	      small radius jets present in the events (Chapter~\ref{c:susys},
	      Section~\ref{mbobjs}) and accordingly in semileptonic events it is expect to be
	      more massive, as it picks up the masses of the jets from the hadronic decay of
	      one of the tops. In dileptonic events, on the other hand, there are fewer jets
	      due to the leptonic decays of both of the tops, and moreover there is also
	      missing energy present from the neutrino accompanying the electron. Therefore
	      this variable is expected to be discriminating. The figure shows a markedly
	      higher distribution for semileptonic events, especially as we move towards the
	      right of the distribution, where dileptonic events are seen less and less and
	      become exceedingly small very quickly. In fact, the right side is mostly
	      background-dominated.


	\item The number of jets, which is shown in Figure~\ref{f:prenjets}. This is a
	      discriminating variable simply because in the case of semileptonic events there
	      are more jets coming from hadronic decay of one of the tops (the other top
	      quark decays leptonically), whereas in dileptonic events the daughter jets of
	      the tops are only the two $b$-jets coming the decays of both of the tops.
	      Accordingly the semileptonic distribution is seen to be higher everywhere.





	\item The sum of the transverse momenta of all jets, shown in
	      Figure~\ref{f:prehtjet}. This is also expected to be discriminating because
	      again more jets are found in semileptonic events due to hadronic decay of one
	      of the tops, and fewer jets in dileptonic events due to leptonic decays of both
	      tops, and as a result a sum over all transverse momenta of the jets leads to a
	      discriminating distribution. Indeed, the figure shows that the semileptonic
	      distribution is higher everywhere.




	\item The transverse momenta of the jet closest to the probe, which is shown in
	      Figure~\ref{f:preptcjet}. It is expected that jets having electrons inside
	      $\Delta R = 0.4$ of themselves tend to be high-$p_T$ jets, as the selections
	      are targeting boosted top quark decays. The figure not only shows that the
	      semileptonic distribution is higher everywhere, but that there is also a class
	      of very low $p_T$ jets closest to the probes.





	\item The fraction of the transverse momentum of the probe over that of the
	      closest jet (Figure~\ref{f:preptfractioncjet}). This variable is discriminating
	      because it is expected that real electrons coming directly from the $W$'s which
	      are produced from the tops tend to have higher $p_T$ than background electrons
	      The figure shows indeed that the low $p_T$ region is dominated by semileptonic
	      events.

\end{itemize}

\begin{figure}[H]
	\includegraphics[width=10cm]{figures/nvariables_pt40_probes_pre_m0_pt100_mar05_loose_real_loose_fake_m_rjet_overlap_el_mar22}
	\centering
	\caption{$m_{\text{rjet}}^{\text{el}}$. The semileptonic contribution is higher everywhere, especially on the right side of the distribution where
		there is little signal contamination.}
	\label{f:premrjet}
\end{figure}

\begin{figure}[H]
	\includegraphics[width=10cm]{figures/nvariables_pt40_probes_pre_m0_pt100_mar05_loose_real_loose_fake_njets_njets_mar22}
	\centering
	\caption{The number of jets. The semileptonic contribution is higher because of the hadronic decay of one of the tops.}
	\label{f:prenjets}
\end{figure}

\begin{figure}[H]
	\includegraphics[width=10cm]{figures/nvariables_pt40_probes_pre_m0_pt100_mar05_loose_real_loose_fake_ht_jet_ht_jet_mar22}
	\centering
	\caption{The sum of the transverse momenta of all jets.}
	\label{f:prehtjet}
\end{figure}


\begin{figure}[H]
	\includegraphics[width=10cm]{figures/nvariables_pt40_probes_pre_m0_pt100_mar05_loose_real_loose_fake_pt_c_jet_pt_c_jet_mar22}
	\centering
	\caption{$p_T$ of the jet closest to the probe electron. }
	\label{f:preptcjet}
\end{figure}


\begin{figure}[H]
	\includegraphics[width=10cm]{figures/nvariables_pt40_probes_pre_m0_pt100_mar05_loose_real_loose_fake_pt_fraction_c_jet_pt_fraction_c_jet_mar22}
	\centering
	\caption{Fraction of the $p_T$ of the probe electron over that of the closest jet. The lower $p_T$ region is dominated by semileptonic events.}
	\label{f:preptfractioncjet}
\end{figure}

\paragraph{Further cuts to arrive at the signal region}\label{p:eidfurthercuts}

Of all the discriminating variables shown above, $m_{\text{rjet}}^{\text{el}}$
seems to be the most discriminating variable. In addition, its distribution
shows two distinct regions, one abundant in signal electrons and one largely
dominated by background electrons. As will be discussed later in the chapter,
the region $< 60$ GeV will define the signal region where the identification
efficiencies are measured, while the region $> 60$ GeV will define the control
region for background estimation. With this in mind, we decided to apply cuts
on the other discriminating variables to further remove the undesired
background, while leaving $m_{\text{rjet}}^{\text{el}}$ untouched.

The cuts are as follows:

\begin{itemize}[label=\ding{111}]
	\item MET $> 25$ GeV, to ensure that the QCD multi-jet background is negligible.

	\item The number of jets $< 5$ and sum of $p_T$ of jets $< 700$ GeV, to remove
	      semileptonic events (Figure~\ref{f:prenjets} and~\ref{f:prehtjet}).


	\item $p_T$ of jet closest to the probe is between $150$ GeV and $500$ GeV, to
	      remove semileptonic events (Figure~\ref{f:preptcjet}) and at the
	      same time make sure that boosted $t\tbar$ dilepton events are selected.


	\item $p_T(\text{probe})  / p_T(\text{closest jet}) > 0.16 $ (Figure~\ref{f:preptfractioncjet}).

\end{itemize}

The resulting distrution $m_{\text{rjet}}^{\text{el}}$ is shown in Figure~\ref{f:crmrjet}.

\begin{figure}[H]
	\includegraphics[width=10cm]{figures/n_pt40_probes_cr1_m0_pt100_mar05_loose_real_loose_fake_m_rjet_overlap_el_mar22}
	\centering

	\caption{$m_{\text{rjet}}^{\text{el}}$ after the pre-selection cuts. The
		region $< 60$ GeV will define the signal region, and the region $\geq 60$ GeV
		will define the control region for background estimation.}

	\label{f:crmrjet}
\end{figure}

%%%%%%%%%%%%%%%%%%%%%%%%%%%%%%%%%%%%%%%%%%%%%%%%%%%%%%%%%%%%
\subsection{Background Estimation}\label{s:eidbe}

We are looking to measure the identification efficiency for electrons inside
jets at a particular operating point. If $\epsilon$ denotes such an efficiency,
then it is the ratio of a numerator and a denominator (Section~\ref{s:eidmet}),
both of which are expected to be contaminated with background electrons that
will need to be estimated. We will write $P$ as the number of electron
candidates passing a particular ID operating point, $B_P$ the number of
background electrons passing the operating point, $N$ the total number of
reconstructed electron candidates in the sample, and $B_N$ the number of
background electrons present in the sample, so that the efficiency $\epsilon$
is in fact

\begin{equation}\label{eqn:effo}
	\epsilon = \frac{P-B_P}{N-B_N}
\end{equation}

The identification efficiencies for the Medium and Tight ID operating points
are the mosted widely used points in ATLAS analyses, especially in SUSY
analyses, and accordingly they are the only points that are measured in this
thesis. Thus, either a Medium or a Tight ID selection will be applied on top of
the sample that represents the denominator, thereby the numerator is obtained.
Background estimations will consist of estimating the term $B_P$ separately for
Medium and Tight in the numerator, and estimating the common term $B_N$ in the
denominator.

\paragraph{Estimating $B_P$} We expect background electrons to rarely pass the
Medium or Tight ID points, and as a result we expect the term $B_P$ to be very
small in either case. Thus $B_P$ may be taken directly from simulation without
significantly affecting the measurements.

Indeed, Figure~\ref{f:eidmedium} and ~\ref{f:eidtight} show the $m_{\text{rjet}}^{\text{el}}$
distributions for electrons that pass the Medium and Tight selections. They are
obtained by applying either a Medium or a Tight ID selection on top of the set
of selections that helps to arrive at the definition of the signal region
(Section~\ref{p:eidfurthercuts}). As can be seen, the number of background
electrons predicted by the simulation is indeed small in each case, accouting
for only $0.3\%$ of the total number in the Medium case
(Figure~\ref{f:eidmedium}), and $0.1\%$ in the Tight case (Figure~\ref{f:eidtight}).

\begin{figure}[H]
	\includegraphics[width=10cm]{figures/pt40_probes_cr1_medium_m0_pt100_mar05_medium_real_medium_fake_m_rjet_overlap_el_mar22}
	\centering

	\caption{The distribution of $m_{\text{rjet}}^{\text{el}}$ for electrons passing
		the Medium operating point. Background electrons figure $0.3\%$.}

	\label{f:eidmedium}

\end{figure}


%%%%%%%%%%%%%%%%%%%%%%%%%%%%%%%%%%%%%%%%%%%%%%%%%%%
\paragraph{Estimating $B_N$}

Since the term $B_P$ in Formula~\ref{eqn:effo} will be estimated by simulation
prediction (for the Medium and Tight case separately), it remains to estimate
the common term $B_N$ that represents background contamination from fake
electrons found in $N$. In order to estimate this term, we will use the
background-dominated region $> 60$ GeV in the distribution of
$m_{\text{rjet}}^{\text{el}}$ (Figure~\ref{f:crmrjet}), as well as the set of
electrons that fail the Loose ID selection, which we will call antiloose
electrons from here. Antiloose electrons will be used as a fake template, to be
normalized to the data to predict the the background in the denominator, and we
will therefore denote them by $T$.

\begin{figure}[H]
	\includegraphics[width=10cm]{figures/pt40_probes_cr1_tight_m0_pt100_mar05_tight_real_tight_fake_m_rjet_overlap_el_mar22}
	\centering

	\caption{The distribution of $m_{\text{rjet}}^{\text{el}}$ for electrons
		passing the Medium operating point. Background electrons figure $0.1\%$.}

	\label{f:eidtight}
	\centering
\end{figure}


\begin{figure}[H]
	\includegraphics[width=10cm]{figures/probe_pt40_probes_contam_m0_pt100_mar05_antiloose_m_rjet_overlap_el_mar22}
	\centering
	\caption{The distribution $m_{\text{rjet}}^{\text{el}}$ for electrons that fail
		the Loose ID point, also called antiloose electrons.}
	\label{f:antiloosemrjet}
\end{figure}


The $m_{\text{rjet}}^{\text{el}}$ distribution of $T$ is shown in
Figure~\ref{f:antiloosemrjet}. Simulation predicts it to be made up mostly of
background electrons, in which semileptonic $t\tbar$ dominates, and there is
about $10\%$ of signal electron contamination in the distribution.
Figure~\ref{f:shapebeforeafter} shows $T$ against the set of background part in
$N$, namely $B_N$, both normalized to unity. It shows that $T$ describes very
well the shape of background electrons $B_N$, and therefore it is reasonable to
estimate $B_N$ using $T$. However, instead of using the simulation distribution
we will use the corresponding data distribution, so that in the following $T$
will mean the corresponding distributions of $m_{\text{rjet}}^{\text{el}}$ in
data. This is to ensure that that typical systematic uncertainties associated
with simulation distributions could be avoided, because these systematic
uncertainties are not small and are also often complicated to obtain.

\begin{figure}[H]
	\includegraphics[width=10cm]{figures/w_allbn_probes_m_rjet_overlap_el_m0_pt100_antiloose_m_rjet_overlap_el_mar22}
	\centering
	\caption{The distribution $m_{\text{rjet}}^{\text{el}}$ of $B_N$ against that
		of $T$, normalized to unity. $T$ describes very well $B_N$ and therefore it is
		reasonable to estimate $B_N$ using $T$.}
	\label{f:shapebeforeafter}
\end{figure}


In the actual measurement of the efficiency, we need to take into account the
fact that $T$ itself is contaminated with some signal electrons, as already
mentioned above. Thus, let $\overline{T}$ be $T$ minus this signal
contamination, which we will obtain by taking the distribution of $T$ in data
minus the signal contamination predicted in the simulation distribution of $T$.
Then, the background-dominated region will be used to find a normalization
factor such that

%
$$B_N = \overline{T} \times \text{normalization factor},$$
%

where the normalization factor will be found as follows. To start, to any
quantity in the signal region $\leq 60$ GeV there corresponds a quantity in the
background-dominated region $> 60$ GeV, which will be denoted with a subscript
$>$. Thus to $N$ there corresponds $N_>$, and to $\overline{T}$ corresponds
$\overline{T}_>$. The term $N_>$ is the set of reconstructed electron
candidates in the background-dominated region. In order to use it in the
following we will subtract any signal contamination it may have, the method of
which will be described in the next section, and denote $\overline{N}_>$ to be
the resulting term, i.e. $N_>$ minus the signal contamination. Then

%
$$\text{normalization factor} = \frac{\overline{N}_>}{\overline{T}_>} $$
%

so that

\begin{equation}\label{eid:bne}
	B_N =  \overline{T} \times \frac{\overline{N}_>}{\overline{T}_>}
\end{equation}

\subsection{The Measurements of the Identification Efficiency}\label{s:meffs}

Formula~\ref{eqn:effo}, where $B_P$ will be taken from simulation and $B_N$
evaluated according to Formula~\ref{eid:bne}, leads to the following formula
for the efficiency:

\begin{equation}\label{eid:efff}
	\epsilon = \frac{P-B_P}{N - \overline{T} \times \frac{\overline{N}_>}{\overline{T}_>}}.
\end{equation}


$\overline{T}$ will be the set of antiloose electrons, $T$, minus signal
contamination. As already mentioned, Figure~\ref{f:shapebeforeafter} shows that
there is a $10\%$ signal contamination expected in $T$. This signal
contamination as predicted by simulation will be subtracted from $T$, resulting
in $\overline{T}$. The term $\overline{T}_>$ will be obtained analogously in
the background-dominated region of the $m_{\text{rjet}}^{\text{el}}$
distribution.

On the other hand, signal contamination in $N_>$ (Figure~\ref{f:crmrjet}), from
which $\overline{N}_>$ is obtained, is larger, and to reduce the contribution
from the estimation of this signal contamination to the uncertainty in the
efficiency we will use data directly for signal contamination subtraction.
Specifically, from Figure~\ref{f:eidmedium} and~\ref{f:eidtight} we expect
negligible background electrons after a Medium or Tight ID is applied. This
means $P$, and the corresponding quantity $P_>$ in the background-dominated
region, is expected to be relatively free of background electrons. We will
therefore write

%
$$\overline{N}_> = N_> - P_> / \epsilon $$
%

where the efficiency in~\ref{eid:efff}, which is being measured, is used again
here. The efficiency will be evaluated iteratively, until the change from one
iteration to the next is less than $0.5\%$. The $0.5\%$ will be taken as the
uncertainty due to signal contamination subtraction in $N_>$.

The next section discusses in detail the treatment of statistical and
systematic uncertainties.

%%%%%%%%%%%%%%%%%%%%%%%%%%%%%%%%%%%%%%%%%%%%%%%%%%%%%%%%%%%%
\subsection{Uncertainties}\label{s:eidunc}

The efficiency is accompanied by statistical and systematic uncertainties, both
of which are discussed in the following.

\paragraph{Statistical Uncertainties} According to Formula~\ref{eid:efff}, the
efficiency will be measured according to the formula


$$
	\epsilon = \frac{P-B_P}{N - \overline{T} \times \frac{\overline{N}_>}{\overline{T}_>}}
$$

where

\begin{itemize}
	\item $P$ is the number of electrons that pass Medium or Tight.

	\item $B_P$ is background contamination due to fake electrons in $P$.

	\item $N$ is the set of reconstructed electron candidates, and $\overline{N}_>$
	      the corresponding quantity in the background-dominated region minus signal
	      contamination.

	\item $\overline{T}$ is the set of antiloose electrons minus signal
	      contamination, and $\overline{T}_>$ the corresponding quantity in the
	      background-dominated region.

\end{itemize}

Since $N$ contains $P$, and $\overline{N}_>$ contains $\overline{T}_>$, the
quantities in the formula are not all independent. We may remove the
correlation between $N$ and $P$ by writing $N = P + F$, where $F$ is the set of
electrons that fail a particular ID point. Then

$$
	\epsilon = \frac{P-B_P}{P + F - \overline{T} \times \frac{\overline{N}_>}{\overline{T}_>}}
$$


The correlation between $\overline{N}_>$ and $\overline{T}_>$ remains, and
moreover $F$ and $\overline{T}$ are also correlated, because in the Medium case
or in the Tight case, $F$ represents electrons failing Medium or Tight
respectively, and since $\overline{T}$ represents electrons failing Loose
(minus signal contamination), in each case $\overline{T}$ is a subset of $F$
and there is accordingly a correlation.

In order to remove all the correlations and write the efficiency completely in
terms of statistically independent quantities we will first multiply both the
numerator and the denominator by $\overline{T}_>$, to write

$$
	\epsilon = \frac{(P-B_P)\overline{T}_>}{P\overline{T}_> + F\overline{T}_> - \overline{T}\times \overline{N}_>}.
$$

Then we will add and subtract $\overline{T} \times \overline{T}_>$, to have


\begin{equation*}
	\begin{split}
		\epsilon & = \frac{(P-B_P)\overline{T}_>}{P\overline{T}_> + F\overline{T}_> - \overline{T}\times \overline{T}_>
			+ \overline{T}\times \overline{T}_> - \overline{T}\times \overline{N}_>}  \\
		&  =  \frac{(P-B_P)\overline{T}_>}{P\overline{T}_> + (F - \overline{T})\overline{T}_>
			- (\overline{N}_> - \overline{T}_>)\overline{T}}
	\end{split}
\end{equation*}

The difference $F - \overline{T}$ represents the set of electrons that fail
Medium or Tight but pass the Loose identification, and the difference
$\overline{N}_> - \overline{T}_>$ represents the set of electrons that pass the
Loose identification. If we treat each of the differences as a single term, and
set $S = F - \overline{T}$ and $\overline{R}_> = \overline{N}_> -
	\overline{T}_>$ respectively, the efficiency becomes

\begin{equation}\label{eq:effco}
	\epsilon = \frac{(P-B_P)\overline{T}_>}{P\overline{T}_> + S\overline{T}_>
		- \overline{R}_>\times \overline{T}}
\end{equation}

which is now a function of six independent quantities, $\epsilon=\epsilon(P,
	B_P, \overline{T}_>, S, \overline{R}_>, T)$. The statistical uncertainty of the
efficiency then follows the standard error propagation formula,

\begin{equation}\label{eq:statprop}
	\Delta \epsilon^2 = \bigg(\frac{\partial \epsilon}{\partial P}\bigg)^2\Delta P^2 + \cdots +
	\bigg(\frac{\partial \epsilon}{\partial T}\bigg)^2\Delta T^2
\end{equation}


Let $A$ denote the numerator in Formula~\ref{eq:effco} and $B$ the denominator.
The terms in the formula above are then

$$\frac{\partial \epsilon}{\partial P} = \frac{B \overline{T}_> - A \overline{T}_> }{B^2},
	\quad
	\frac{\partial \epsilon}{\partial B_P} = \frac{- B \overline{T}_> }{B^2},
	\quad
	\frac{\partial \epsilon}{\partial \overline{T}_>} = \frac{B (P - B_P) - A(P + S) }{B^2},
$$

$$\frac{\partial \epsilon}{\partial S} = \frac{- A \overline{T}_> }{B^2},
	\quad
	\frac{\partial \epsilon}{\partial \overline{R}_>} = \frac{AT }{B^2},
	\quad
	\frac{\partial \epsilon}{\partial \overline{T}} = \frac{A \overline{R}_> }{B^2}.
$$

The statistical uncertainty contribution due to the term $B_P$ will be treated
as negligible, as this term comes from simulation. Then, among the remaining
terms, only $P$ and $S$ are present in the signal region that are not used for
background estimation, and as a result the statistical uncertainty of the
efficiency will be taken from the contributions of these two terms. The
contributions to the uncertainty from other terms will be taken as
contributions to the total systematic uncertainty, which is discussed below.


%%%%%%%%%%%%%%%%%%%%%%%%%%%%%%%%%%%%%%%%%%%%%%%%%%%%%%%%%%%%%%%%%%%%%%%%%%%%%%%%%%%%%%
%%%%%%%%%%%%%%%%%%%%%%%%%%%%%%%%%%%%%%%%%%%%%%%%%%%%%%%%%%%%%%%%%%%%%%%%%%%%%%%%%%%%%%
%%%%%%%%%%%%%%%%%%%%%%%%%%%%%%%%%%%%%%%%%%%%%%%%%%%%%%%%%%%%%%%%%%%%%%%%%%%%%%%%%%%%%%

\paragraph{Systematic Uncertainties} Contributions to the total systematic
uncertainty from different sources will be added in quadrature. The sources are
listed below.

\begin{itemize}[label=\ding{111}]


	\item The variation of the signal region, i.e. instead of marking the signal
	      region at $60$ GeV, we may mark it at $50$ or $80$ GeV, the asymmetry is due to
	      the fact that signal distributions on both sides of the $60$ GeV mark are not
	      equal in equal intervals.


	\item The term $B_P$ which represents the background contamination in $P$ is
	      taken from simulation and may be varied up and down. To be conservative, we
	      have decided to make a $50\%$ variation.


	\item The uncertainty due to signal contamination subtraction from $T$ and
	      $T_>$, from which result $\overline{T}$ and $\overline{T}_>$, may be obtained
	      by conservatively varying the signal contamination $25\%$ up and down.

	\item The template $T$, which is the distribution of antiloose electrons, may
	      be replaced by the distribution of antiloose electrons in events with exactly 2
	      $b$-jets.


	\item In addition, the contributions from the counting of $\overline{T}_>$,
	      $\overline{R}_>$, and $\overline{T}$ in Formula~\ref{eq:effco} are treated as
	      contributions to the total systematic uncertainty as well.


\end{itemize}


%%%%%%%%%%%%%%%%%%%%%%%%%%%%%%%%%%%%%%%%%%%%%%%%%%%%%%%%%%%%%%%%%%%%%%%%%%%%%%%%%%%%%%%%%%%%
%%%%%%%%%%%%%%%%%%%%%%%%%%%%%%%%%%%%%%%%%%%%%%%%%%%%%%%%%%%%%%%%%%%%%%%%%%%%%%%%%%%%%%%%%%%%
%%%%%%%%%%%%%%%%%%%%%%%%%%%%%%%%%%%%%%%%%%%%%%%%%%%%%%%%%%%%%%%%%%%%%%%%%%%%%%%%%%%%%%%%%%%%
%%%%%%%%%%%%%%%%%%%%%%%%%%%%%%%%%%%%%%%%%%%%%%%%%%%%%%%%%%%%%%%%%%%%%%%%%%%%%%%%%%%%%%%%%%%%

\section{Identification Efficiencies}\label{s:eideff}

In this section the integrated efficiencies as well as the binned efficiencies
for the Medium and Tight operating points are presented, along with the
associated uncertainties.


\subsection{Integrated Efficiencies}

The identification efficiencies for electrons inside jets are measured for the
Medium and Tight operating points; they are evaluated iteratively according to
Formula~\ref{eid:efff}, which is

$$
	\epsilon = \frac{P-B_P}{N - \overline{T} \times \frac{\overline{N}_>}{\overline{T}_>}}.
$$

where

$$\overline{N}_> = N_> - P_> / \epsilon $$

The integrated efficiencies are presented in the following, along with the
associated uncertainties.

\subsubsection{The efficiencies}

The efficiencies, as well as the total statistical and systematic
uncertainties, are listed in Table~\ref{t:inteffss}. It is seen that the
efficiency is higher for Medium than for Tight, consistent with expectation.
The statistical uncertainties are slightly larger for Tight, also consistent
with expectation, as the stats for Tight is slightly less than that for Medium.

\begin{table}[H]
	\centering
	\begin{tabularx}{0.8\textwidth}{| c | *{2}{Y|} }
		\cline{2-3}
		\multicolumn{1}{c |}{}  &  \thead{MEDIUM \\ \textbf{0.870}}   &  \thead{TIGHT \\ \textbf{0.784}} \\[1.0ex]
		\hline\hline
		\toprule
		Statistical Uncertainty & $\pm 0.017$  & $ \pm 0.019$ \\
		\hline\hline
		Systematic Uncertainty  & $\pm 0.031 $ & $ \pm 0.020$ \\
		\hline
	\end{tabularx}
	\caption{Efficiencies, Statistical and Systematic Uncertainties in Data for the
		Medium and Tight operating points.}
	\label{t:inteffss}
\end{table}

The relevant quantities in Formula~\ref{eid:efff} that are used to compute the
efficiencies in data are listed in Table~\ref{t:inteffq}.

\renewcommand{\arraystretch}{1.15}
\begin{table}[H]
	\centering
	\begin{tabularx}{0.8\textwidth}{| c | *{2}{Y|} }
		\cline{2-3}
		\multicolumn{1}{c |}{} & MEDIUM & TIGHT   \\[1.0ex]
		\hline\hline
		\toprule
		~~~~~~$P$~~~~~~        & $ 392$ & $ 356$  \\
		\hline
		$B_P$                  & $1.47$ & $ 0.40$ \\
		\hline
		$N$                    & $734$  & $ 734$  \\
		\hline
		$\thead{\overline{N_>} \\ }$ & $368$ &  $ 368$ \\
		\hline
		$P_>$                  & $49$   & $ 48$   \\
		\hline
		$\thead{\overline{T} \\ }$  & $267.35$ &  $ 267.35$ \\
		\hline
		$\thead{\overline{T_>} \\ }$  & $292.52$ &  $ 292.52$ \\
		\hline
		\toprule
	\end{tabularx}
	\caption{The relevant quantities for computing the efficiencies according to
		Formula~\ref{eid:efff}.}
	\label{t:inteffq}
\end{table}
\renewcommand{\arraystretch}{1}

The efficiencies in simulation for the Medium and Tight operating points
are also computed and are listed in
Table~\ref{t:inteffssmc} along with the statistical uncertainties. The relevant
quantites are displayed in Table~\ref{t:inteffssmcq}.

\begin{table}[H]
	\centering
	\begin{tabularx}{0.8\textwidth}{|*{2}{Y|} }
		\hline
		\thead{MEDIUM \\ \textbf{ 0.871}}   &  \thead{TIGHT \\ \textbf{0.807}} \\[1.0ex]
		\hline\hline
		\toprule
		$\pm 0.010$ & $ \pm 0.011$ \\
		\hline
	\end{tabularx}
	\caption{The integrated efficiencies in simulations for the Medium and Tight
		operating points, along with the associated statistical uncertainties.}
	\label{t:inteffssmc}
\end{table}


\begin{table}[H]
	\centering
	\begin{tabularx}{0.8\textwidth}{| c | *{2}{Y|} }
		\cline{2-3}
		\multicolumn{1}{c |}{} & MEDIUM & TIGHT  \\[1.0ex]
		\hline\hline
		\toprule
		\multicolumn{3}{|c|}{Numerator}\\
		\hline
		Dilepton               & 435.45 & 403.31 \\
		\hline
		Single top signal      & 12.39  & 11.81  \\
		\hline
		\toprule
		\multicolumn{3}{|c|}{Denominator}\\
		\hline
		Dilepton               & 484.53 & 484.53 \\
		\hline
		Single top background  & 29.66  & 29.66  \\
		\toprule
	\end{tabularx}
	\caption{The relevant quantities for computing the efficiencies in simulations for
		the Medium and Tight operating points.}
	\label{t:inteffssmcq}
\end{table}


\subsubsection{Statistical Uncertainties}

As has been discussed in Section~\ref{s:eidunc}, when evaluating the
efficiencies in data, the quantities in the signal region that are not used for
background estimation are $P$ and $S$, and the statistical uncertainty of the
efficiency is taken from the contributions of these two terms, computed
according to Formula~\ref{eq:effco} and shown in Table~\ref{t:inteffss}. Of the
two, the contribution from $S$ is the dominant one; the contribution from $P$
is small ($< 0.5\%$).

\subsubsection{Systematic Uncertainties}

The total systematic uncertainty receives contributions from different sources,
as discussed in Section~\ref{s:eidunc}. The individual contributions are shown
below.

\paragraph{Contributions from $\overline{T}_>$, $\overline{R}_>$, and
	$\overline{T}$} The contributions to the total systematic uncertainty that come
from the counting of $\overline{T}_>$, $\overline{R}_>$, and $\overline{T}$ in
Formula~\ref{eq:effco} are listed in Table~\ref{t:statsources} below. They each
contributes $< 1\%$ to the total uncertainty. The relevant quantities used for
the calculations are listed in Table~\ref{t:syssourcesstats}. The term
$\overline{R}_>$ is computed as the difference $\overline{N}_> -\overline{T}_>$
(Section~\ref{s:eidunc}), where $\overline{N}_> = N_> - P_> /\epsilon $ as
discussed in Section~\ref{s:meffs}.

\renewcommand{\arraystretch}{1.20}
\begin{table}[H]
	\centering
	\begin{tabularx}{0.8\textwidth}{| c | *{2}{Y|} }
		\cline{2-3}
		\multicolumn{1}{c |}{}                          & MEDIUM      & TIGHT        \\[1.0ex]
		\hline\hline
		\toprule
		$\Delta \overline{R}_> = \sqrt{\overline{R}_>}$ & $\pm 0.008$ & $ \pm 0.006$ \\
		\hline
		$\Delta \overline{T} = \sqrt{\overline{T}}$     & $\pm 0.002$ & $ \pm 0.001$ \\
		\hline
		$\Delta \overline{T}_> = \sqrt{\overline{T}_>}$ & $\pm 0.002$ & $ \pm 0.001$ \\
		\hline
	\end{tabularx}
	\caption{Contributions to the total systematic uncertainty from the individual
		sources.}
	\label{t:statsources}
\end{table}
\renewcommand{\arraystretch}{1.0}


\begin{table}[H]
	\centering
	\begin{tabularx}{0.8\textwidth}{| c |*{2}{Y|} }
		\cline{2-3}
		\multicolumn{1}{c |}{}       & \thead{MEDIUM } & \thead{TIGHT } \\[1.0ex]
		\hline\hline
		\toprule
		~~~~~~$\overline{R}_>$~~~~~~ & 19.11           & 14.24          \\
		\hline
		$\overline{T} $              & 267.35          & 267.35         \\
		\hline
		$\overline{T}_>$             & 292.52          & 292.52         \\
		\hline
	\end{tabularx}
	\caption{The quantities $\overline{R}_>$, $\overline{T}_>$, and $\overline{T}$
		that are used to compute the contributions of these terms to the total
		systematic uncertainty according to Formula~\ref{eq:effco}. The term
		$\overline{R}_>$ is computed as the difference $\overline{N}_> -
			\overline{T}_>$ (Section~\ref{s:eidunc}).}
	\label{t:syssourcesstats}
\end{table}


%%%%%%%%%%%%%%%%%%%%%%%%%%%%%%%%%%%%%%%%%%%%%%
\paragraph{Contribution from $B_P$}

The contribution to the total systematic uncertainty from varying the term
$B_P$ (Formula~\ref{eq:effco}) is shown below in Table~\ref{t:syssourcesbp}.
The total contribution is taken to be the sum in quadrature of the two
variations. The countings of $B_P$ in each of the variations are shown in
Table~\ref{t:syssourcesb}. We see that the contribution from $B_P$ is very
small.

\begin{table}[H]
	\centering
	\begin{tabularx}{0.8\textwidth}{| c | *{2}{Y|} }
		\cline{2-3}
		\multicolumn{1}{c |}{}   & MEDIUM & TIGHT \\[1.0ex]
		\hline\hline
		\toprule
		$B_P\times 1.5$          & 0.002  & 0.000 \\
		\hline
		$B_P\div 1.5$            & 0.001  & 0.000 \\
		\hline
		\toprule
		$B_P$ total contribution & 0.002  & 0.000 \\
		\hline
	\end{tabularx}
	\caption{Contributions to the total systematic uncertainty from the term $B_P$.}
	\label{t:syssourcesbp}
\end{table}

\begin{table}[H]
	\centering
	\begin{tabularx}{0.8\textwidth}{| l |*{2}{Y|} }
		\cline{2-3}
		\multicolumn{1}{c |}{} & \thead{MEDIUM } & \thead{TIGHT } \\[1.0ex]
		\hline\hline
		\toprule
		Semileptonic $t\tbar$  & 1.13            & 0.40           \\
		\hline
		Background single top  & 0.34            & 0.00           \\
		\hline
		W+jets                 & 0.00            & 0.00           \\
		\toprule
		\hline
		$B_P\times 1.5$        & 2.2             & 0.61           \\
		\hline
		$B_P\div 1.5$          & 0.98            & 0.27           \\
		\hline
		%\toprule 
		%Uncertainty &  &   \\
		\hline
	\end{tabularx}
	\caption{Contributions to the systematic uncertinty from $B_P$.}
	\label{t:syssourcesb}
\end{table}

%%%%%%%%%%%%%%%%%%%%%%%%%%%%%%%%%%%%%%%%%%%%%%%%%

\paragraph{Contribution from signal contaminations in $T$ and $T_>$} The
contributions due to signal comtamination in the terms $T$ and $T_>$ are listed
in Table~\ref{t:syssourcestn}. These are a major sources of contribution to the
total systematic uncertainty. The signal contaminations in the two terms $T$
and $T_>$ are varied simultaneously, either up or down by $25\%$. The countings
obtained from the variations are shown in Table~\ref{t:syssourcest}.

\begin{table}[H]
	\centering
	\begin{tabularx}{0.8\textwidth}{| c | *{2}{Y|} }
		\cline{2-3}
		\multicolumn{1}{c |}{} & MEDIUM & TIGHT \\[1.0ex]
		\hline\hline
		\toprule
		$\times 1.25$          & 0.014  & 0.012 \\
		\hline
		$\div 1.25$            & 0.010  & 0.010 \\
		\toprule
		\hline
		Total contribution     & 0.017  & 0.015 \\
		\hline
	\end{tabularx}
	\caption{}
	\label{t:syssourcestn}
\end{table}



\begin{table}[H]
	\centering
	\begin{tabularx}{0.8\textwidth}{| c |*{2}{Y|} }
		\cline{2-3}
		\multicolumn{1}{c |}{}          & \thead{MEDIUM } & \thead{TIGHT } \\[1.0ex]
		\hline\hline
		\toprule
		$T$                             & 312             & 312            \\
		\hline
		Dilepton contamination          & 28.43           & 28.43          \\
		\hline
		Single top signal contamination & 16.23           & 16.23          \\
		\hline
		$\overline{T} \times 1.5$       & ?               & ?              \\
		\hline
		$\overline{T} \div 1.5$         & ?               & ?              \\
		\hline\hline
		\toprule
		$T_>$                           & 317             & 317            \\
		\hline
		Dilepton contamination          & 5.45            & 5.45           \\
		\hline
		Single top signal contamination & 19.03           & 19.03          \\
		\hline
		$\overline{T}_> \times 1.5$     & ?               & ?              \\
		\hline
		$\overline{T}_> \div 1.5$       & ?               & ?              \\
		\hline
	\end{tabularx}
	\caption{}
	\label{t:syssourcest}
\end{table}

%%%%%%%%%%%%%%%%%%%%%%%%%%%%%%%%%%%%%%%%%%%%

\paragraph{Contribution from changing the template to antiloose+2b} The
contribution to the total systematic uncertainty from the template change is
shown in Table~\ref{t:syssources2b}. The template change leads to changes in
quantities $\overline{T}$ and $\overline{T}_>$, which are listed in
Table~\ref{t:syssources2bc}.

\begin{table}[H]
	\centering
	\begin{tabularx}{0.8\textwidth}{| c | *{2}{Y|} }
		\cline{2-3}
		\multicolumn{1}{c |}{} & MEDIUM & TIGHT \\[1.0ex]
		\hline\hline
		\toprule
		antiloose+2b           & 0.008  & 0.007 \\
		\hline
	\end{tabularx}
	\caption{Contributions to the total systematic uncertainty from changing the template to
		antiloose+2b}
	\label{t:syssources2b}
\end{table}

\begin{table}[H]
	\centering
	\begin{tabularx}{0.8\textwidth}{| l |*{2}{Y|} }
		\cline{2-3}
		\multicolumn{1}{c |}{}       & \thead{MEDIUM } & \thead{TIGHT } \\[1.0ex]
		\hline\hline
		\toprule
		~~~~~~$\overline{T}$~~~~~~   & 106.03          & 106.03         \\
		\hline
		~~~~~~$\overline{T}_>$~~~~~~ & 114.49          & 114.49         \\
		\hline
	\end{tabularx}
	\caption{}
	\label{t:syssources2bc}
\end{table}

\paragraph{Contributions from re-marking the signal region} The contribution to
the total systematic uncertainty from re-marking the signal region is shown in
Table~\ref{t:syssourcessn}. It is another major contribution to the total
systematic uncertainty. The signal region is marked, in place of at $60$ GeV,
at 50 GeV and 80 GeV in turn. The total contribution is computed as a sum of
quadrature of the two individual contributions. The relevant quantities used
for the calculations of the efficiencies are listed in Table~\ref{t:inteffq50}
and~\ref{t:inteffq80}.

\begin{table}[H]
	\centering
	\begin{tabularx}{0.8\textwidth}{| c | *{2}{Y|} }
		\cline{2-3}
		\multicolumn{1}{c |}{} & MEDIUM & TIGHT \\[1.0ex]
		\hline\hline
		\toprule
		50 GeV                 & 0.010  & 0.004 \\
		\hline
		80 GeV                 & 0.020  & 0.001 \\
		\toprule
		\hline
		Total contribution     & 0.022  & 0.010 \\
		\hline
	\end{tabularx}
	\caption{Contributions to the total systematic uncertainty re-marking the signal
		region at 50 GeV and 80 GeV.}
	\label{t:syssourcessn}
\end{table}

\renewcommand{\arraystretch}{1.15}
\begin{table}[H]
	\centering
	\begin{tabularx}{0.8\textwidth}{| c | *{2}{Y|} }
		\cline{2-3}
		\multicolumn{1}{c |}{}  & MEDIUM & TIGHT   \\[1.0ex]
		\hline\hline
		\toprule
		~~~~~~~~~~$P$~~~~~~~~~~ & $ 350$ & $ 314$  \\
		\hline
		$B_P$                   & $1.47$ & $ 0.40$ \\
		\hline
		$N$                     & $593$  & $ 593$  \\
		\hline
		$\thead{\overline{N_>} \\ }$ & $509$ &  $ 368$ \\
		\hline
		$P_>$                   & $91$   & $ 90$   \\
		\hline
		$\thead{\overline{T} \\ }$  & $182.76$ &  $ 182.76$ \\
		\hline
		$\thead{\overline{T_>} \\ }$  & $377.11$ &  $ 377.11$ \\
		\hline
		\toprule
	\end{tabularx}

	\caption{The relevant quantities to compute the efficiencies for the Medium and
		Tight operating points when marking the signal region at 50 GeV.}

	\label{t:inteffq50}

\end{table}
\renewcommand{\arraystretch}{1.0}


\renewcommand{\arraystretch}{1.15}
\begin{table}[H]
	\centering
	\begin{tabularx}{0.8\textwidth}{| c | *{2}{Y|} }
		\cline{2-3}
		\multicolumn{1}{c |}{}  & MEDIUM & TIGHT   \\[1.0ex]
		\hline\hline
		\toprule
		~~~~~~~~~~$P$~~~~~~~~~~ & $ 420$ & $ 384$  \\
		\hline
		$B_P$                   & $1.56$ & $ 0.43$ \\
		\hline
		$N$                     & $734$  & $ 734$  \\
		\hline
		$\thead{\overline{N_>} \\ }$ & $238$ &  $ 238$ \\
		\hline
		$P_>$                   & $21$   & $ 20$   \\
		\hline
		$\thead{\overline{T} \\ }$  & $355.54$ &  $ 355.54$ \\
		\hline
		$\thead{\overline{T_>} \\ }$  & $204.33$ &  $ 204.33$ \\
		\hline
		\toprule
	\end{tabularx}

	\caption{The relevant quantities to compute the efficiencies for the Medium and
		Tight operating points when marking the signal region at 80 GeV.}

	\label{t:inteffq80}

\end{table}
\renewcommand{\arraystretch}{1.0}

\subsection{Efficiencies in Bins}

In addition to the integrated efficiencies, binned efficiencies are also
measured, to check the possible dependencies of the efficiencies on certain
variables. The variables and their associated binnings are:

\begin{itemize}[label=\ding{109}]
	\item $p_T$ of the probe, in five bins
	      \begin{itemize}
		      \item 30-60 GeV,
		      \item 60-80 GeV,
		      \item 80-110 GeV,
		      \item 110-140 GeV,
		      \item $> 140$ GeV.
	      \end{itemize}
	\item $\abs{\eta}$ of the probe, in five bins
	      \begin{itemize}
		      \item 0.0-0.3,
		      \item 0.3-0.6,
		      \item 0.6-0.9,
		      \item 0.9-1.3,
		      \item $> 1.3$.
	      \end{itemize}
	\item $\Delta R $ between the probe and the closest overlapping jet, in five bins
	      \begin{itemize}
		      \item 0.0-0.15,
		      \item 0.15-0.19,
		      \item 0.19-0.23,
		      \item 0.23-0.27,
		      \item 0.27-0.4.
	      \end{itemize}
	\item $p_T$ of the closest overlapping jet, in five bins
	      \begin{itemize}
		      \item 150-220 GeV,
		      \item 220-280 GeV,
		      \item 280-340 GeV,
		      \item 340-400 GeV,
		      \item 400-500 GeV.
	      \end{itemize}
\end{itemize}

The distributions of these variables are shown in Figure~\ref{f:bins01} and~\ref{f:bins02}.
The binned efficiencies are shown in Figure~\ref{f:binspt01} and~\ref{f:binspt02}.


\begin{figure}[H]
	\begin{subfigure}{0.5\textwidth}
		\includegraphics[width=8cm]{figures/nvariables_pt40_probes_cr1_m0_pt100_mar05_loose_real_loose_fake_dr_el_cjet_dr_el_cjet_mar22}
		\label{bfptprobe}
		\caption{$\Delta R $ }
	\end{subfigure}
	\begin{subfigure}{0.5\textwidth}
		\includegraphics[width=8cm]{figures/nvariables_pt40_probes_cr1_m0_pt100_mar05_loose_real_loose_fake_el_eta_el_eta_mar22}
		\caption{$\abs{\eta}$ of the probe}
	\end{subfigure}

	\centering

	\caption{The distributions of $\Delta R $ between the probe and the closest
		overlapping jet and $\abs{\eta}$ of the probe.}

	\label{f:bins01}
\end{figure}

\begin{figure}
	\begin{subfigure}{0.5\textwidth}
		\includegraphics[width=8cm]{figures/nvariables_pt40_probes_cr1_m0_pt100_mar05_loose_real_loose_fake_el_pt_el_pt_mar22}
		\caption{$p_T$ of the probe}
	\end{subfigure}
	\begin{subfigure}{0.5\textwidth}
		\includegraphics[width=8cm]{figures/nvariables_pt40_probes_cr1_m0_pt100_mar05_loose_real_loose_fake_pt_c_jet_pt_c_jet_mar22}
		\caption{$p_T$ of the closest overlapping jet}
	\end{subfigure}

	\centering

	\caption{The distributions of $p_T$ of the probe and $p_T$ of the closest
		overlapping jet.}

	\label{f:bins02}
\end{figure}


\begin{figure}[H]
	\begin{subfigure}{0.5\textwidth}
		\includegraphics[width=8cm]{figures/sep04_bins_probe_pt_medium_tight_dt_mc}
	\end{subfigure}
	\begin{subfigure}{0.5\textwidth}
		\includegraphics[width=8cm]{figures/sep04_bins_probe_eta_medium_tight_dt_mc}
	\end{subfigure}

	\centering
	\caption{The binned efficiencies in $p_T$ of the probe as well as in $\abs{eta}$ of the
		probe.}
	\label{f:binspt01}
\end{figure}

\begin{figure}
	\begin{subfigure}{0.5\textwidth}
		\includegraphics[width=8cm]{figures/sep04_bins_dr_medium_tight_dt_mc}
	\end{subfigure}
	\begin{subfigure}{0.5\textwidth}
		\includegraphics[width=8cm]{figures/sep04_bins_jetpt_medium_tight_dt_mc}
	\end{subfigure}

	\centering
	\caption{The binned efficiencies in $\Delta R$ between the probe and the closest jet,
		as well as in $p_T$ of the closest overlapping jet.}
	\label{f:binspt02}
\end{figure}


\section{Conclusions}\label{s:eidcon}

This chapter describes the work to measure the identification efficiencies for
in-jet electrons. It was the first attempt to perform such a measurement since
Run 2 began. The measurement used the data collected in the period 2015-2016,
at $13$ TeV center-of-mass and totalled $36.47~\text{fb}^{-1}$ in integrated
luminosity. A sample of electrons for the measurements was obtained by
selecting boosted $t\bar{t}$ dilepton ($e\mu$) events. Background estimations
used both simulations and data, and the efficiencies were evaluated
iteratively. The efficiencies were measured for the Medium and Tight operating
points, both on data and simulation. Efficiencies in bins were also measured.
In all of the results, the efficiencies predicted by simulation are quite close
to those obtained from the measurements on data.


