
\chapter{ELECTRON RECONSTRUCTION AND IDENTIFICATION AT ATLAS}\label{c:ereid}

The central theme of this thesis is the improvement of the selection of signal
electrons in SUSY searches. In particular, Chapter~\ref{c:cid} deals with the
estimation of electron charge mis-identification, Chapter~\ref{c:susys}
discusses a supersymmetry seach that involves leptons (electrons or muons) in
the final state, and Chapter~\ref{c:eid} measures the identification
efficiencies for in-jet electrons\footnote{These refer to electrons that are
	found within $\Delta R = 0.4$ of high $p_T$ jets.}. This chapter presents a
more extended discussion of electron reconstruction and identification at
ATLAS~\cite{atlaselcid}.

At ATLAS, a signal electron that has been selected passes through two major
steps, reconstruction and identification. Electron reconstruction, discussed in
Section~\ref{s:ereco}, is the selection, using information from the inner
detector and the electromagnetic calorimeter, of a set of objects which are
called electron candidates. Electron identification, on the other hand, refers
to the selection from a pool of electron candidates; it is discussed in
Section~\ref{s:eid}.

Figure~\ref{f:elcid}\cite{atlaselcid} shows the hypothetical path of an
electron, in red trajectory, through the ATLAS detector. The electron emerges
near the collision point, passes the tracking system (made up of the pixel
detectors, the silicon-strip detectors, and the TRT) before entering the
electromagnetic calorimeter. Also shown in the figure is the path of a photon,
in dashed trajectory, produced by the interaction of the electron with the
material in the tracking system.

\begin{figure}[h]
	\includegraphics[width=12cm]{figures/IdSketch}
	\centering

	\caption{The hypothetical path of an electron through the
		detector~\cite{atlaselcid} is shown in red in the figure. The electron moves
		through the tracking system (pixel detectors, silicon-strip detectors, and the
		TRT) before entering the electromagnetic calorimeter. The dashed red line shows
		the path of a photon that comes from the interaction of the electron with the
		material in the tracking system.}

	\label{f:elcid}

\end{figure}

\section{Electron Reconstruction}\label{s:ereco}

At ATLAS, we expect most electrons passing through the detector to interact
with the material of the latter and lose a significant amount of energy through
bremsstrahlung. These interactions, which may happen along the path of the
electrons, typically cause radiated photons and consequently electron-position
pairs, all of which tend to collimate. Thus, electron reconstruction at ATLAS
consists of three fundamental components:

\begin{itemize}
	\item Localized clusters of energy deposits in the electromagnetic calorimeter;
	\item Tracks in the inner detector;
	\item Matching of tracks to the clusters.
\end{itemize}

These components will be discussed in some detail in the following. The
discussion will focus on what was done before a new electron and photon
reconstruction method, dynamic and topological cell clustering-based, was
introduced at ATLAS~\cite{atlaselscluster}.


\subsection{Seed-cluster reconstruction}

Electromagnetic energy cluster candidates are reconstructed from localized
energy deposits in the electromagnetic calorimeter (Section~\ref{p:emc}) using
an algorithm known as the sliding-window algorithm~\cite{calclusalg}. To this
end, the $\eta\times \phi$ plane of the electromagnetic calorimeter is divided
into a grid of $200\times 256$ elements, also called towers, of size $\Delta
	\eta \times \Delta \phi = 0.025 \times 0.025$. The algorithm starts from
localized energy deposits of size $3 \times 5$ towers in $\eta \times \phi$
where the total transverse energy exceeeds $2.5$ GeV, moving in steps of
$0.025$ in either the $\eta$ or the $\phi$ direction and amassing neighboring
localized energy deposits. These accumulated clusters of energy deposits are
referred to as seed-cluster candidates. In the case where two candidates
overlap within an area of $\Delta \eta\times \Delta \phi = 5 \times 9$ units of
$0.025 \times 0.025$, both will undergo a selection process in which

\begin{itemize}

	\item Only the one that has transverse energy at least $10\%$ higher than the
	      other is kept; or otherwise

	\item Only the one that contains the highest transverse momentum in the central
	      tower is kept.

\end{itemize}

The reconstruction efficiency is found to depend on $\eta$ and on the
transverse energy. Figure~\ref{f:elcidseed} shows the dependency on $E_T$.
The efficiency ranges from $96\%$ at $E_T = 7$ GeV to more than $99\%$
above $E_T = 15$ GeV.


\begin{figure}[h]
	\includegraphics[width=12cm]{figures/fig_02b}
	\centering

	\caption{The reconstruction efficiency relative to reconstructed clusters
		as a function of $E_T$ for $Z\to ee$ events.}

	\label{f:elcidseed}

\end{figure}

\subsection{Track reconstruction}

The interactions of charged particles with the inner detector material created
hits~\cite{eltrackhits} in the latter. In track reconstruction, these hits are
assembled into clusters in the pixel and SCT detectors, from which
three-dimensional measurements called space-points are built. In the
silicon-detector layers, sets of three space-points are used to form track
seeds. Then a pattern-recognition algorithm proceeds to build track candidates,
in which energy loss of a particle due to its interactions with the detector
material is modelled assuming the particle is a pion. A modified
pattern-recognition will be used in the case where a track seed having $p_T >
	1$ GeV cannot be extended to a full track of at least seven hits per track
candidate and the associated electromagnetic calorimeter cluster satisfies
shower width and depth requirements. The modified algorithm allows up to $30\%$
energy loss for bremsstrahlung at each intersection of the track with the
detector material.

Track candidates with $p_T > 400$ MeV are fit using the ATLAS Global $\chi^2$
Track Fitter \cite{globalchisqfit}, taking into account which
pattern-recognition algorithm was used. Ambiguities arising from track
candidates sharing hits are also resolved in this step.
Figure~\ref{f:elcidtrackeff} shows that the reconstruction efficiency ranges
from $80\%$ at $E_T=1$ GeV to more than $98\%$ above $E_T = 10$ GeV.

An additional fit, using the Gaussian-sum filter (GSF)~\cite{gsffit} method to
better model energy loss of the particle, is applied on tracks having at least
four silicon hits and that are loosely matched to electromagnetic clusters. The
method takes into account non-linear effects related to bremsstrahlung and
models experimental noise by a sum of Gaussian functions.


\begin{figure}[H]
	\includegraphics[width=12cm]{figures/fig_02a}
	\centering

	\caption{The total reconstruction efficiency for simulated electrons in a single-electron
		sample as a function of the true generator $E_T$ for each step in the reconstruction
		process: $\Delta \eta\times \Delta \phi=3\times 5$ seed-cluster reconstruction (red triangles),
		seed-track reconstruction using the Global $\chi^2$ Track Fitter (blue open circles), both
		steps but using GST tracking (yellow squares), and the final reconstructed electron
		candidate (black closed circles).}

	\label{f:elcidtrackeff}

\end{figure}


\subsection{Electron-candidate reconstruction}

In this final step, the GSF-track candidate is matched to the candidate
calorimeter seed cluster and the final cluster size is determined. If during
the matching procedure several tracks may be matched to a same cluster then an
algorithm using such information as the number of hits in the silicon
detectors, the number of hits in the innermost sillicon layer, and others, is
applied to select out the primary track. The resulting object is called an
electron candidate if it has an associated track with at least four hits in the
silicon layers and no association with a vertex from photon conversion. If on
the other hand its primary track can be matched to a secondary vertex and has
no pixel hits, the object is classified as a photon candidate instead.

Subsequently the candidate electron undergoes an additional classification ---
mainly to keep a high photon-reconstruction efficiency --- to determine if it
still should be considered as a potential photon candidate. The classification
uses the candidate electron's $E/p$ and $p_T$, the presence of a pixel hit, and
the secondary-vertex information.

The energy of the final electron candidate is computed from the calibrated
energy of the extended-window cluster, which is formed from the original seed
cluster by expanding the size of the latter of $\phi$ or $\eta$. The
calibration uses multivariate techniques~\cite{elclusecal1, elclusecal2}.

Figure~\ref{f:elcidtrackeff} and~\ref{f:elcidreceff} show the reconstruction
efficiency as a function of $E_T$ and as a function of $\eta$ in bins of $E_T$,
respectively, from $Z\to e^+e^-$ events. They show that for $E_T > 15$ GeV, the
reconstruction efficiency varies from approximately $97\%$ to $99\%$. Moreover,
simulation efficiency is lower than data efficiency in the low $E_T$ region
($E_T < 30$ GeV) but is higher in the higher $E_T$ region ($E_T > 30$ GeV).

\begin{figure}[H]
	\includegraphics[width=7cm]{figures/c4fig_04a}
	\includegraphics[width=7cm]{figures/c4fig_04b}


	\includegraphics[width=7cm]{figures/c4fig_04c}
	\includegraphics[width=7cm]{figures/c4fig_04d}

	\centering

	\caption{Reconstruction efficiencies relative to reconstructed clusters
		evalued in the 2015-2016 dataset (closed points) and in simulation (open
		points) and the ratios between the two in $Z\to ee$ events. The efficiencies
		are shown as a function of $\eta$ in four $E_T$ bins: 15-20 GeV (top left),
		25-30 GeV (top right), 40-45 GeV (bottom left), and 80-150 GeV (bottom
		right).}

	\label{f:elcidreceff}

\end{figure}


\section{Electron Identification}\label{s:eid}

Electron candidates coming out of reconstruction consist mostly of background
electrons that are made up of hadrons, electrons from photon conversions, and
electrons from heavy-flavor decays. Electron identification is the step whereby
these background electrons are reduced.


\subsection{Likelihood Identification}

Prompt electrons\footnote{These refer to electrons that originate from the
	prompt decays of particles such as $W$, $Z$, and other beyond the Standard
	Model particles.} that enter the central region ($\abs{\eta} < 2.47$) are
selected using a method called the likelihood identification. In this method,
the differences in shower shapes, in track conditions, penetration depth, and
others between prompt electrons and background electrons are analyzed in
detail. Specifically, the following quantities~\cite{atlaselcid}, which are classified into seven
types, are used:

\begin{itemize}
	\item Hadronic leakage:
	      \begin{itemize}
		      \item $R_{\text{had1}}$: Ratio of the transverse momentum in the first layer
		            of the Hadronic Calorimeter to that of the Electromagnetic Calorimeter.

		      \item $R_{\text{had}}$: Ratio of the transverse momentum in the Hadronic Calorimeter
		            to that of the Electromagnetic Calorimeter cluster (used in range $0.8 < \abs{\eta} < 1.37$).

	      \end{itemize}

	\item Third layer of EM calorimeter
	      \begin{itemize}
		      \item $f_3$: Ratio of the energy in the third layer to the total energy in the Electromagnetic
		            Calorimeter (used only for $E_T < 80$ GeV).

	      \end{itemize}
	\item Second layer of EM calorimeter
	      \begin{itemize}
		      \item $\omega_{\eta^2}$: Lateral shower width.

		      \item $R_{\phi}$: Ratio of the energy in $3\times 3$ cells over the energy in $3\times 7$ cells
		            centered at the electron cluster position.

		      \item $R_{\eta}$: Ratio of the energy in $3\times 7$ cells over the energy in $7\times 7$ cells
		            centered at the electron cluster position.
	      \end{itemize}
	\item First layer of EM calorimeter
	      \begin{itemize}
		      \item $\omega_{\text{stot}}$
		      \item $E_{\text{ratio}}$
		      \item $f_1$
	      \end{itemize}
	\item Track conditions
	      \begin{itemize}
		      \item $n_{\text{Blayer}}$: the number of hits in the innermost pixel layer.
		      \item $n_{\text{Pixel}}$: the number of hits in the Pixel detector.
		      \item $n_{\text{Si}}$: the total number of hits in the pixel and SCT detectors.
		      \item $d_0$
		      \item $\abs{d_0/ \sigma(d_0)}$
		      \item $\Delta p / p$
	      \end{itemize}
	\item TRT
	      \begin{itemize}
		      \item eProbabilityHT
	      \end{itemize}
	\item Track-cluster matching
	      \begin{itemize}

		      \item $\Delta\eta_1$: $\Delta \eta$ between the cluster position in the
		            first layer and the extrapolated track

		      \item $\Delta \Phi_{\text{res}}$

		      \item $E/p$: ratio of the cluster energy to the track momentum (for $E_T > 150$ GeV)
	      \end{itemize}
\end{itemize}

These are used as inputs to two likelihood functions, one for signal electrons
and one for background electrons, which take the forms

$$L_S(\xv) = \prod_{i=1}^n P_{S,i}(x_i), \qquad
	L_B(\xv) = \prod_{i=1}^n P_{B,i}(x_i)
$$

respectively. Here $\xv$ is a vector of entries $x_i$ which are the inputs that
correspond to the quantities listed above, each of which has a signal
probability distribution function (pdf) and a background pdf. $P_{S,i}(x_i)$ is
the value of the signal pdf of the quantity $i$ at the value $x_i$, and
likewise $P_{B,i}(x_i)$ is the value of the background pdf. The pdfs are
derived using simulation samples, with corrections applied when discrepancies
with the corresponding data are found. The correlations between the inputs are
neglected.

Then, from each electron candidate discriminant value $d_L$ is computed
according to the formula

$$
	d_L = \frac{L_S}{L_S + L_B}
$$

This discriminant is actually transformed into

\begin{equation}\label{eq:dpl}
	d'_L = -\tau^{-1}\ln (d_L^{-1} - 1)
\end{equation}

which then serves as a quantity to assess if an electron candidate should be
considered a prompt electron. The parameter $\tau$ is set to
$15$~\cite{c4taup}. Figure~\ref{f:elcidllhd} shows a comparison of $d_L$ and
$d'_L$ for prompt electrons from $Z$-boson decays and for background,
illustrating the effective separation between the two.


\subsection{Operating Points}\label{ss:elop}

In general, background rejection and identification efficiency are inversely
related i.e. the higher is one, the lower is the other, and vice versa. In
order to cover various signal efficiencies and background rejection factors as
needed by physics analyses, ATLAS has defined four so-called identification
operating points. They are, in order of increasing background rejection power,
VeryLoose, Loose, Medium, and Tight. All operating points have fixed
requirements on tracking criteria:

\begin{itemize}

	\item Loose, Medium, and Tight: at least two hits in the Pixel detector and a
	      total of seven hits in the pixel and silicon strip detectors combined. To
	      reduce background from photo conversions, Medium and Tight require one of these
	      pixel hits to be in the innermost pixel layer or, if this layer is out of
	      order, the layer immediately after it.

	      There is a variation of the Loose operating point, called LooseAndBLayer, that
	      is the same as Loose except with the addition of the requirement of a hit in
	      the innermost layer also.

	\item  VeryLoose: one hit in the pixel detector, regardless of the layer.
\end{itemize}


A particular value of $d'_L$ (Equation~\ref{eq:dpl}) is defined for each of
them, and in the context of an operating point, electron candidates with
computed $d'_L$ larger than the defined value are considered prompt electrons.
In the likelihood method, candidates that pass an operating point having a
higher background rejection power also pass the operating points having lower
background rejection powers.

\begin{figure}[H]
	\includegraphics[width=12cm]{figures/fig_05}
	\centering

	\caption{The discriminant $d'$ (Formula~\ref{eq:dpl}) for reconstructed
		electron candidates with good quality tracks with 30 GeV $< E_T <$ 35 GeV. The
		black distribution shows prompt electrons in a $Z\to ee$ simulation sample,
		and the red distribution shows background electrons in a generic two-to-two
		process simulation sample.}

	\label{f:elcidllhd}

\end{figure}

Figure~\ref{f:ch4ideffa} shows the efficiencies measured in $J/\psi \to ee$ and
$Z\to ee$ events for data and the corresponding data-to-simulation ratios.
Specifically, the variations of the efficiencies in $E_T$ and $\eta$ for the
Loose, Medium, and Tight operating points are displayed. The efficiency ranges
from $55\%$ at $E_T= 4.5$ GeV to $90\%$ at $E_T=100$ GeV for the Tight point,
and from $85\%$ at $E_T = 20$ GeV to $96\%$ at $E_T = 100$ GeV for the Loose
point.

\begin{figure}[H]
	\includegraphics[width=10cm]{figures/c4fig_08a}
	\includegraphics[width=10cm]{figures/c4fig_08b}
	\centering

	\caption{The likelihood identification efficiencies as functions of $E_T$ and
		$\eta$ in $Z\to ee$ events for Loose, Medium, and Tight (shown in blue, red,
		and black respectively). The data efficiencies are obtained by applying
		data-to-simulation efficiency ratios measured in $J/\psi\to ee$ and $Z\to ee$
		events to $Z\to ee$ simulation.}

	\label{f:ch4ideffa}

\end{figure}


